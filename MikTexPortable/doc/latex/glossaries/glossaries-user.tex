\documentclass[report]{nlctdoc}

\usepackage[inner=0.5in,includemp]{geometry}

\usepackage{array}
\usepackage{alltt}
\usepackage{pifont}
\usepackage[utf8]{inputenc}
\usepackage[T1]{fontenc}
\usepackage{lmodern}
\ifpdf
\usepackage{mathpazo}
\usepackage[scaled=.88]{helvet}
\usepackage{courier}
\fi
\usepackage[colorlinks,
            bookmarks,
            hyperindex=false,
            pdfauthor={Nicola L.C. Talbot},
            pdftitle={User Manual for glossaries.sty},
            pdfkeywords={LaTeX,package,glossary,acronyms}]{hyperref}
\usepackage{xr-hyper}
\usepackage[xindy,nonumberlist,seeautonumberlist,toc,style=altlist]{glossaries}
\usepackage{glossary-inline}

\pagestyle{headings}

\renewcommand*{\glsgroupskip}{}

\renewcommand*{\glsseeformat}[3][\seename]{%
  (\xmakefirstuc{#1} \glsseelist{#2}.)%
}

\renewcommand*{\glossarypreamble}{%
\emph{This glossary style was setup using:}
\begin{ttfamily}
\begin{tabbing}
\cs{usepackage}[\=xindy,\\
\+\>nonumberlist,\\
  seeautonumberlist,\\
  toc,\\
  style=altlist]\{glossaries\}
\end{tabbing}
\cs{renewcommand*}\{\ics{glsgroupskip}\}\{\}\newline
\cs{renewcommand*}\{\ics{glsseeformat}\}[3][\ics{seename}]\{\% \newline
(\ics{xmakefirstuc}\{\#1\} \ics{glsseelist}\{\#2\}.)\}
\end{ttfamily}
}

\ifpdf
\else
  % Need an extra line break in the html version
  % (Don't have time to fiddle with cfg files!)
  \renewcommand*{\glossaryentryfield}[5]{%
    \item[\glsentryitem{#1}\glstarget{#1}{#2}]\mbox{}\newline
      #3\glspostdescription\space #5\newline}%
\fi

\makeglossaries

\renewcommand*{\main}[1]{\hyperpage{#1}}
\newcommand*{\htextbf}[1]{\textbf{\hyperpage{#1}}}
\newcommand*{\itermdef}[1]{\index{#1|htextbf}}

\IndexPrologue{\chapter*{\indexname}
 \markboth{\indexname}{\indexname}}

\newglossaryentry{indexingapp}%
{
  name={Indexing application},
  text={indexing application},
  description={an application (piece of software) separate from
  \TeX/\LaTeX\ that collates and sorts information that has an
  associated page reference. Generally the information is an index
  entry but in this case the information is a glossary entry.
  There are two main indexing applications that are used with \TeX:
  \protect\gls{makeindex} and \protect\gls{xindy}. These are both 
  \gls{cli} applications}
}

\newglossaryentry{cli}
{
  name={Command Line Interface (CLI)},
  first={command line interface (CLI)},
  text={CLI},
  description={an application that doesn't have a graphical user
  interface. That is, an application that doesn't have any windows,
  buttons or menus and can be run in a command
  prompt or terminal (see
  \url{http://www.dickimaw-books.com/latex/novices/html/terminal.html})}
}

\newglossaryentry{xindy}{
name={\appfmt{xindy}\index{xindy|htextbf}},
sort={xindy},
text={\protect\app{xindy}},
description={A flexible \protect\gls{indexingapp} with multilingual
support written in Perl}
}

\newglossaryentry{makeindex}{%
name={\appfmt{makeindex}\index{makeindex|htextbf}},%
sort={makeindex},%
text={\protect\app{makeindex}},%
description={An \protect\gls{indexingapp}},
}

\newglossaryentry{makeglossaries}{%
name={\appfmt{makeglossaries}\index{makeglossaries|htextbf}},%
sort={makeglossaries},%
text={\protect\app{makeglossaries}},%
description={A \styfmt{glossaries} custom designed Perl script interface 
to \gls{xindy} and \gls{makeindex}}
}

\newglossaryentry{makeglossariesgui}{%
name={\appfmt{makeglossariesgui}\index{makeglossariesgui|htextbf}},%
sort={makeglossariesgui},%
text={\protect\app{makeglossariesgui}},%
description={A Java GUI alternative to \gls{makeglossaries} that
also provides diagnostic tools. Home page:
\url{http://www.dickimaw-books.com/apps/makeglossariesgui/}. Also
available on CTAN}
}

\newglossaryentry{numberlist}{%
name={Number list\itermdef{number list}},%
sort={number list},%
text={\protect\term{number list}},%
description={A list of \glslink{entrylocation}{entry locations} (also 
called a location list). The number list can be suppressed using the 
\pkgopt{nonumberlist} package option}
}

\newglossaryentry{entrylocation}{%
name={Entry location\itermdef{entry location}},%
sort={entry location},%
text={\protect\term{entry location}},%
description={The location of the entry in the document. This
defaults to the page number on which the entry appears. An entry may
have multiple locations}
}

\newglossaryentry{locationlist}{%
name={Location list},%
text={location list},
sort={location list},%
description={A list of \glslink{entrylocation}{entry locations}},%
see={numberlist}
}

\newglossaryentry{linktext}{%
name={Link text\itermdef{link text}},
sort={link text},%
text={\protect\term{link text}},
description={The text produced by commands such as \ics{gls}. It may
or may not be a hyperlink to the glossary}
}

\let\glsd\glsuseri
\let\glsation\glsuserii

\newglossaryentry{firstuse}{%
name={First use\ifirstuse},
sort={first use},%
text={first use},%
user1={first used},
description={The first time a glossary entry is used (from the start of the document or after a reset) with one of the
following commands: \ics{gls}, \ics{Gls}, \ics{GLS}, \ics{glspl},
\ics{Glspl}, \ics{GLSpl} or \ics{glsdisp}.
\glsseeformat{firstuseflag,firstusetext}{}\nopostdesc}%
}

\newglossaryentry{firstuseflag}{%
name={First use flag\ifirstuseflag},
sort={first use flag},%
text={first use flag},%
description={A conditional that determines whether or not the entry
has been used according to the rules of \gls{firstuse}. Commands
to unset or reset this conditional are described in 
\sectionref{sec:glsunset}}%
}

\newglossaryentry{firstusetext}{%
name={First use text\ifirstusetext},
sort={first use text},%
text={first use text},%
description={The text that is displayed on \gls{firstuse}, which is
governed by the \gloskey{first} and \gloskey{firstplural} keys of
\ics{newglossaryentry}. (May be overridden by
\ics{glsdisp}.)\nopostdesc}%
}

\newglossaryentry{sanitize}{%
name={Sanitize\itermdef{sanitize}},%
sort={sanitize},
text={\protect\term{sanitize}},%
user1={sanitized\protect\iterm{sanitize}},%
user2={sanitization\protect\iterm{sanitize}},
description={Converts command names into character sequences. That is, a command called,
say, \cs{foo}, is converted into the sequence of characters:
\cs{}, \texttt{f}, \texttt{o}, \texttt{o}. Depending on the font,
the backslash character may appear as a dash when used in the
main document text, so \cs{foo} will appear as: ---foo.
\glspar
When \TeX\ writes information to a file, fragile
commands must be protected. The \gloskey{name}, \gloskey{description},
\gloskey{symbol} and \gloskey{sort} keys all have their values written to a file,
which means that care must be taken if those values contain fragile
commands. There are two approaches: 1) the fragile commands must be
protected using \ics{protect}; 2) the values are sanitized.
Sanitizing the values gets rid of the inconvenience of having to
protect fragile commands, but at the expense of no longer being able
to use those values in the document. Sanitization is governed by the
package option \pkgopt{sanitize} described in \sectionref{sec:pkgopts-general}}
}

\ifpdf
\externaldocument{glossaries}
\fi

\doxitem{Option}{option}{package options}
\doxitem{GlsKey}{key}{glossary keys}
\doxitem{Style}{style}{glossary styles}
\doxitem{Counter}{counter}{glossary counters}

\setcounter{IndexColumns}{2}

\newcommand*{\tick}{\ding{51}}

\newcommand*{\ifirstuse}{\iterm{first use}}
\newcommand*{\ifirstuseflag}{\iterm{first use>flag}}
\newcommand*{\ifirstusetext}{\iterm{first use>text}}

\newcommand*{\firstuse}{\gls{firstuse}}
\newcommand*{\firstuseflag}{\gls{firstuseflag}}
\newcommand*{\firstusetext}{\gls{firstusetext}}


\newcommand*{\istkey}[1]{\appfmt{#1}\index{makeindex=\appfmt{makeindex}>#1=\texttt{#1}|hyperpage}}
\newcommand*{\locfmt}[1]{\texttt{#1}\SpecialMainIndex{#1}}
\newcommand*{\mkidxspch}{\index{makeindex=\appfmt{makeindex}>special characters|hyperpage}}

\newcommand*{\igloskey}[2][newglossaryentry]{\icsopt{#1}{#2}}
\newcommand*{\gloskey}[2][newglossaryentry]{\csopt{#1}{#2}}

\newcommand*{\glostyle}[1]{\textsf{#1}\index{glossary styles:>#1={\protect\ttfamily#1}|main}}

\newcommand*{\samplefile}[1]{\hyperref[ex:sample#1]{\texttt{sample#1.tex}}}

\ifpdf
  \newcommand*{\htmldoc}[2]{\qt{#1}}
\else
  \newcommand*{\htmldoc}[2]{\href{#2.html}{\qt{#1}}}
\fi

\begin{document}
\MakeShortVerb{"}
\DeleteShortVerb{\|}

 \title{User Manual for glossaries.sty v3.04}
 \author{Nicola L.C. Talbot\\%
  \url{http://www.dickimaw-books.com/}}

 \date{2012-11-18}
 \maketitle

\noindent
The \styfmt{glossaries} bundle comes with the following documentation:
\begin{description}
\item[\url{glossariesbegin.pdf}] 
If you are a complete beginner, start with 
\htmldoc{The glossaries package: a guide for
beginners}{glossariesbegin}.

\item[\url{glossary2glossaries.pdf}] 
If you are moving over from the obsolete \sty{glossary} package,
read \htmldoc{Upgrading from the glossary package to the
glossaries package}{glossary2glossaries}.

\item[glossaries-user.pdf]
This document is the main user guide for the \styfmt{glossaries}
package.

\item[\url{mfirstuc-manual.pdf}]
The commands provided by the \sty{mfirstuc} package are briefly
described in \htmldoc{mfirstuc.sty: uppercasing first
letter}{mfirstuc-manual}.

\item[\url{glossaries.pdf}]
Advanced users wishing to know more about the inner workings of all the
packages provided in the \styfmt{glossaries} bundle should read
\qt{Documented Code for glossaries v3.04}.
This includes the documented code for the \sty{mfirstuc} package.

\item[INSTALL] Installation instructions.

\item[CHANGES] Change log.

\item[README] Package summary.

\end{description}

\begin{important}
If you use \sty{hyperref} and \styfmt{glossaries}, you must load
\sty{hyperref} \emph{first}. Similarly the \sty{doc} package must
also be loaded before \styfmt{glossaries}. (If \sty{doc} is loaded,
the file extensions for the default main glossary are changed to
\filetype{gls2}, \filetype{glo2} and \filetype{.glg2} to avoid
conflict with \sty{doc}'s changes glossary.)
\end{important}

\clearpage
\tableofcontents

\clearpage
\printglossaries

 \chapter{Introduction}
\label{sec:intro}

The \styfmt{glossaries} package is provided to assist generating
glossaries. It has a certain amount of flexibility, allowing the
user to customize the format of the glossary and define multiple
glossaries. It also supports acronyms and glossary styles that
include symbols (in addition to a name and description) for glossary
entries. There is provision for loading a database of glossary
terms. Only those terms used\footnote{That is, if the term has been
referenced using any of the commands described in
\sectionref{sec:glslink} and \sectionref{sec:glsadd} or via
\ics{glssee} (or the \gloskey{see} key) or commands such as
\ics{acrshort}.} in the document will be added to the glossary.

\textbf{This package replaces the \sty{glossary} package which is
now obsolete.} Please see the document \qtdocref{Upgrading from the
glossary package to the glossaries package}{glossary2glossaries} for
assistance in upgrading.

One of the strengths of this package is its flexibility, however
the drawback of this is the necessity of having a large manual
that can cover all the various settings. If you are daunted by the
size of the manual, try starting off with the much shorter
\docref{guide for beginners}{glossariesbegin}.

\begin{important}
The \styfmt{glossaries} package comes with a
\href{http://www.perl.org/about.html}{Perl} script called
\gls{makeglossaries}. This provides a convenient interface to the
\glspl{indexingapp} \gls{makeindex} or \gls{xindy}. It is strongly
recommended that you use this script, but \emph{it is not
essential}. If you are reluctant to install Perl, or for any other
reason you don't want to use \gls*{makeglossaries}, you can call
\gls*{makeindex} or \gls*{xindy} explicitly. See
\sectionref{sec:makeglossaries} for further details.
\end{important}
This document uses the \styfmt{glossaries} package. For example,
when viewing the PDF version of this document in a
hyperlinked-enabled PDF viewer (such as Adobe Reader) if
you click on the word ``\gls{xindy}'' you'll be taken to the entry
in the glossary where there's a brief description of
what ``\gls*{xindy}'' is.

The remainder of this introductory section covers the following:
\begin{itemize}
\item \sectionref{sec:samples} lists the sample documents provided 
with this package.

\item \sectionref{sec:languages} provides information for users who
wish to write in a language other than English.

\item \sectionref{sec:makeglossaries} describes how to use a
post-processor to create the sorted glossaries for your document.

\end{itemize}


\section{Sample Documents}
\label{sec:samples}

The \styfmt{glossaries} package is provided with some sample
documents that illustrate the various functions. These should
be located in the \texttt{samples} subdirectory (folder) of the
\styfmt{glossaries} documentation directory. This location varies
according to your operating system and \TeX\ distribution. You
can use \texttt{texdoc} to locate the main glossaries documentation.
For example, in a
\href{http://www.dickimaw-books.com/latex/novices/html/terminal.html}{terminal or command prompt}, type:
\begin{prompt}
texdoc -l glossaries
\end{prompt}
This should display the full pathname of the file 
\texttt{glossaries.pdf}. View the contents of that directory and
see if it contains the \texttt{samples} subdirectory.

If you can't find the sample files, they are available in the
subdirectory \texttt{doc/latex/glossaries/samples/} in the
\texttt{glossaries.tds.zip} archive which can be downloaded from
\href{http://tug.ctan.org/tex-archive/macros/latex/contrib/glossaries/}{CTAN}.


The sample documents are as follows:
\begin{description}
\item[minimalgls.tex]\label{ex:minimalgls} This document is a
minimal working example. You can test your installation using this
file. To create the complete document you will need to do the
following steps:
  \begin{enumerate}
  \item Run \texttt{minimalgls.tex} through \LaTeX\ either by 
  typing
\begin{prompt}
latex minimalgls
\end{prompt}
  in a terminal or by using the relevant button or menu item in
  your text editor or front-end. This will create the required 
  associated files but you will not see the glossary. If you use 
  PDF\LaTeX\ you will also get warnings about non-existent 
  references. These warnings may be ignored on the first run.

  If you get a \verb"Missing \begin{document}" error, then 
  it's most likely that your version of \sty{xkeyval} is 
  out of date. Check the log file for a warning of that nature. 
  If this is the case, you will need to update the \styfmt{xkeyval}
  package.

  \item Run \gls{makeglossaries} on the document (\sectionref{sec:makeglossaries}). This can
  be done on a terminal either by typing
\begin{prompt}
makeglossaries minimalgls
\end{prompt}
  or by typing
\begin{prompt}
perl makeglossaries minimalgls
\end{prompt}
  If your system doesn't recognise the command \texttt{perl} then
  it's likely you don't have Perl installed. In which case you
  will need to use \gls{makeindex} directly. You can do this
  in a terminal by typing (all on one line):
\begin{prompt}
makeindex -s minimalgls.ist -t minimalgls.glg -o minimalgls.gls minimalgls.glo
\end{prompt}
  (See \sectionref{sec:makeindexapp} for further details on using 
   \gls*{makeindex} explicitly.)

  Note that if you need to specify the full path and the path
  contains spaces, you will need to delimit the file names with
  the double-quote character.

  \item Run \texttt{minimalgls.tex} through \LaTeX\ again (as step~1)
  \end{enumerate}
You should now have a complete document. The number following
each entry in the glossary is the location number. By default, this 
is the page number where the entry was referenced.

\item[sample4col.tex]\label{ex:sample4col} This document illustrates
a four column glossary where the entries have a symbol in addition
to the name and description. To create the complete document, you
need to do:
\begin{prompt}
latex sample4col
makeglossaries sample4col
latex sample4col
\end{prompt}
As before, if you don't have Perl installed, you will need to use
\gls{makeindex} directly instead of using
\gls{makeglossaries}. The vertical gap between entries is the
gap created at the start of each group. This can be suppressed
using the \pkgopt{nogroupskip} package option.

\item[sampleAcr.tex]\label{ex:sampleAcr} This document has some
sample acronyms. It also adds the glossary to the table of contents,
so an extra run through \LaTeX\ is required to ensure the document
is up to date:
\begin{prompt}
latex sampleAcr
makeglossaries sampleAcr
latex sampleAcr
latex sampleAcr
\end{prompt}

\item[sampleAcrDesc.tex]\label{ex:sampleAcrDesc} This is similar to
the previous example, except that the acronyms have an associated
description. As with the previous example, the glossary is added to
the table of contents, so an extra run through \LaTeX\ is required:
\begin{prompt}
latex sampleAcrDesc
makeglossaries sampleAcrDesc
latex sampleAcrDesc
latex sampleAcrDesc
\end{prompt}

\item[sampleDesc.tex]\label{ex:sampleDesc} This is similar to the
previous example, except that it defines the acronyms using
\ics{newglossaryentry} instead of \ics{newacronym}. As with the
previous example, the glossary is added to the table of contents, so
an extra run through \LaTeX\ is required:
\begin{prompt}
latex sampleDesc
makeglossaries sampleDesc
latex sampleDesc
latex sampleDesc
\end{prompt}

\item[sample-custom-acronym.tex]\label{ex:sample-custom-acronym}
This document illustrates how to define your own acronym style if
the predefined styles don't suit your requirements.
\begin{prompt}
latex sample-custom-acronym
makeglossaries sample-custom-acronym
latex sample-custom-acronym
\end{prompt}

\item[sample-crossref.tex]\label{ex:sample-crossref}
This document illustrates how to cross-reference entries in the
glossary.
\begin{prompt}
latex sample-crossref
makeglossaries sample-crossref
latex sample-crossref
\end{prompt}

\item[sampleDB.tex]\label{ex:sampleDB} This document illustrates how
to load external files containing the glossary definitions. It also
illustrates how to define a new glossary type. This document has the
\gls{numberlist} suppressed and uses \ics{glsaddall} to add all
the entries to the glossaries without referencing each one
explicitly. To create the document do:
\begin{prompt}
latex sampleDB
makeglossaries sampleDB
latex sampleDB
\end{prompt}
The glossary definitions are stored in the accompanying files
\texttt{database1.tex} and \texttt{database2.tex}. Note that if you
don't have Perl installed, you will need to use \gls{makeindex}
twice instead of a single call to \gls{makeglossaries}:
\begin{enumerate}
\item Create the main glossary:
\begin{prompt}
makeindex -s sampleDB.ist -t sampleDB.glg -o sampleDB.gls sampleDB.glo
\end{prompt}
\item Create the secondary glossary:
\begin{prompt}
makeindex -s sampleDB.ist -t sampleDB.nlg -o sampleDB.not sampleDB.ntn
\end{prompt}
\end{enumerate}

\item[sampleEq.tex]\label{ex:sampleEq} This document illustrates how
to change the location to something other than the page number. In
this case, the \texttt{equation} counter is used since all glossary
entries appear inside an \env{equation} environment. To create
the document do:
\begin{prompt}
latex sampleEq
makeglossaries sampleEq
latex sampleEq
\end{prompt}

\item[sampleEqPg.tex]\label{ex:sampleEqPg} This is similar to the
previous example, but the \glspl{numberlist} are a
mixture of page numbers and equation numbers. This example adds the
glossary to the table of contents, so an extra \LaTeX\ run is
required:
\begin{prompt}
latex sampleEqPg
makeglossaries sampleEqPg
latex sampleEqPg
latex sampleEqPg
\end{prompt}

\item[sampleSec.tex]\label{ex:sampleSec} This document also
illustrates how to change the location to something other than the
page number. In this case, the \texttt{section} counter is used.
This example adds the glossary to the table of contents, so an extra
\LaTeX\ run is required:
\begin{prompt}
latex sampleSec
makeglossaries sampleSec
latex sampleSec
latex sampleSec
\end{prompt}

\item[sampleNtn.tex]\label{ex:sampleNtn} This document illustrates
how to create an additional glossary type. This example adds the
glossary to the table of contents, so an extra \LaTeX\ run is
required:
\begin{prompt}
latex sampleNtn
makeglossaries sampleNtn
latex sampleNtn
latex sampleNtn
\end{prompt}
Note that if you don't have Perl installed, you will need to use
\gls{makeindex} twice instead of a single call to
\gls{makeglossaries}:
\begin{enumerate}
\item Create the main glossary:
\begin{prompt}
makeindex -s sampleNtn.ist -t sampleNtn.glg -o sampleNtn.gls sampleNtn.glo
\end{prompt}
\item Create the secondary glossary:
\begin{prompt}
makeindex -s sampleNtn.ist -t sampleNtn.nlg -o sampleNtn.not sampleNtn.ntn
\end{prompt}
\end{enumerate}

\item[sample.tex]\label{ex:sample} This document illustrates some of
the basics, including how to create child entries that use the same
name as the parent entry. This example adds the glossary to the
table of contents and it also uses \ics{glsrefentry}, so an extra \LaTeX\ 
run is required:
\begin{prompt}
latex sample
makeglossaries sample
latex sample
latex sample
\end{prompt}
You can see the difference between word and letter ordering if you
substitute \pkgopt[word]{order} with \pkgopt[letter]{order}. (Note
that this will only have an effect if you use
\gls{makeglossaries}. If you use \gls{makeindex} explicitly,
you will need to use the \texttt{-l} switch to indicate letter
ordering.)

\item[sample-inline.tex]\label{ex:sample-inline} This document is
like \texttt{sample.tex}, above, but uses the \glostyle{inline}
glossary style to put the glossary in a footnote.

\item[sampletree.tex]\label{ex:sampletree} This document illustrates
a hierarchical glossary structure where child entries have different
names to their corresponding parent entry. To create the document
do:
\begin{prompt}
latex sampletree
makeglossaries sampletree
latex sampletree
\end{prompt}

\item[sample-dual.tex]\label{ex:sample-dual} This document
illustrates how to define an entry that both appears in the list of
acronyms and in the main glossary. To create the document do:
\begin{prompt}
latex sample-dual
makeglossaries sample-dual
latex sample-dual
\end{prompt}

\item[sample-langdict]\label{ex:sample-langdict} This document
illustrates how to use the glossaries package to create English
to French and French to English dictionaries. To create the document
do:
\begin{prompt}
latex sample-langdict
makeglossaries sample-langdict
latex sample-langdict
\end{prompt}

\item[samplexdy.tex]\label{ex:samplexdy} This document illustrates
how to use the \styfmt{glossaries} package with \gls{xindy} instead
of \gls{makeindex}. The document uses UTF8 encoding (with the
\sty{inputenc} package). The encoding is picked up by
\gls{makeglossaries}. By default, this document will create a
\gls{xindy} style file called \texttt{samplexdy.xdy}, but if you
uncomment the lines
\begin{verbatim}
\setStyleFile{samplexdy-mc}
\noist
\GlsSetXdyLanguage{}
\end{verbatim}
it will set the style file to \texttt{samplexdy-mc.xdy} instead.
This provides an additional letter group for entries starting with
\qt{Mc} or \qt{Mac}. If you use \gls{makeglossaries}, you don't
need to supply any additional information. If you don't use
\gls*{makeglossaries}, you will need to specify the required
information. Note that if you set the style file to
\texttt{samplexdy-mc.xdy} you must also specify \ics{noist},
otherwise the \styfmt{glossaries} package will overwrite
\texttt{samplexdy-mc.xdy} and you will lose the \qt{Mc} letter
group.

To create the document do:
\begin{prompt}
latex samplexdy
makeglossaries samplexdy
latex samplexdy
\end{prompt}
If you don't have Perl installed, you will have to call 
\gls{xindy} explicitly instead of using \gls{makeglossaries}.
If you are using the default style file \texttt{samplexdy.xdy}, then
do (no line breaks):
\begin{prompt}
xindy -L english -C utf8 -I xindy -M samplexdy -t samplexdy.glg -o samplexdy.gls samplexdy.glo
\end{prompt}
otherwise, if you are using \texttt{samplexdy-mc.xdy}, then do
(no line breaks):
\begin{prompt}
xindy -I xindy -M samplexdy-mc -t samplexdy.glg -o samplexdy.gls samplexdy.glo
\end{prompt}

\item[samplexdy2.tex]\label{ex:samplexdy2} This document illustrates
how to use the \styfmt{glossaries} package where the location
numbers don't follow a standard format. This example will only work
with \gls{xindy}. To create the document do:
\begin{prompt}
pdflatex samplexdy2
makeglossaries samplexdy2
pdflatex samplexdy2
\end{prompt}
If you can't use \gls{makeglossaries} then you need to do:
\begin{prompt}
xindy -L english -C utf8 -I xindy -M samplexdy2 -t samplexdy2.glg -o samplexdy2.gls samplexdy2.glo
\end{prompt}
See \sectionref{sec:xindyloc} for further details.

\item[sampleutf8.tex]\label{ex:sampleutf8} This is another example
that uses \gls{xindy}. Unlike \gls{makeindex},
\gls{xindy} can cope with accented or non-Latin characters. This
document uses UTF8 encoding. To create the document do:
\begin{prompt}
latex sampleutf8
makeglossaries sampleutf8
latex sampleutf8
\end{prompt}
If you don't have Perl installed, you will have to call
\gls{xindy} explicitly instead of using \gls{makeglossaries}
(no line breaks):
\begin{prompt}
xindy -L english -C utf8 -I xindy -M sampleutf8 -t sampleutf8.glg -o sampleutf8.gls sampleutf8.glo
\end{prompt}

If you remove the \pkgopt{xindy} option from \texttt{sampleutf8.tex} 
and do:
\begin{prompt}
latex sampleutf8
makeglossaries sampleutf8
latex sampleutf8
\end{prompt}
you will see that the entries that start with a non-Latin character
now appear in the symbols group, and the word \qt{man\oe uvre} is now 
after \qt{manor} instead of before it. If you are unable to use
\gls{makeglossaries}, the call to \gls{makeindex} is as
follows (no line breaks):
\begin{prompt}
makeindex -s sampleutf8.ist -t sampleutf8.glg -o sampleutf8.gls sampleutf8.glo
\end{prompt}

\item[sampleaccsupp.tex]\label{ex:sampleaccsupp} This document
uses the experimental \sty{glossaries-accsupp} package. The
symbol is set to the replacement text. Note that some PDF
viewers don't use the accessibility support. Information 
about the \styfmt{glossaries-accsupp} package can be found in
\sectionref{sec:accsupp}.

\end{description}


\section{Multi-Lingual Support}
\label{sec:languages}

As from version 1.17, the \styfmt{glossaries} package can now be
used with \gls{xindy} as well as \gls{makeindex}. If you
are writing in a language that uses accented characters or
non-Latin characters it is recommended that you use \gls*{xindy}
as \gls*{makeindex} is hard-coded for Latin languages. This
means that you are not restricted to the A, \ldots, Z letter groups.
If you want to use \gls*{xindy}, remember to use the
\pkgopt{xindy} package option. For example:
\begin{verbatim}
\documentclass[frenchb]{article}
\usepackage[utf8]{inputenc}
\usepackage[T1]{fontenc}
\usepackage{babel}
\usepackage[xindy]{glossaries}
\end{verbatim}

\begin{important}
Note that although an accented character, such as é, looks like a plain
character in your tex file, it's actually a macro and can therefore
cause problems.

\begin{enumerate}
\item
If you use an accented (or other expandable) character at the start of
an entry name, you must place it in a group, or it will cause
a problem for commands that convert the first letter to uppercase
(e.g.\ \ics{Gls}) due to expansion issues. For example:
\begin{verbatim}
\newglossaryentry{elite}{name={{é}lite},
description={select group or class}}
\end{verbatim}

\item If you use an accented (or other expandable) character in an
entry name and you haven't switched off the \gloskey{name} key
\glsation{sanitize}, you must use commands like
\ics{glsentrytext} or \ics{glstext} instead of \ics{glsentryname} or
\ics{glsname} or you will end up with strange looking characters in
your document.

\end{enumerate}
\end{important}

If you use the \sty{inputenc} package, \gls{makeglossaries}
will pick up the encoding from the auxiliary file. If you use
\gls{xindy} explicitly instead of via \gls*{makeglossaries},
you may need to specify the encoding using the \texttt{-C} 
option. Read the \gls*{xindy} manual for further details.

\subsection{Changing the Fixed Names}
\label{sec:fixednames}

As from version 1.08, the \styfmt{glossaries} package now has
limited multi-lingual support, thanks to all the people who have sent
me the relevant translations either via email or via 
\texttt{comp.text.tex}.
However you must load \sty{babel} or \sty{polyglossia} \emph{before} 
\styfmt{glossaries} to enable this. Note that if \sty{babel} is loaded 
and the \sty{translator} package is detected on \TeX's path, then the
\sty{translator} package will be loaded automatically.  However,
it may not pick up on the required languages so, if the predefined
text is not translated, you may need to explicitly load the
\sty{translator} package with the required languages. For example:
\begin{verbatim}
\usepackage[spanish]{babel}
\usepackage[spanish]{translator}
\usepackage{glossaries}
\end{verbatim}
Alternatively, specify the language as a class option rather
than a package option. For example:
\begin{verbatim}
\documentclass[spanish]{report}

\usepackage{babel}
\usepackage{glossaries}
\end{verbatim}

If you want to use \sty{ngerman} or \sty{german} instead
of \sty{babel}, you will need to include the \sty{translator} package 
to provide the translations. For example:
\begin{verbatim}
\documentclass[ngerman]{article}
\usepackage{ngerman}
\usepackage{translator}
\usepackage{glossaries}
\end{verbatim}

The languages are currently supported by the
\styfmt{glossaries} package are listed in
\tableref{tab:supportedlanguages}. Please note that (apart from
spelling mistakes) I don't intend to change the default translations
as it will cause compatibility problems.

\begin{table}[htbp]
\caption{Supported Languages}
\label{tab:supportedlanguages}
\begin{center}
\begin{tabular}{lc}
\bfseries Language & \bfseries As from version\\
Brazilian Portuguese & 1.17\\
Danish & 1.08\\
Dutch & 1.08\\
English & 1.08\\
French & 1.08\\
German & 1.08\\
Irish & 1.08\\
Italian & 1.08\\
Hungarian & 1.08\\
Polish & 1.13\\
Serbian & 2.06\\
Spanish & 1.08
\end{tabular}
\end{center}
\end{table}

The language dependent commands and \sty{translator} keys used by the 
glossaries package are listed in \tableref{tab:predefinednames}.

\begin{table}[htbp]
\caption{Customised Text}
\label{tab:predefinednames}
\centering
\setlength{\tabcolsep}{3pt}
\begin{tabular}{@{}l>{\raggedright}p{0.3\linewidth}>{\raggedright}p{0.4\linewidth}@{}}
\bfseries Command Name & \bfseries Translator Key Word &
\bfseries Purpose\cr
\ics{glossaryname} & \texttt{Glossary} & Title of the main glossary.\cr
\ics{acronymname} & \texttt{Acronyms} & Title of the list of acronyms
(when used with package option \pkgopt{acronym}).\cr
\ics{entryname} & \texttt{Notation (glossaries)} & 
Header for first column in the glossary (for 2, 3 or 4 column glossaries 
that support headers).\cr
\ics{descriptionname} & \texttt{Description (glossaries)} &
Header for second column in the glossary (for 2, 3 or 4 column glossaries
that support headers).\cr
\ics{symbolname} & \texttt{Symbol (glossaries)} & Header for symbol
column in the glossary for glossary styles that support this option.\cr
\ics{pagelistname} & \texttt{Page List (glossaries)} & 
Header for page list column in the glossary for glossaries that support 
this option.\cr
\ics{glssymbolsgroupname} & \texttt{Symbols (glossaries)} & 
Header for symbols section of the glossary for glossary styles that 
support this option.\cr
\ics{glsnumbersgroupname} & \texttt{Numbers (glossaries)} & Header for
numbers section of the glossary for glossary styles that support
this option.
\end{tabular}
\end{table}

Due to the varied nature of glossaries, it's likely that the
predefined translations may not be appropriate. If you are using the
\sty{babel} package and do not have the \sty{translator} package
installed, you need to be familiar with the advice given in
\urlref{http://www.tex.ac.uk/cgi-bin/texfaq2html?label=latexwords}{changing
the words babel uses}. If you have the \sty{translator} package
installed, then you can provide your own dictionary with the
necessary modifications (using \cs{deftranslation}) and load it
using \cs{usedictionary}.

\begin{important}
Note that the dictionaries are loaded at the beginning of the
document, so it won't have any effect if you put \cs{deftranslation}
in the preamble. It should be put in your personal dictionary
instead (as in the example below). See the \sty{translator}
documentation for further details. (Now with \sty{beamer}
documentation.)
\end{important}

Your custom dictionary doesn't have to be just a translation from
English to another language. You may prefer to have a dictionary for
a particular type of document. For example, suppose your
institution's in-house reports have to have the glossary labelled as
\qt{Nomenclature} and the page list should be labelled
\qt{Location}, then you can create a file called, say,
\begin{verbatim}
myinstitute-glossaries-dictionary-English.dict
\end{verbatim}
that contains the following:
\begin{verbatim}
\ProvidesDictionary{myinstitute-glossaries-dictionary}{English}
\deftranslation{Glossary}{Nomenclature}
\deftranslation{Page List (glossaries)}{Location}
\end{verbatim}
You can now load it using:
\begin{verbatim}
\usedictionary{myinstitute-glossaries-dictionary}
\end{verbatim}
(Make sure that \texttt{myinstitute-glossaries-dictionary-English.dict}
can be found by \TeX.) If you want to share your custom dictionary,
you can upload it to \href{http://www.ctan.org/}{CTAN}.

If you are using \sty{babel} and don't want to use the
\sty{translator} interface, you can suppress it using the package
option \pkgopt[false]{translate}, and either load
\sty{glossaries-babel} after \styfmt{glossaries} or specify you're own
translations. For example:
\begin{verbatim}
\documentclass[british]{article}

\usepackage{babel}
\usepackage[translate=false]{glossaries}
\usepackage{glossaries-babel}
\end{verbatim}
or:
\begin{verbatim}
\documentclass[british]{article}

\usepackage{babel}
\usepackage[translate=false]{glossaries}

\addto\captionsbritish{%
    \renewcommand*{\glossaryname}{List of Terms}%
    \renewcommand*{\acronymname}{List of Acronyms}%
    \renewcommand*{\entryname}{Notation}%
    \renewcommand*{\descriptionname}{Description}%
    \renewcommand*{\symbolname}{Symbol}%
    \renewcommand*{\pagelistname}{Page List}%
    \renewcommand*{\glssymbolsgroupname}{Symbols}%
    \renewcommand*{\glsnumbersgroupname}{Numbers}%
}
\end{verbatim}

If you are using \sty{polyglossia} instead of \sty{babel}, 
\sty{glossaries-polyglossia} will automatically be loaded unless
you specify the package option \pkgopt[false]{translate}.

Note that \gls{xindy} provides much better multi-lingual support
than \gls{makeindex}, so it's recommended that you use \gls*{xindy}
if you have glossary entries that contain diacritics or
non-Roman letters. See \sectionref{sec:xindy} for further
details.

\section{Generating the Associated Glossary Files}
\label{sec:makeglossaries}

In order to generate a sorted glossary with compact \glspl{locationlist},
it is necessary to use an external \gls{indexingapp} as an
intermediate step. It is this application that creates the file
containing the code that typesets the glossary. If this step is
omitted, the glossaries will not appear in your document. The two
\glspl*{indexingapp} that are most commonly used with \LaTeX\ are
\gls{makeindex} and \gls{xindy}. As from version 1.17, the
\styfmt{glossaries} package can be used with either of these
applications. Previous versions were designed to be used with
\gls*{makeindex} only. Note that \gls*{xindy} has much better
multi-lingual support than \gls*{makeindex}, so \gls*{xindy} is
recommended if you're not writing in English. Commands that only
have an effect when \gls*{xindy} is used are described in
\sectionref{sec:xindy}.

The \styfmt{glossaries} package comes with the Perl script
\gls{makeglossaries} which will run \gls{makeindex} or \gls{xindy}
on all the glossary files using a customized style file (which is
created by \ics{makeglossaries}). See
\sectionref{sec:makeglossariesapp} for further
details. Perl is stable, cross-platform, open source software that
is used by a number of \TeX-related applications. Further
information is available at \url{http://www.perl.org/about.html}.
The advantages of using \gls*{makeglossaries}:
\begin{itemize}
\item It automatically detects whether to use \gls*{makeindex} or
\gls*{xindy} and sets the relevant application switches.

\item One call of \gls*{makeglossaries} will run 
\gls*{makeindex}/\gls*{xindy} for each glossary type.

\item If things go wrong, \gls{makeglossaries} will scan the
messages from \gls{makeindex} or \gls{xindy} and attempt to diagnose
the problem in relation to the \styfmt{glossaries} package. This
will hopefully provide more helpful messages in some cases. If it
can't diagnose the problem, you will have to read the relevant transcript
file and see if you can work it out from the \gls*{makeindex} or
\gls*{xindy} messages.

\end{itemize}

There is also a Java GUI alternative called \gls{makeglossariesgui},
distributed separately, that has diagnostic tools.

Whilst it is strongly recommended that you use the
\gls{makeglossaries} script or \gls{makeglossariesgui}, it is
possible to use the \styfmt{glossaries} package without using either
application. However, note that some commands and package options
have no effect if you don't use \gls*{makeglossaries} or
\gls*{makeglossariesgui}. These are listed in
\tableref{tab:makeglossariesCmds}.

\begin{important}
If you are choosing not to use \gls*{makeglossaries} because you
don't want to install Perl, you will only be able to use
\gls*{makeindex} as \gls*{xindy} also requires Perl.
\end{important}

Note that if any of your entries use an entry
that is not referenced outside the glossary, you will need to
do an additional \gls{makeglossaries}, \gls{makeindex}
or \gls{xindy} run, as appropriate. For example,
suppose you have defined the following entries:\footnote{As from
v3.01 \ics{gls} is no longer fragile and doesn't need protecting.}
\begin{verbatim}
\newglossaryentry{citrusfruit}{name={citrus fruit},
description={fruit of any citrus tree. (See also 
\gls{orange})}}

\newglossaryentry{orange}{name={orange},
description={an orange coloured fruit.}}
\end{verbatim}
and suppose you have \verb|\gls{citrusfruit}| in your document
but don't reference the \texttt{orange} entry, then the
\texttt{orange} entry won't appear in your glossary until
you first create the glossary and then do another run
of \gls{makeglossaries}, \gls{makeindex} or \gls{xindy}.
For example, if the document is called \texttt{myDoc.tex}, then
you must do:
\begin{prompt}
latex myDoc
makeglossaries myDoc
latex myDoc
makeglossaries myDoc
latex myDoc
\end{prompt}

Likewise, an additional \gls{makeglossaries} and \LaTeX\ run
may be required if the document pages shift with re-runs. For
example, if the page numbering is not reset after the table of
contents, the insertion of the table of contents on the second
\LaTeX\ run may push glossary entries across page boundaries, which
means that the \glspl{numberlist} in the glossary may 
need updating.

The examples in this document assume that you are accessing
\gls{makeglossaries}, \gls{xindy} or \gls{makeindex} via a terminal.
Windows users can use the MSDOS Prompt which is usually accessed via
the \menu{Start}\submenu{All Programs} menu or
\menu{Start}\submenu{All Programs}\submenu{Accessories} menu.

Alternatively, your text editor may have the facility to create a
function that will call the required application. The article
\href{http://www.latex-community.org/index.php?option=com_content&view=article\&id=263:glossaries-nomenclature-lists-of-symbols-and-acronyms&catid=55:latex-general&Itemid=114}{\qt{Glossaries, Nomenclature, List of Symbols and
Acronyms}}
in the \urlfootref{http://www.latex-community.org/}{\LaTeX\
Community's} Know How section
describes how to do this for TeXnicCenter, and the thread
\href{http://groups.google.com/group/comp.text.tex/browse_thread/thread/edd83831b81b0759?hl=en}{\qt{Executing Glossaries' makeindex from a WinEdt
macro}} on the \texttt{comp.text.tex} newsgroup
describes how to do it for WinEdt. For other editors see the editor's
user manual for further details.

If any problems occur, remember to check the transcript files 
(e.g.\ \filetype{.glg} or \filetype{.alg}) for messages.

\begin{table}[htbp]
\caption{Commands and package options that have no effect when
using \texttt{xindy} or \texttt{makeindex} explicitly}
\label{tab:makeglossariesCmds}
\vskip\baselineskip
\begin{tabular}{@{}lll@{}}
\bfseries Command or Package Option & \gls{makeindex} & 
\gls{xindy}\\
\pkgopt[letter]{order} & use \texttt{-l} &
 use \texttt{-M ord/letorder}\\
\pkgopt[word]{order} & default & default\\
\pkgopt{xindy}=\{\pkgoptfmt{language=}\meta{lang}\pkgoptfmt{,codename=}\meta{code}\} &
N/A & use \texttt{-L} \meta{lang} \texttt{-C} \meta{code}\\
\ics{GlsSetXdyLanguage}\marg{lang} & N/A &
use \texttt{-L} \meta{lang}\\
\ics{GlsSetXdyCodePage}\marg{code} & N/A &
use \texttt{-C} \meta{code}
\end{tabular}
\par
\end{table}


\subsection{Using the makeglossaries Perl Script}
\label{sec:makeglossariesapp}

The \gls{makeglossaries} script picks up the relevant information
from the auxiliary (\filetype{.aux}) file and will either call
\gls{xindy} or \gls{makeindex}, depending on the supplied
information. Therefore, you only need to pass the document's name
without the extension to \gls*{makeglossaries}. For example, if your
document is called \texttt{myDoc.tex}, type the following in your
terminal:
\begin{prompt}
latex myDoc
makeglossaries myDoc
latex myDoc
\end{prompt}
You may need to explicitly load \gls{makeglossaries} into Perl:
\begin{prompt}
perl makeglossaries myDoc
\end{prompt}
There is a batch file called \texttt{makeglossaries.bat} which does
this for Windows users, but you must have Perl installed to be able
to use it. You can specify in which directory the \filetype{.aux}, 
\filetype{.glo} etc files are located using the \texttt{-d} switch.
For example:
\begin{prompt}
pdflatex -output-directory myTmpDir myDoc
makeglossaries -d myTmpDir myDoc
\end{prompt}

The \gls{makeglossaries} script contains POD (Plain Old
Documentation). If you want, you can create a man page for
\gls*{makeglossaries} using \app{pod2man} and move the 
resulting file onto the man path. Alternatively do
\texttt{makeglossaries -{}-help} for a list of all options or
\texttt{makeglossaries -{}-version} for the version number.

\begin{important}
When upgrading the \styfmt{glossaries} package, make sure you also
upgrade your version of \gls{makeglossaries}. The current version is
2.04.
\end{important}

\subsection{Using xindy explicitly}
\label{sec:xindyapp}

\Gls{xindy} comes with TeXLive, but not with MiKTeX. However MikTeX
users can install it. There is
\href{http://www.latex-community.org/forum/viewtopic.php?f=51&t=5383}{a
thread} in the Makeindex section of the
\urlfootref{http://www.latex-community.org/}{\LaTeX\ Community} that
describes how to do this.

If you want to use \gls{xindy} to process the glossary
files, you must make sure you have used the 
\pkgopt{xindy} package option:
\begin{verbatim}
\usepackage[xindy]{glossaries}
\end{verbatim}
This is required regardless of whether you use \gls{xindy}
explicitly or whether it's called implicitly via 
\gls{makeglossaries} or \gls{makeglossariesgui}. This causes the glossary 
entries to be written in raw \gls*{xindy} format, so you need to
use \texttt{-I xindy} \emph{not} \texttt{-I tex}.

To run \gls{xindy} type the following in your terminal 
(all on one line):
\begin{prompt}
xindy -L \meta{language} -C \meta{encoding} -I xindy -M \meta{style} -t \meta{base}.glg -o \meta{base}.gls \meta{base}.glo
\end{prompt}
where \meta{language} is the required language name, \meta{encoding}
is the encoding, \meta{base} is the name of the document without the
\filetype{.tex} extension and \meta{style} is the name of the
\gls{xindy} style file without the \filetype{.xdy} extension.
The default name for this style file is \meta{base}\filetype{.xdy}
but can be changed via \ics{setStyleFile}\marg{style}. You may need
to specify the full path name depending on the current working
directory. If any of the file names contain spaces, you must delimit
them using double-quotes.

For example, if your document is called \texttt{myDoc.tex} and
you are using UTF8 encoding in English, then type the
following in your terminal:
\begin{prompt}
xindy -L english -C utf8 -I xindy -M myDoc -t myDoc.glg -o myDoc.gls myDoc.glo
\end{prompt}

Note that this just creates the main glossary. You need to do
the same for each of the other glossaries (including the
list of acronyms if you have used the \pkgopt{acronym}
package option), substituting \filetype{.glg}, \filetype{.gls}
and \filetype{.glo} with the relevant extensions. For example,
if you have used the \pkgopt{acronym} package option, then 
you would need to do:
\begin{prompt}
xindy -L english -C utf8 -I xindy -M myDoc -t myDoc.alg -o myDoc.acr myDoc.acn
\end{prompt}
For additional glossaries, the extensions are those supplied
when you created the glossary with \ics{newglossary}.

Note that if you use \gls{makeglossaries} instead, you can
replace all those calls to \gls{xindy} with just one call
to \gls*{makeglossaries}:
\begin{prompt}
makeglossaries myDoc
\end{prompt}
Note also that some commands and package options have no effect if 
you use \gls{xindy} explicitly instead of using 
\gls*{makeglossaries}. These are listed in 
\tableref{tab:makeglossariesCmds}.


\subsection{Using makeindex explicitly}
\label{sec:makeindexapp}

If you want to use \gls{makeindex} explicitly, you must
make sure that you haven't used the \pkgopt{xindy} package
option or the glossary entries will be written in the wrong
format. To run \gls*{makeindex}, type the following in
your terminal:
\begin{prompt}
makeindex -s \meta{style}.ist -t \meta{base}.glg -o \meta{base}.gls \meta{base}.glo
\end{prompt}
where \meta{base} is the name of your document without the
\filetype{.tex} extension and \meta{style}\filetype{.ist} is the 
name of the \gls{makeindex} style file. By default, this is
\meta{base}\filetype{.ist}, but may be changed via
\ics{setStyleFile}\marg{style}. Note that there are other options, 
such as \texttt{-l} (letter ordering). See the \gls*{makeindex}
manual for further details.

For example, if your document is called \texttt{myDoc.tex}, then
type the following at the terminal:
\begin{prompt}
makeindex -s myDoc.ist -t myDoc.glg -o myDoc.gls myDoc.glo
\end{prompt}
Note that this only creates the main glossary. If you have
additional glossaries (for example, if you have used the
\pkgopt{acronym} package option) then you must call 
\gls{makeindex} for each glossary, substituting 
\filetype{.glg}, \filetype{.gls} and \filetype{.glo} with the
relevant extensions. For example, if you have used the
\pkgopt{acronym} package option, then you need to type the
following in your terminal:
\begin{prompt}
makeindex -s myDoc.ist -t myDoc.alg -o myDoc.acr myDoc.acn
\end{prompt}
For additional glossaries, the extensions are those supplied
when you created the glossary with \ics{newglossary}.

Note that if you use \gls{makeglossaries} instead, you can
replace all those calls to \gls{makeindex} with just one call
to \gls*{makeglossaries}:
\begin{prompt}
makeglossaries myDoc
\end{prompt}
Note also that some commands and package options have no effect if 
you use \gls*{makeindex} explicitly instead of using 
\gls{makeglossaries}. These are listed in 
\tableref{tab:makeglossariesCmds}.


\subsection{Note to Front-End and Script Developers}
\label{sec:notedev}

The information needed to determine whether to use \gls{xindy}
or \gls{makeindex} and the information needed to call those
applications is stored in the auxiliary file. This information can
be gathered by a front-end, editor or script to make the glossaries
where appropriate. This section describes how the information is
stored in the auxiliary file.

The file extensions used by each defined glossary are given by
\begin{definition}[\DescribeMacro{\@newglossary}]
\cs{@newglossary}\marg{label}\marg{log}\marg{out-ext}\marg{in-ext}
\end{definition}
where \meta{in-ext} is the extension of the
\emph{\gls{indexingapp}['s]} input file (the output file from the
\styfmt{glossaries} package's point of view), \meta{out-ext} is the
extension of the \emph{\gls*{indexingapp}['s]} output file (the
input file from the \styfmt{glossaries} package's point of view) and
\meta{log} is the extension of the \gls*{indexingapp}['s] transcript
file. The label for the glossary is also given for information
purposes only, but is not required by the \glspl*{indexingapp}. For
example, the information for the default main glossary is written
as:
\begin{verbatim}
\@newglossary{main}{glg}{gls}{glo}
\end{verbatim}

The \gls{indexingapp}['s] style file is specified by
\begin{definition}[\DescribeMacro{\@istfilename}]
\cs{@istfilename}\marg{filename}
\end{definition}
The file extension indicates whether to use \gls{makeindex}
(\filetype{.ist}) or \gls{xindy} (\filetype{.xdy}). Note that
the glossary information is formatted differently depending on
which \gls*{indexingapp} is supposed to be used, so it's 
important to call the correct one.

Word or letter ordering is specified by:
\begin{definition}[\DescribeMacro{\@glsorder}]
\cs{@glsorder}\marg{order}
\end{definition}
where \meta{order} can be either \texttt{word} or \texttt{letter}.

If \gls{xindy} should be used, the language and code page
for each glossary is specified by
\begin{definition}[\DescribeMacro{\@xdylanguage}\DescribeMacro{\@gls@codepage}]
\cs{@xdylanguage}\marg{label}\marg{language}\\
\cs{@gls@codepage}\marg{label}\marg{code}
\end{definition}
where \meta{label} identifies the glossary, \meta{language} is
the root language (e.g.\ \texttt{english}) and \meta{code}
is the encoding (e.g.\ \texttt{utf8}). These commands are omitted
if \gls{makeindex} should be used.


\chapter{Package Options}
\label{sec:pkgopts}

This section describes the available \styfmt{glossaries} package options.

\section{General Options}
\label{sec:pkgopts-general}

\begin{description}
\item[\pkgopt{nowarn}] This suppresses all warnings generated by
the \styfmt{glossaries} package.

\item[\pkgopt{nomain}] This suppresses the creation of the main
glossary. Note that if you use this option, you must create another
glossary in which to put all your entries (either via the
\pkgopt{acronym} package option described in
\sectionref{sec:pkgopts-acronym} or via \ics{newglossary} described
in \sectionref{sec:newglossary}).

\item[\pkgopt{sanitize}] This is a
\meta{key}=\meta{value} option whose value is also a
\meta{key}=\meta{value} list. By default, the \styfmt{glossaries}
package \glspl{sanitize} the values of the \gloskey{name},
\gloskey{description}, \gloskey{symbol} and \gloskey{sort} keys used when defining a
new glossary entry. This means that you can use fragile commands in
those keys, but it may lead to unexpected results if you try to
display these values within the document text. This sanitization can
be switched off using the \pkgopt{sanitize} package option. 
For example, to switch off the sanitization
for the \gloskey{description} and \gloskey{name} keys, but not for
the \gloskey{symbol} key, do:
\begin{verbatim}
\usepackage[sanitize={name=false,description=false,%
symbol=true}]{glossaries}
\end{verbatim}
You can use \pkgopt[none]{sanitize} as a shortcut for\newline
\pkgoptfmt{sanitize=\{name=false,description=false,symbol=false\}}.
Note that this shortcut doesn't change the \gloskey{sort}
sanitization. That one needs to be switched off explicitly:
\begin{verbatim}
\usepackage{sanitize=none,sanitize={sort=false}}
\end{verbatim}

\begin{important}%
\textbf{Note}: this sanitization only applies to the \gloskey{name}, 
\gloskey{description}, \gloskey{symbol} and \gloskey{sort} keys. It doesn't apply
to any of the other keys so fragile commands contained in the value of
the other keys must always be protected using \cs{protect}.
Since the value of the \gloskey{text} key is obtained from the
\gloskey{name} key, you will still need to protect fragile commands
in the \gloskey{name} key if you don't use the \gloskey{text} key. 
\end{important}

\item[\pkgopt{savewrites}] This is a boolean option to minimises the
number of write registers used by the \styfmt{glossaries} package.
(Default is \pkgopt[false]{savewrites}.) There are only a limited
number of write registers, and if you have a large number of
glossaries or if you are using a class or other packages that
create a lot of external files, you may exceed the maximum number
of available registers. If \pkgopt{savewrites} is set, the glossary
information will be stored in token registers until the end of the
document when they will be written to the external files. If you run
out of token registers, you can use \sty{etex}.

\begin{important}
If you want to use \TeX's \ics{write18} mechanism to call
\gls{makeindex} or \gls{xindy} from your document and use
\pkgopt{savewrites}, you must create the external files with 
\cs{glswritefiles} before you call \gls*{makeindex}/\gls*{xindy}. Also set
\cs{glswritefiles} to nothing or \cs{relax} before the end of the
document to avoid rewriting the files. For example:
\begin{verbatim}
\glswritefiles
\write18{makeindex -s \istfilename\space 
-t \jobname.glg -o \jobname.gls \jobname}
\let\glswritefiles\relax
\end{verbatim}
\end{important}

\item[\pkgopt{translate}] This is a boolean option. The default is
\pkgoptval{true}{translate} if \sty{babel}, \sty{polyglossia} or 
\sty{translator} have been loaded, otherwise the default value is
\pkgoptval{false}{translate}.

  \begin{description}
  \item[{\pkgopt[true]{translate}}] If \sty{babel} has been loaded
  and the \sty{translator} package is installed, \sty{translator}
  will be loaded and the translations will be provided by the
  \sty{translator} package interface. You can modify the 
  translations by providing your own dictionary. If the 
  \sty{translator} package isn't installed and \sty{babel} is
  loaded, the \sty{glossaries-babel} package will 
  be loaded and the translations will be provided using \styfmt{babel}'s
  \cs{addto}\cs{caption}\meta{language} mechanism. If 
  \sty{polyglossia} has been loaded, \sty{glossaries-polyglossia}
  will be loaded.

  \item[{\pkgopt[false]{translate}}] Don't provide translations, even
  if \sty{babel} or \sty{polyglossia} has been loaded. You can 
  then provide you're own translations or explicitly load 
  \sty{glossaries-babel} or \sty{glossaries-polyglossia}.
  \end{description}

See \sectionref{sec:fixednames} for further details.

\item[\pkgopt{hyperfirst}] This is a boolean option that specifies
whether each term has a hyperlink on \firstuse. The default is 
\pkgopt[true]{hyperfirst} (terms on \gls{firstuse} have a hyperlink, 
unless explicitly suppressed using starred versions of commands
such as \ics{gls*}).

\item[\pkgopt{nohypertypes}] Use this option if you have multiple
glossaries and you want to suppress the entry hyperlinks for a
particular glossary or glossaries. The value of this option should
be a comma-separated list of glossary types where \ics{gls} etc
shouldn't have hyperlinks by default. Make sure you enclose the
value in braces if it contains any commas. Example:
\begin{verbatim}
\usepackage[acronym,nohypertypes={acronym,notation}]{glossaries}
\newglossary[nlg]{notation}{not}{ntn}{Notation}
\end{verbatim}
See \sectionref{sec:glslink} for further details.

\item[\pkgopt{savenumberlist}] This is a boolean option that
specifies whether or not to gather and store the \gls{numberlist}
for each entry. The default is \pkgopt[false]{savenumberlist}. (See
\ics{glsentrynumberlist} and \ics{glsdisplaynumberlist} in
\sectionref{sec:glsnolink}.)

\end{description}

\section{Sectioning, Headings and TOC Options}
\label{sec:pkgopts-sec}

\begin{description}
\item[\pkgopt{toc}] Add the glossaries to the table of contents.
Note that an extra \LaTeX\ run is required with this option.
Alternatively, you can switch this function on and off using
\begin{definition}[\DescribeMacro{\glstoctrue}]
\cs{glstoctrue}
\end{definition}
and
\begin{definition}[\DescribeMacro{\glstocfalse}]
\cs{glstocfalse}
\end{definition}

\item[\pkgopt{numberline}] When used with \pkgopt{toc}, this will
add \ics{numberline}\verb|{}| in the final argument of 
\ics{addcontentsline}. This will align the table of contents entry 
with the numbered section titles. Note that this option has no
effect if the \pkgopt{toc} option is omitted. If \pkgopt{toc} is
used without \pkgopt{numberline}, the title will be aligned with
the section numbers rather than the section titles.

\item[\pkgopt{section}] This is a \meta{key}=\meta{value} option.  Its
value should be the name of a sectional unit (e.g.\ chapter).
This will make the glossaries appear in the named sectional unit,
otherwise each glossary will appear in a chapter, if chapters
exist, otherwise in a section. Unnumbered sectional units will be used
by default. Example:
\begin{verbatim}
\usepackage[section=subsection]{glossaries}
\end{verbatim}
You can omit the value if you want to use sections, i.e.\
\begin{verbatim}
\usepackage[section]{glossaries}
\end{verbatim}
is equivalent to
\begin{verbatim}
\usepackage[section=section]{glossaries}
\end{verbatim}
You can change this value later in the document using
\begin{definition}[\DescribeMacro{\setglossarysection}]
\cs{setglossarysection}\marg{name}
\end{definition}
where \meta{name} is the sectional unit.

The start of each glossary adds information to the page header via
\begin{definition}[\DescribeMacro{\glossarymark}]
\cs{glossarymark}\marg{glossary title}
\end{definition}
This defaults to \cs{@mkboth} unless \cls{memoir} is loaded, but you may 
need to redefine it.
For example, to only change the right header:
\begin{verbatim}
\renewcommand{\glossarymark}[1]{\markright{#1}}
\end{verbatim}
or to prevent it from changing the headers:
\begin{verbatim}
\renewcommand{\glossarymark}[1]{}
\end{verbatim}
If you want \cs{glossarymark} to use \cs{MakeUppercase} in the header, use the
\pkgopt{ucmark} option described below.

Occasionally you may find that another package defines 
\linebreak\cs{cleardoublepage} when it is not required. This may cause an 
unwanted blank page to appear before each glossary. This can
be fixed by redefining \ics{glsclearpage}:
\begin{verbatim}
\renewcommand*{\glsclearpage}{\clearpage}
\end{verbatim}

\item[\pkgopt{ucmark}] This is a boolean option (default: \pkgopt[false]{ucmark}). If
set, \ics{glossarymark} is defined to use \ics{MakeUppercase}.

\item[\pkgopt{numberedsection}]
The glossaries are placed in unnumbered sectional units by default,
but this can be changed using \pkgopt{numberedsection}. This option can take
three possible values: \pkgoptval{false}{numberedsection} (no
number, i.e.\ use starred form), \pkgoptval{nolabel}{numberedsection}
(numbered, i.e.\ unstarred form, but not labelled) and
\pkgoptval{autolabel}{numberedsection} (numbered with automatic
labelling). If \pkgopt[autolabel]{numberedsection} is used, each
glossary is given a label that matches the glossary type, so the
main (default) glossary is labelled \texttt{main}, the list of
acronyms is labelled \texttt{acronym}\footnote{if the
\pkgopt{acronym} option is used, otherwise the list of
acronyms is the main glossary} and additional glossaries are labelled
using the value specified in the first mandatory argument to 
\cs{newglossary}. For example, if you load \styfmt{glossaries}
using:
\begin{verbatim}
\usepackage[section,numberedsection=autolabel]{glossaries}
\end{verbatim}
then each glossary will appear in a numbered section, and can
be referenced using something like:
\begin{verbatim}
The main glossary is in section~\ref{main} and the list of
acronyms is in section~\ref{acronym}.
\end{verbatim}
If you can't decide whether to have the acronyms in the main
glossary or a separate list of acronyms, you can use
\ics{acronymtype} which is set to \texttt{main} if the
\pkgopt{acronym} option is not used and is set to \texttt{acronym}
if the \pkgopt{acronym} option is used. For example:
\begin{verbatim}
The list of acronyms is in section~\ref{\acronymtype}.
\end{verbatim}

As from version 1.14, you can add a prefix to the label by
redefining 
\begin{definition}[\DescribeMacro{\glsautoprefix}]
\cs{glsautoprefix}
\end{definition}
For example:
\begin{verbatim}
\renewcommand*{\glsautoprefix}{glo:}
\end{verbatim}
will add \texttt{glo:} to the automatically generated label, so
you can then, for example, refer to the list of acronyms as follows:
\begin{verbatim}
The list of acronyms is in 
section~\ref{glo:\acronymtype}.
\end{verbatim}
Or, if you are undecided on a prefix:
\begin{verbatim}
The list of acronyms is in 
section~\ref{\glsautoprefix\acronymtype}.
\end{verbatim}

\end{description}

\section{Glossary Appearance Options}
\label{sec:pkgopts-printglos}

\begin{description}
\item[\pkgopt{entrycounter}] This is a boolean option. (Default
is \pkgopt[false]{entrycounter}.) If set, each main (level~0)
glossary entry will be numbered when using the standard glossary
styles. This option creates the counter
\DescribeCounter{glossaryentry}\ctrfmt{glossaryentry}.

If you use this option, you can reference the entry number
within the document using
\begin{definition}[\DescribeMacro{\glsrefentry}]
\cs{glsrefentry}\marg{label}
\end{definition}
where \meta{label} is the label associated with that glossary entry.

\begin{important}
If you use \cs{glsrefentry}, you must run \LaTeX\ twice after
creating the glossary files using \gls{makeglossaries},
\gls{makeindex} or \gls{xindy} to ensure the cross-references are
up-to-date.
\end{important}

\item[\pkgopt{counterwithin}] This is a \meta{key}=\meta{value}
option where \meta{value} is the name of a counter. If used, this
option will automatically set \pkgopt[true]{entrycounter} and the
\ctr{glossaryentry} counter will be reset every time \meta{value} is
incremented.

\DescribeMacro{\glsresetentrycounter}
\begin{important}
The \ctr{glossaryentry} counter isn't automatically reset at the
start of each glossary, except when glossary section numbering is on
and the counter used by \pkgopt{counterwithin} is the same as the
counter used in the glossary's sectioning command. If you want the
counter reset at the start of each glossary, you can redefine
\ics{glossarypreamble} to use \cs{glsresetentrycounter}, which sets
\ctr{glossaryentry} to zero:
\begin{verbatim}
\renewcommand{\glossarypreamble}{%
  \glsresetentrycounter
}
\end{verbatim}
\end{important}

\item[\pkgopt{subentrycounter}] This is a boolean option. (Default
is \pkgopt[false]{subentrycounter}.) If set, each level~1
glossary entry will be numbered when using the standard glossary
styles. This option creates the counter
\DescribeCounter{glossarysubentry}\ctrfmt{glossarysubentry}.
The counter is reset with each main (level~0) entry. Note that this
package option is independent of \pkgopt{entrycounter}. You can
reference the number within the document using
\ics{glsrefentry}\marg{label} where \meta{label} is the label
associated with the sub-entry.

\item[\pkgopt{style}] This is a \meta{key}=\meta{value} option.
(Default is \pkgopt[list]{style}.) Its value should be the name of
the glossary style to use. This key may only be used for styles
defined in \sty{glossary-list}, \sty{glossary-long},
\sty{glossary-super} or \sty{glossary-tree}. (See
\sectionref{sec:styles}.)

\item[\pkgopt{nolong}] This prevents the \styfmt{glossaries} package
from automatically loading \sty{glossary-long} (which means that
the \sty{longtable} package also won't be loaded). This reduces
overhead by not defining unwanted styles and commands. Note that if
you use this option, you won't be able to use any of the
glossary styles defined in the \styfmt{glossary-long} package.

\item[\pkgopt{nosuper}] This prevents the \styfmt{glossaries} package
from automatically loading \sty{glossary-super} (which means that
the \sty{supertabular} package also won't be loaded). This reduces
overhead by not defining unwanted styles and commands. Note that if
you use this option, you won't be able to use any of the
glossary styles defined in the \styfmt{glossary-super} package.

\item[\pkgopt{nolist}] This prevents the \styfmt{glossaries} package
from automatically loading \sty{glossary-list}. This reduces
overhead by not defining unwanted styles. Note that if
you use this option, you won't be able to use any of the
glossary styles defined in the \styfmt{glossary-list} package.
Note that since the default style is \glostyle{list}, you will 
also need to use the \pkgopt{style} option to set the style to
something else.

\item[\pkgopt{notree}] This prevents the \styfmt{glossaries} package
from automatically loading \sty{glossary-tree}. This reduces
overhead by not defining unwanted styles. Note that if
you use this option, you won't be able to use any of the
glossary styles defined in the \styfmt{glossary-tree} package.

\item[\pkgopt{nostyles}] This prevents all the predefined styles
from being loaded. If you use this option, you need to load a
glossary style package (such as \sty{glossary-mcols}). Also if you
use this option, you can't use the \pkgopt{style} package option.
Instead you must either use \ics{glossarystyle}\marg{style} or the
\gloskey[printglossary]{style} key in the optional argument to
\ics{printglossary}. Example:
\begin{verbatim}
\usepackage[nostyles]{glossaries}
\usepackage{glossary-mcols}
\glossarystyle{mcoltree}
\end{verbatim}

\item[\pkgopt{nonumberlist}] This option will suppress the 
associated \glspl{numberlist} in the glossaries (see also 
\sectionref{sec:numberlists}).

\item[\pkgopt{seeautonumberlist}] If you suppress the
\glspl{numberlist} with \pkgopt{nonumberlist}, described above, this
will also suppress any cross-referencing information supplied by the
\gloskey{see} key in \ics{newglossaryentry} or \ics{glssee}. If you
use \pkgopt{seeautonumberlist}, the \gloskey{see} key will
automatically implement \gloskey{nonumberlist=false} for that entry.
(Note this doesn't affect \cs{glssee}.) For further details see
\sectionref{sec:crossref}.

\item[\pkgopt{counter}] This is a \meta{key}=\meta{value} option.
(Default is \pkgopt[page]{counter}.) The value should be the name of
the default counter to use in the \glspl{numberlist}
(see \sectionref{sec:numberlists}).

\item[\pkgopt{nopostdot}] This is a boolean option. If no value is
specified, \texttt{true} is assumed. When set to \texttt{true}, this 
option suppresses the default post description dot used by some of 
the predefined styles. The default setting is
\pkgopt[false]{nopostdot}.

\item[\pkgopt{nogroupskip}] This is a boolean option. If no value is
specified, \texttt{true} is assumed. When set to \texttt{true}, this
option suppresses the default vertical gap between groups used by
some of the predefined styles. The default setting is
\pkgopt[false]{nogroupskip}.

\end{description}

\section{Sorting Options}
\label{sec:pkgopts-sort}

\begin{description}
\item[\pkgopt{sort}] This is a \meta{key}=\meta{value} option where
the option can only have one of the following values:

\begin{itemize}
\item \pkgoptval{standard}{sort} : entries are sorted according to
the value of the \gloskey{sort} key used in \ics{newglossaryentry}
(if present) or the \gloskey{name} key (if \gloskey{sort} key is
missing);

\item \pkgoptval{def}{sort} : entries are sorted in the order in
which they were defined (the \gloskey{sort} key in
\cs{newglossaryentry} is ignored);

\item \pkgoptval{use}{sort} : entries are sorted according to the
order in which they are used in the document (the \gloskey{sort} key
in \cs{newglossaryentry} is ignored).
\end{itemize}

The default is \pkgopt[standard]{sort}.

\item[\pkgopt{order}] This may take two values: 
\pkgoptval{word}{order} or \pkgoptval{letter}{order}. The default
is word ordering.

\begin{important}
Note that the \pkgopt{order} option has no effect if you don't use 
\gls{makeglossaries}.
\end{important}

\item[\pkgopt{makeindex}] (Default) The glossary information and
indexing style file will be written in \gls{makeindex} format. If
you use \gls{makeglossaries}, it will automatically detect that it
needs to call \gls*{makeindex}. If you don't use
\gls*{makeglossaries}, you need to remember to use \gls*{makeindex}
not \gls{xindy}. The indexing style file will been given a
\filetype{.ist} extension.

\item[\pkgopt{xindy}] The glossary information and indexing style
file will be written in \gls{xindy} format. If you use
\gls{makeglossaries}, it will automatically detect that it needs to
call \gls*{xindy}. If you don't use \gls*{makeglossaries}, you need to
remember to use \gls*{xindy} not \gls{makeindex}. The indexing style
file will been given a \filetype{.xdy} extension.

The \pkgopt{xindy} package option may additionally have a value that
is a \meta{key}=\meta{value} comma-separated list to override the
language and codepage. For example:
\begin{verbatim}
\usepackage[xindy={language=english,codepage=utf8}]{glossaries}
\end{verbatim}
You can also specify whether you want a number group in the
glossary. This defaults to true, but can be suppressed. For example:
\begin{verbatim}
\usepackage[xindy={glsnumbers=false}]{glossaries}
\end{verbatim}
See \sectionref{sec:xindy} for further details on using \gls{xindy}
with the \styfmt{glossaries} package.

\end{description}

\section{Acronym Options}
\label{sec:pkgopts-acronym}

\begin{description}
\item[\pkgopt{acronym}] This creates a new glossary with the
label \texttt{acronym}. This is equivalent to:
\begin{verbatim}
\newglossary[alg]{acronym}{acr}{acn}{\acronymname}
\end{verbatim}
If the \pkgopt{acronym} package option is used, \ics{acronymtype}
is set to \texttt{acronym} otherwise it is set to 
\texttt{main}.\footnote{Actually it sets \ics{acronymtype} to
\ics{glsdefaulttype} if the \pkgopt{acronym} package option is
not used, but \ics{glsdefaulttype} usually has the value
\texttt{main}.} 
Entries that are defined using \ics{newacronym} are placed in
the glossary whose label is given by \ics{acronymtype}, unless
another glossary is explicitly specified.

\item[\pkgopt{acronymlists}] By default, only the \cs{acronymtype}
glossary is considered to be a list of acronyms. If you have other
lists of acronyms, you can specify them as a comma-separated list
in the value of \pkgopt{acronymlists}. For example, if you use the
\pkgopt{acronym} package option but you also want the \texttt{main} 
glossary to also contain a list of acronyms, you can do:
\begin{verbatim}
\usepackage[acronym,acronymlists={main}]{glossaries}
\end{verbatim}
No check is performed to determine if the listed glossaries exist,
so you can add glossaries you haven't defined yet. For example:
\begin{verbatim}
\usepackage[acronym,acronymlists={main,acronym2}]{glossaries}
\newglossary[alg2]{acronym2}{acr2}{acn2}{Statistical Acronyms}
\end{verbatim}

\item[\pkgopt{description}] This option changes the definition of
\ics{newacronym} to allow a description. This option has no effect if
you defined your own custom acronym style. See
\sectionref{sec:acronyms} for further details.

\item[\pkgopt{footnote}] This option changes the definition of
\ics{newacronym} and the way that acronyms are displayed. This
option has no effect if you defined your own custom acronym style. See
\sectionref{sec:acronyms} for further details.

\item[\pkgopt{smallcaps}] This option changes the definition of
\ics{newacronym} and the way that acronyms are displayed. This
option may have no effect if you defined your own custom acronym
style. See \sectionref{sec:acronyms} for further details.

\item[\pkgopt{smaller}] This option changes the definition of
\ics{newacronym} and the way that acronyms are displayed. If you use
this option, you will need to include the \sty{relsize} package or
otherwise define \ics{textsmaller} or redefine \ics{acronymfont}.
This option may have no effect if you defined your own custom acronym
style. See \sectionref{sec:acronyms} for further details.

\item[\pkgopt{dua}] This option changes the definition of
\ics{newacronym} so that acronyms are always expanded. This
option has no effect if you defined your own custom acronym style. 
See \sectionref{sec:acronyms} for further details.

\item[\pkgopt{shortcuts}] This option provides shortcut commands
for acronyms.  See \sectionref{sec:acronyms} for further details.

\end{description}

\chapter{Setting Up}
\label{sec:setup}

The command
\begin{definition}[\DescribeMacro{\makeglossaries}]
\cs{makeglossaries}
\end{definition}
must be placed in the preamble in order to create the customised
\gls{makeindex} (\filetype{.ist}) or \gls{xindy} (\filetype{.xdy})
style file and to ensure that glossary entries are written to the
appropriate output files. If you omit \cs{makeglossaries} none of
the glossaries will be created.

\begin{important}
Note that some of the commands provided by the \styfmt{glossaries}
package must be placed before \cs{makeglossaries} as they are
required when creating the customised style file. If you attempt
to use those commands after \cs{makeglossaries} you will generate
an error.
\end{important}

You can suppress the creation of the customised \gls{xindy}
or \gls{makeindex} style file using
\begin{definition}[\DescribeMacro{\noist}]
\cs{noist}
\end{definition}
Note that this command must be used before \cs{makeglossaries}.
\begin{important}
Note that if you have a custom \filetype{.xdy} file created when using 
\styfmt{glossaries} version 2.07 or below, you will need to use the
\pkgopt{compatible-2.07} package option with it.
\end{important}

The default name for the customised style file is given by
\ics{jobname}\filetype{.ist} (for \gls{makeindex}) or 
\ics{jobname}\filetype{.xdy} (for \gls{xindy}). This name may be
changed using:
\begin{definition}[\DescribeMacro{\setStyleFile}]
\cs{setStyleFile}\marg{name}
\end{definition}
where \meta{name} is the name of the style file without the 
extension. Note that this command must be used before
\cs{makeglossaries}.

Each glossary entry is assigned a \gls{numberlist} that lists all 
the locations in the document where that entry was used. By default,
the location refers to the page number but this may be overridden
using the \pkgopt{counter} package option. The default form of
the location number assumes a full stop compositor (e.g.\ 1.2),
but if your location numbers use a different compositor (e.g. 1-2)
you need to set this using
\begin{definition}[\DescribeMacro{\glsSetCompositor}]
\cs{glsSetCompositor}\marg{symbol}
\end{definition}
For example:
\begin{verbatim}
\glsSetCompositor{-}
\end{verbatim}
Note that this command must be used before \cs{makeglossaries}.

If you use \gls{xindy}, you can have a different compositor for page
numbers starting with an uppercase alphabetical character using:
\begin{definition}[\DescribeMacro{\glsSetAlphaCompositor}]
\cs{glsSetAlphaCompositor}\marg{symbol}
\end{definition}
Note that this command has no effect if you haven't used the
\pkgopt{xindy} package option. For example, if you want 
\glspl{numberlist} containing a mixture of A-1 and 2.3 style
formats, then do:
\begin{verbatim}
\glsSetCompositor{.}\glsSetAlphaCompositor{-}
\end{verbatim}
See \sectionref{sec:numberlists} for further information about
\glspl{numberlist}.

\chapter{Defining Glossary Entries}
\label{sec:newglosentry}

All glossary entries must be defined before they are used, so it is
better to define them in the preamble to ensure this.\footnote{The
only preamble restriction on \ics{newglossaryentry} and
\ics{newacronym} was removed in version 1.13, but the restriction
remains for \ics{loadglsentries}.} However only those entries that
occur in the document (using any of the commands described in
\sectionref{sec:glslink}, \sectionref{sec:glsadd} or
\sectionref{sec:crossref}) will appear in the glossary. Each time an
entry is used in this way, a line is added to an associated glossary
file (\filetype{.glo}), which then needs to be converted into a
corresponding \filetype{.gls} file which contains the typeset
glossary which is input by \ics{printglossary} or
\ics{printglossaries}. The Perl script \gls{makeglossaries} can be
used to call \gls{makeindex} or \gls{xindy}, using a customised
indexing style file, for each of the glossaries that are defined in
the document. \textbf{Note that there should be no need for you to
explicitly edit or input any of these external
files.}\footnote{Except possibly the style file but then you'll need
to use \ics{noist} to prevent your changes from being
overwritten.} See
\sectionref{sec:makeglossaries} for further details.

New glossary entries are defined using the command:
\begin{definition}[\DescribeMacro{\newglossaryentry}]
\cs{newglossaryentry}\marg{label}\marg{key-val list}
\end{definition}
The first argument, \meta{label}, must be a unique label with which
to identify this entry. The second argument, \meta{key-val list}, is
a \meta{key}=\meta{value} list that supplies the relevant
information about this entry. There are two required fields:
\gloskey{description} and either \gloskey{name} or \gloskey{parent}.
Available fields are listed below:

\begin{description}
\item[{\gloskey{name}}] The name of the entry (as it will appear in 
the glossary). If this key is omitted and the \gloskey{parent}
key is supplied, this value will be the same as the parent's name.

\item[{\gloskey{description}}] A brief description of this term (to
appear in the glossary). Within this value, you can use 
\begin{definition}[\DescribeMacro{\nopostdesc}]
\cs{nopostdesc}
\end{definition}
to suppress the
description terminator for this entry. For example, if this
entry is a parent entry that doesn't require a description, you
can do \verb|description={\nopostdesc}|. If you want a paragraph
break in the description use
\begin{definition}[\DescribeMacro{\glspar}]
\cs{glspar}
\end{definition}
However, note that not all glossary styles support multi-line
descriptions. If you are using one of the tabular-like glossary
styles that permit multi-line descriptions, use \ics{newline}
not \verb"\\" if you want to force a line break.

\item[{\gloskey{parent}}] The label of the parent entry. Note that
the parent entry must be defined before its sub-entries.
See \sectionref{sec:subentries} for further details.

\item[{\gloskey{descriptionplural}}] The plural form of the
description (as passed to \ics{glsdisplay} and \ics{glsdisplayfirst}
by \ics{glspl}, \ics{Glspl} and \ics{GLSpl}). If omitted, the value
is set to the same as the \gloskey{description} key. (Note that if
you want the description to appear in the main body of the document,
you must switch off the description \glsation{sanitize} using the
\pkgopt{sanitize} package option described in
\sectionref{sec:pkgopts-general}.)

\item[{\gloskey{text}}] How this entry will appear in the document text
when using \ics{gls} (or one of its uppercase variants). If this
field is omitted, the value of the \gloskey{name} key is used.

\item[{\gloskey{first}}] How the entry will appear in the document text
on \firstuse\ with \ics{gls} (or one of its uppercase
variants). If this field is omitted, the value of the \gloskey{text}
key is used. Note that if you use \ics{glspl}, \ics{Glspl},
\ics{GLSpl}, \ics{glsdisp} before using \ics{gls}, the
\gloskey{firstplural} value won't be used with \ics{gls}.

\item[{\gloskey{plural}}] How the entry will appear in the document text
when using \ics{glspl} (or one of its uppercase variants).  If this
field is omitted, the value is obtained by appending
\ics{glspluralsuffix} to the value of the \gloskey{text} field.  The
default value of \ics{glspluralsuffix} is the letter \qt{s}.

\item[{\gloskey{firstplural}}] How the entry will appear in the
document text on \firstuse\ with \ics{glspl} (or one of its
uppercase variants). If this field is omitted, the value is obtained
from the \gloskey{plural} key, if the \gloskey{first} key is omitted,
or by appending \ics{glspluralsuffix} to the value of the
\gloskey{first} field, if the \gloskey{first} field is present. Note
that if you use \ics{gls}, \ics{Gls}, \ics{GLS}, \cs{glsdisp} before
using \ics{glspl}, the \gloskey{firstplural} value won't be used
with \ics{glspl}.

\textbf{Note:} prior to version 1.13, the default value of 
\gloskey{firstplural} was always taken by appending \qt{s} to the
\gloskey{first} key, which meant that you had to specify both
\gloskey{plural} and \gloskey{firstplural}, even if you hadn't
used the \gloskey{first} key.

\item[{\gloskey{symbol}}] This field is provided to allow the user
to specify an associated symbol. If omitted, the value is set to
\cs{relax}. Note that not all glossary styles display the symbol.
(If you want the symbol to appear in the main body of the document,
you must switch off the symbol \glsation{sanitize} using the
\pkgopt{sanitize} package option described in
\sectionref{sec:pkgopts-general}.)

\item[{\gloskey{symbolplural}}] This is the plural form of the
symbol (as passed to \ics{glsdisplay} and \ics{glsdisplayfirst}
by \ics{glspl}, \ics{Glspl} and \ics{GLSpl}). If omitted, the value
is set to the same as the \gloskey{symbol} key.

\item[{\gloskey{sort}}] This value indicates how \gls{makeindex} or
\gls{xindy} should sort this entry. If omitted, the value is given
by the \gloskey{name} field. In general, it's best to use the
\gloskey{sort} key if the \gloskey{name} contains commands (e.g.\
\verb|\ensuremath{\alpha}|). Note that the package options
\pkgopt[def]{sort} and \pkgopt[use]{sort} override the
\gloskey{sort} key in \ics{newglossaryentry} (see
\sectionref{sec:pkgopts-sort}).

\item[{\gloskey{type}}] This specifies the label of the glossary in 
which this entry belongs. If omitted, the default glossary is 
assumed unless \ics{newacronym} is used (see
\sectionref{sec:acronyms}).

\item[{\gloskey{user1}, \ldots, \gloskey{user6}%
\igloskey{user2}\igloskey{user3}\igloskey{user4}\igloskey{user5}}]
Six keys provided for any additional information the user may want
to specify.  (For example an associated dimension or an alternative
plural or some other grammatical construct.)

\item[{\gloskey{nonumberlist}}] A boolean key. If the value is
missing or is "true", this will suppress the \gls{numberlist} just for
this entry. Conversely, if you have used the package option
\pkgopt{nonumberlist}, you can activate the \gls*{numberlist} just
for this entry with \gloskey{nonumberlist=false}.
(See \sectionref{sec:numberlists}.)

\item[{\gloskey{see}}] Cross-reference another entry. Using the
\gloskey{see} key will automatically add this entry to the
glossary, but will not automatically add the cross-referenced entry.
The referenced entry should be supplied as the value to this key.
If you want to override the \qt{see} tag, you can supply the new
tag in square brackets before the label. For example
\verb|see=[see also]{anotherlabel}|. Note that if you have
suppressed the \gls{numberlist}, the cross-referencing information
won't appear in the glossary. You can override this for individual
glossary entries using \gloskey{nonumberlist=false} (see above).
Alternatively, you can use the \pkgopt{seeautonumberlist} package
option.  For further details, see \sectionref{sec:crossref}.

\end{description}

The following keys are reserved for \ics{newacronym} (see
\sectionref{sec:acronyms}): \gloskey{long}, \gloskey{longplural},
\gloskey{short} and \gloskey{shortplural}.

Note that if the name starts with an accented letter or non-Roman
character, you must group the character, otherwise it will 
cause a problem for commands like \ics{Gls} and \ics{Glspl}. 
For example:
\begin{verbatim}
\newglossaryentry{elite}{name={{\'e}lite},
description={select group or class}}
\end{verbatim}
Note that the same applies if you are using the \sty{inputenc}
package:
\begin{verbatim}
\newglossaryentry{elite}{name={{é}lite},
description={select group or class}}
\end{verbatim}
Note that in both of the above examples, you will also need to
supply the \gloskey{sort} key if you are using \gls{makeindex}
whereas \gls{xindy} is usually able to sort accented letters
correctly.

\section{Plurals}
\label{sec:plurals}

You may have noticed from above that you can specify the plural form
when you define a term. If you omit this, the plural will be 
obtained by appending 
\begin{definition}[\DescribeMacro{\glspluralsuffix}]
\cs{glspluralsuffix}
\end{definition}
to the singular form. This command defaults to the letter \qt{s}. 
For example:
\begin{verbatim}
\newglossaryentry{cow}{name=cow,description={a fully grown
female of any bovine animal}}
\end{verbatim}
defines a new entry whose singular form is \qt{cow} and plural form
is \qt{cows}. However, if you are writing in archaic English, you
may want to use \qt{kine} as the plural form, in which case you
would have to do:
\begin{verbatim}
\newglossaryentry{cow}{name=cow,plural=kine,
description={a fully grown female of any bovine animal}}
\end{verbatim}

If you are writing in a language that supports
multiple plurals (for a given term) then use the \gloskey{plural} key
for one of them and one of the user keys to specify the
other plural form. For example:
\begin{verbatim}
\newglossaryentry{cow}{name=cow,description={a fully grown
female of any bovine animal (plural cows, archaic plural kine)},
user1={kine}}
\end{verbatim}
You can then use \verb|\glspl{cow}| to produce \qt{cows} and 
\verb|\glsuseri{cow}| to produce \qt{kine}. You can, of course,
define an easy to remember synonym. For example:
\begin{verbatim}
\let\glsaltpl\glsuseri
\end{verbatim}
Then you don't have to remember which key you used to store the
alternative plural.

If you are using a language that usually forms plurals
by appending a different letter, or sequence of letters, you can
redefine \cs{glspluralsuffix} as required. However, this must be
done \emph{before} the entries are defined. For languages that don't
form plurals by simply appending a suffix, all the plural forms must be 
specified using the \gloskey{plural} key (and the \gloskey{firstplural}
key where necessary). 

\section{Other Grammatical Constructs}
\label{sec:grammar}

You can use the six user keys to provide alternatives, such as
participles. For example:
\begin{verbatim}
\let\glsing\glsuseri
\let\glsd\glsuserii

\newcommand*{\ingkey}{user1}
\newcommand*{\edkey}{user2}

\newcommand*{\newterm}[3][]{%
  \newglossaryentry{#2}{%
   name={#2},%
   description={#3},%
   \edkey={#2ed},%
   \ingkey={#2ing},#1%
  }%
}
\end{verbatim}
With the above definitions, I can now define terms like this:
\begin{verbatim}
\newterm{play}{to take part in activities for enjoyment}
\newterm[\edkey={ran},\ingkey={running}]{run}{to move fast using 
the legs}
\end{verbatim}
and use them in the text:
\begin{verbatim}
Peter is \glsing{play} in the park today.

Jane \glsd{play} in the park yesterday.

Peter and Jane \glsd{run} in the park last week.
\end{verbatim}

\section{Sub-Entries}
\label{sec:subentries}

As from version 1.17, it is possible to specify sub-entries. These
may be used to order the glossary into categories, in which case the
sub-entry will have a different name to its parent entry, or it may
be used to distinguish different definitions for the same word, in
which case the sub-entries will have the same name as the parent
entry. Note that not all glossary styles support hierarchical
entries and may display all the entries in a flat format. Of the
styles that support sub-entries, some display the sub-entry's name
whilst others don't. Therefore you need to ensure that you use a
suitable style. (See \sectionref{sec:styles} for a list of predefined
styles.) As from version 3.0, level~1 sub-entries are automatically
numbered in the predefined styles if you use the
\pkgopt{subentrycounter} package option (see
\sectionref{sec:pkgopts-printglos} for further details).

Note that the parent entry will automatically be added to the
glossary if any of its child entries are used in the document.
If the parent entry is not referenced in the document, it will not
have a \gls{numberlist}. Note also that \gls{makeindex} has a
restriction on the maximum sub-entry depth.

\subsection{Hierarchical Categories}
\label{sec:hierarchical}

To arrange a glossary with hierarchical categories, you need to
first define the category and then define the sub-entries using the
relevant category entry as the value of the \gloskey{parent} key.
For example, suppose I want a glossary of mathematical symbols that
are divided into Greek letters and Roman letters. Then I can define
the categories as follows:
\begin{verbatim}
\newglossaryentry{greekletter}{name={Greek letters},
description={\nopostdesc}}

\newglossaryentry{romanletter}{name={Roman letters},
description={\nopostdesc}}
\end{verbatim}

Note that in this example, the category entries don't need a
description so I have set the descriptions to \ics{nopostdesc}.
This gives a blank description and suppresses the
description terminator.

I can now define my sub-entries as follows:
\begin{verbatim}
\newglossaryentry{pi}{name={\ensuremath{\pi}},sort={pi},
description={ratio of the circumference of a circle to 
the diameter},
parent=greekletter}

\newglossaryentry{C}{name={\ensuremath{C}},sort={C},
description={Euler's constant},
parent=romanletter}
\end{verbatim}


\subsection{Homographs}
\label{sec:homographs}

Sub-entries that have the same name as the parent entry, don't need
to have the \gloskey{name} key. For example, the word \qt{glossary}
can mean a list of technical words or a collection of glosses. In
both cases the plural is \qt{glossaries}. So first define the parent
entry:
\begin{verbatim}
\newglossaryentry{glossary}{name=glossary,
description={\nopostdesc},
plural={glossaries}}
\end{verbatim}
Again, the parent entry has no description, so the description
terminator needs to be suppressed using \ics{nopostdesc}.

Now define the two different meanings of the word:
\begin{verbatim}
\newglossaryentry{glossarylist}{
description={list of technical words},
sort={1},
parent={glossary}}

\newglossaryentry{glossarycol}{
description={collection of glosses},
sort={2},
parent={glossary}}
\end{verbatim}
Note that if I reference the parent entry, the location will be
added to the parent's \gls{numberlist}, whereas if I reference any
of the child entries, the location will be added to the child
entry's number list. Note also that since the sub-entries have the
same name, the \gloskey{sort} key is required unless you are using
the \pkgopt[use]{sort} or \pkgopt[def]{sort} package options. You
can use the \pkgopt{subentrycounter} package option to automatically
number the first-level child entries. See
\sectionref{sec:pkgopts-printglos} for further details.

In the above example, the plural form for both of the child entries
is the same as the parent entry, so the \gloskey{plural} key was
not required for the child entries. However, if the sub-entries
have different plurals, they will need to be specified. For example:
\begin{verbatim}
\newglossaryentry{bravo}{name={bravo},
description={\nopostdesc}}

\newglossaryentry{bravocry}{description={cry of approval 
(pl.\ bravos)},
sort={1},
plural={bravos},
parent=bravo}

\newglossaryentry{bravoruffian}{description={hired 
ruffian or killer (pl.\ bravoes)},
sort={2},
plural={bravoes},
parent=bravo}
\end{verbatim}


\section{Loading Entries From a File}
\label{sec:loadglsentries}

You can store all your glossary entry definitions in another
file and use:
\begin{definition}[\DescribeMacro{\loadglsentries}]
\cs{loadglsentries}\oarg{type}\marg{filename}
\end{definition}
where \meta{filename} is the name of the file containing all the
\ics{newglossaryentry} commands. The optional argument
\meta{type} is the name of the glossary to which those entries
should belong, for those entries where the \gloskey{type} key has 
been omitted (or, more specifically, for those entries whose
type has been specified by \ics{glsdefaulttype}, which is what
\ics{newglossaryentry} uses by default). For example, suppose I have
a file called \texttt{myentries.tex} which contains:
\begin{verbatim}
\newglossaryentry{perl}{type=main,
name={Perl},
description={A scripting language}}

\newglossaryentry{tex}{name={\TeX},
description={A typesetting language},sort={TeX}}

\newglossaryentry{html}{type=\glsdefaulttype,
name={html},
description={A mark up language}}
\end{verbatim}
and suppose in my document preamble I use the command:
\begin{verbatim}
\loadglsentries[languages]{myentries}
\end{verbatim}
then this will add the entries \texttt{tex} and \texttt{html}
to the glossary whose type is given by \texttt{languages}, but
the entry \texttt{perl} will be added to the main glossary, since
it explicitly sets the type to \texttt{main}.

\textbf{Note:} if you use \ics{newacronym} (see
\sectionref{sec:acronyms}) the type is set as
\verb|type=\acronymtype| unless you explicitly override it. For
example, if my file \texttt{myacronyms.tex} contains:
\begin{verbatim}
\newacronym{aca}{aca}{a contrived acronym}
\end{verbatim}
then (supposing I have defined a new glossary type called
\texttt{altacronym})
\begin{verbatim}
\loadglsentries[altacronym]{myacronyms}
\end{verbatim}
will add \texttt{aca} to the glossary type \texttt{acronym}, if the
package option \pkgopt{acronym} has been specified, or will add
\texttt{aca} to the glossary type \texttt{altacronym}, if the
package option \pkgopt{acronym} is not specified.\footnote{This is
because \ics{acronymtype} is set to \ics{glsdefaulttype} if the
\pkgopt{acronym} package option is not used.}

If you have used the \pkgopt{acronym} package option,
there are two possible solutions to this problem:
\begin{enumerate}
\item Change \texttt{myacronyms.tex} so that entries are defined in
the form:
\begin{verbatim}
\newacronym[type=\glsdefaulttype]{aca}{aca}{a 
contrived acronym}
\end{verbatim}
and do:
\begin{verbatim}
\loadglsentries[altacronym]{myacronyms}
\end{verbatim}

\item Temporarily change \cs{acronymtype} to the target glossary:
\begin{verbatim}
\let\orgacronymtype\acronymtype
\def\acronymtype{altacronym}
\loadglsentries{myacronyms}
\let\acronymtype\orgacronymtype
\end{verbatim}
\end{enumerate}

Note that only those entries that have been used
in the text will appear in the relevant glossaries.
Note also that \cs{loadglsentries} may only be used in the 
preamble.

\section{Moving Entries to Another Glossary}
\label{sec:moveentry}

As from version~3.02, you can move an entry from one glossary to
another using:
\begin{definition}[\DescribeMacro{\glsmoveentry}]
\cs{glsmoveentry}\marg{label}\marg{target glossary label}
\end{definition}
where \meta{label} is the unique label identifying the required
entry and \meta{target glossary label} is the unique label
identifying the glossary in which to put the entry.

Note that no check is performed to determine the existence of
the target glossary. This means that you can, for example, move an
entry to an undefined glossary so you can use the entry in the
document text but not have it listed in any of the glossaries.
(Maybe you have an acronym so common it doesn't need listing.)

\begin{important}
If you move an entry to an undefined glossary and you have
hyperlinked entries, the link will point to an undefined target.
(Unless you identify that glossary using \pkgopt{nohypertypes} or
\ics{GlsDeclareNoHyperList}, as described in \sectionref{sec:glslink}.)
Also, you will get warnings about no file defined for that glossary
(unless you use the \pkgopt{nowarn} package option). Unpredictable
results may occur if you move an entry to a different glossary from
its parent or children.
\end{important}

\chapter{Number lists}
\label{sec:numberlists}

Each entry in the glossary has an associated \gls{numberlist}.
By default, these numbers refer to the pages on which that entry has
been used (using any of the commands described in
\sectionref{sec:glslink} and \sectionref{sec:glsadd}). The number
list can be suppressed using the \pkgopt{nonumberlist} package
option, or an alternative counter can be set as the default using
the \pkgopt{counter} package option. The number list is also
referred to as the location list\index{location list|see{number
list}}.

Both \gls{makeindex} and \gls{xindy} concatenate a
sequence of 3 or more consecutive pages into a range. With 
\gls*{xindy} you can vary the minimum sequence length using
\ics{GlsSetXdyMinRangeLength}\marg{n} where \meta{n} is either
an integer or the keyword \texttt{none} which indicates that there
should be no range formation.

\begin{important}
Note that \cs{GlsSetXdyMinRangeLength} must be used before
\ics{makeglossaries} and has no effect if \ics{noist} is used.
\end{important}

With both \gls{makeindex} and \gls{xindy}, you can replace
the separator and the closing number in the range using:
\begin{definition}[\DescribeMacro{\glsSetSuffixF}]
\cs{glsSetSuffixF}\marg{suffix}
\end{definition}
\begin{definition}[\DescribeMacro{\glsSetSuffixFF}]
\cs{glsSetSuffixFF}\marg{suffix}
\end{definition}
where the former command specifies the suffix to use for a 2 page
list and the latter specifies the suffix to use for longer lists.
For example:
\begin{verbatim}
\glsSetSuffixF{f.}
\glsSetSuffixFF{ff.}
\end{verbatim}
Note that if you use \gls{xindy}, you will also need to
set the minimum range length to 1 if you want to change these
suffixes:
\begin{verbatim}
\GlsSetXdyMinRangeLength{1}
\end{verbatim}
Note that if you use the \sty{hyperref} package, you will need
to use \ics{nohyperpage} in the suffix to ensure that the hyperlinks
work correctly. For example:
\begin{verbatim}
\glsSetSuffixF{\nohyperpage{f.}}
\glsSetSuffixFF{\nohyperpage{ff.}}
\end{verbatim}

\begin{important}
Note that \cs{glsSetSuffixF} and \cs{glsSetSuffixFF} must be used 
before \ics{makeglossaries} and have no effect if \ics{noist} is 
used.
\end{important}


\chapter{Links to Glossary Entries}
\label{sec:glslink}

Once you have defined a glossary entry using \ics{newglossaryentry},
you can refer to that entry in the document using one of the
commands listed in this section. The text which appears at that
point in the document when using one of these commands is referred
to as the \gls{linktext} (even if there are no hyperlinks). The
commands in this section also add a line to an external file that is
used by \gls{makeindex} or \gls{xindy} to generate the relevant
entry in the glossary. This information includes an associated
location that is added to the \gls{numberlist} for that entry. By
default, the location refers to the page number. For further
information on number lists, see \sectionref{sec:numberlists}.

\begin{important}
It is strongly recommended that you don't use the commands
defined in this section in the arguments of sectioning or caption
commands or any other command that has a moving argument.
\end{important}

The above warning is particularly important if you are using the
\styfmt{glossaries} package in conjunction with the \sty{hyperref}
package. Instead, use one of the commands listed in
\sectionref{sec:glsnolink} (such as \ics{glsentrytext}) or provide an
alternative via the optional argument to the sectioning/caption
command. Examples:
\begin{verbatim}
\chapter{An overview of \glsentrytext{perl}}
\chapter[An overview of Perl]{An overview of \gls{perl}}
\end{verbatim}

If you want the \gls{linktext} to produce a hyperlink to the
corresponding entry details in the glossary, you should load the
\sty{hyperref} package \emph{before} the \styfmt{glossaries}
package. That's what I've done in this document, so if you see a
hyperlinked term, such as \gls{linktext}, you can click on the word
or phrase and it will take you to a brief description in this
document's glossary.

It may be that you only want terms in a certain glossary to have
links, but not for another glossary. In which case, you can use the
package option \pkgopt{nohypertypes} to identify the glossary lists
that shouldn't have hyperlinked \gls{linktext}. For example, suppose
your document contains lots of technical acronyms that the reader
might not know, but it also contains some very common acronyms that
most readers will recognise. So you might want two acronym lists,
but only the technical list will get displayed in your document. The
technical acronyms can be hyperlinked to that list, but common
acronyms shouldn't have hyperlinks as there's no target for them. In
this case, identify the common acronym list as having
non-hyperlinked entries using \pkgopt{nohypertypes}. Example:
\begin{verbatim}
\usepackage[acronym,nohypertypes={common}]{glossaries}
\newglossary{common}{cacr}{cacn}{Common Acronyms}
\end{verbatim}
Alternatively, you can use
\begin{definition}[\DescribeMacro{\GlsDeclareNoHyperList}]
\cs{GlsDeclareNoHyperList}\marg{type}
\end{definition}
For example:
\begin{verbatim}
\usepackage[acronym]{glossaries}
\newglossary{common}{cacr}{cacn}{Common Acronyms}
\GlsDeclareNoHyperList{common}
\end{verbatim}
Note that no check is performed to see if the glossary types listed
in \pkgopt{nohypertypes} or \cs{GlsDeclareNoHyperList} have been
defined.

The way the \gls{linktext} is displayed depends on 
\begin{definition}[\DescribeMacro{\glstextformat}]
\cs{glstextformat}\marg{text}
\end{definition}
For example, to make all \gls{linktext} appear in a sans-serif
font, do:
\begin{verbatim}
\renewcommand*{\glstextformat}[1]{\textsf{#1}}
\end{verbatim}

Each entry has an associated conditional referred to as the
\firstuseflag. This determines whether \ics{gls}, 
\ics{glspl} (and their uppercase variants) should use the
value of the \gloskey{first} or \gloskey{text} keys. Note that an 
entry can be used without affecting the \firstuseflag\ (for example,
when used with \cs{glslink}). See \sectionref{sec:glsunset} for 
commands that unset or reset this conditional.

The command:
\begin{definition}[\DescribeMacro{\glslink}]
\cs{glslink}\oarg{options}\marg{label}\marg{text}
\end{definition}
will place \cs{glstextformat}\marg{text} in the document at that
point and add a line into the associated glossary file for the
glossary entry given by \meta{label}. If hyperlinks are supported,
\meta{text} will be a hyperlink to the relevant line in the
glossary. (Note that this command doesn't affect the \firstuseflag:
use \ics{glsdisp} instead.) The optional argument \meta{options}
must be a \meta{key}=\meta{value} list which can take any of the
following keys:

\begin{description}
\item[{\gloskey[glslink]{format}}] This specifies how to format the
associated location number for this entry in the glossary. This 
value is equivalent to the \gls{makeindex} encap value, and (as
with \ics{index}) the value needs to be the name of a command
\emph{without} the initial backslash. As with \ics{index}, the
characters \verb"(" and \verb")" can also be used to specify the
beginning and ending of a number range.  Again as with \ics{index},
the command should be the name of a command which takes an argument
(which will be the associated location).  Be careful not to use a
declaration (such as \texttt{bfseries}) instead of a text block command
(such as \texttt{textbf}) as the effect is not guaranteed to be
localised. If you want to apply more than one style to a given entry
(e.g.\ \textbf{bold} and \emph{italic}) you will need to create a
command that applies both formats, e.g.\ 
\begin{verbatim}
\newcommand*{\textbfem}[1]{\textbf{\emph{#1}}}
\end{verbatim}
and use that command.

In this document, the standard formats refer to the standard text 
block commands such as \ics{textbf} or \ics{emph} or any of the 
commands listed in \tableref{tab:hyperxx}.

\begin{important}
If you use \gls{xindy} instead of \gls{makeindex}, you
must specify any non-standard formats that you want to use
with the \gloskey[glslink]{format} key using
\ics{GlsAddXdyAttribute}\marg{name}. So if you use
\gls*{xindy} with the above example, you would need to add:
\begin{verbatim}
\GlsAddXdyAttribute{textbfem}
\end{verbatim}
See \sectionref{sec:xindy} for further details.
\end{important}

Note that unlike \ics{index}, you can't have anything following the
command name, such as an asterisk or arguments. If you want to
cross-reference another entry, either use the \gloskey{see} key when
you define the entry or use \ics{glssee} (described in
\sectionref{sec:crossref}).


If you are using hyperlinks and you want to change the font of the
hyperlinked location, don't use \ics{hyperpage} (provided by the
\sty{hyperref} package) as the locations may not refer to a page
number.  Instead, the \styfmt{glossaries} package provides number
formats listed in \tableref{tab:hyperxx}.

\begin{table}[htbp]
\caption{Predefined Hyperlinked Location Formats}
\label{tab:hyperxx}
\centering
\vskip\baselineskip
\begin{tabular}{ll}
\locfmt{hyperrm} & serif hyperlink\\
\locfmt{hypersf} & sans-serif hyperlink\\
\locfmt{hypertt} & monospaced hyperlink\\
\locfmt{hyperbf} & bold hyperlink\\
\locfmt{hypermd} & medium weight hyperlink\\
\locfmt{hyperit} & italic hyperlink\\
\locfmt{hypersl} & slanted hyperlink\\
\locfmt{hyperup} & upright hyperlink\\
\locfmt{hypersc} & small caps hyperlink\\
\locfmt{hyperemph} & emphasized hyperlink
\end{tabular}
\par
\end{table}

Note that if the \ics{hyperlink} command hasn't been defined, the
\texttt{hyper}\meta{xx} formats are equivalent to the analogous
\texttt{text}\meta{xx} font commands (and \texttt{hyperemph} is
equivalent to \texttt{emph}). If you want to make a new format, you
will need to define a command which takes one argument and use that.
For example, if you want the location number to be in a bold
sans-serif font, you can define a command called, say,
\ics{hyperbsf}:
\begin{verbatim}
\newcommand{\hyperbsf}[1]{\textbf{\hypersf{#1}}}
\end{verbatim}
and then use \texttt{hyperbsf} as the value for the \gloskey{format}
key. (See also \ifpdf section~\ref*{sec:code:printglos} \fi
\qt{Displaying the glossary} in the documented code,
\texttt{glossaries.pdf}.) Remember that if you use \gls{xindy}, you
will need to add this to the list of location attributes:
\begin{verbatim}
\GlsAddXdyAttribute{hyperbsf}
\end{verbatim}

\item[{\gloskey[glslink]{counter}}] This specifies which counter
to use for this location. This overrides the default counter
used by this entry. (See also \sectionref{sec:numberlists}.)

\item[{\gloskey[glslink]{hyper}}] This is a boolean key which can
be used to enable/disable the hyperlink to the relevant entry
in the glossary. (Note that setting \texttt{hyper=true} will have no
effect if \ics{hyperlink} has not been defined.) The default
value is \texttt{hyper=true}, unless the entry belongs to a glossary
that either has been listed in the package option
\pkgopt{nohypertypes} or has been identified using
\ics{GlsDeclareNoHyperList} in which case the default is
\texttt{hyper=false}.

\item[\gloskey{glslink}{local}] This is a boolean key that only
makes a different when used with commands that change the 
entry's \gls{firstuseflag} (such as \ics{gls}). If
\texttt{local=true}, the change to the \gls*{firstuseflag} will be
localised to the current scope. The default is \texttt{local=false}.

\end{description}
There is also a starred version:
\begin{definition}[\DescribeMacro{\glslink*}]
\cs{glslink*}\oarg{options}\marg{label}\marg{text}
\end{definition}
which is equivalent to \cs{glslink}, except it sets
\texttt{hyper=false}. Similarly, all the following commands 
described in this section also have a starred version that disables
the hyperlink.

\begin{important}
Don't use commands like \cs{glslink} or \cs{gls} in the \meta{text} 
argument of \cs{glslink}.
\end{important}

The command:
\begin{definition}[\DescribeMacro{\gls}]
\cs{gls}\oarg{options}\marg{label}\oarg{insert}
\end{definition}
is the same as \cs{glslink}, except that the \gls{linktext}
is determined from the values of the \gloskey{text} and 
\gloskey{first} keys supplied when the entry was defined using
\ics{newglossaryentry}. If the entry has been marked as having
been used\ifirstuseflag, the value of the \gloskey{text} key will be used, otherwise
the value of the \gloskey{first} key will be used. 
On completion, \cs{gls} will mark the entry's \gls{firstuseflag} as
used.

There are two uppercase variants:
\begin{definition}[\DescribeMacro{\Gls}]
\cs{Gls}\oarg{options}\marg{label}\oarg{insert}
\end{definition}
and
\begin{definition}[\DescribeMacro{\GLS}]
\cs{GLS}\oarg{options}\marg{label}\oarg{insert}
\end{definition}
which make the first letter of the link text or all the link text
uppercase, respectively.

The final optional argument \meta{insert}, allows you to insert
some additional text into the link text. By default, this will
append \meta{insert} at the end of the link text, but this
can be changed (see \sectionref{sec:glsdisplay}). 

The first optional argument \meta{options} is the same as the optional
argument to \ics{glslink}. As with \ics{glslink}, these commands also
have a starred version that disable the hyperlink.

\begin{important}
Don't use commands like \cs{glslink} or \cs{gls} in the \meta{insert} 
argument of \cs{gls} and its variants.
\end{important}

There are also analogous plural forms:
\begin{definition}[\DescribeMacro{\glspl}]
\cs{glspl}\oarg{options}\marg{label}\oarg{insert}
\end{definition}
\begin{definition}[\DescribeMacro{\Glspl}]
\cs{Glspl}\oarg{options}\marg{label}\oarg{insert}
\end{definition}
\begin{definition}[\DescribeMacro{\GLSpl}]
\cs{GLSpl}\oarg{options}\marg{label}\oarg{insert}
\end{definition}
These determine the link text from the \gloskey{plural} and
\gloskey{firstplural} keys supplied when the entry was first
defined. As before, these commands also have
a starred version that disable the hyperlink.

\begin{important}
Be careful when you use glossary entries in math mode especially if you 
are using \sty{hyperref} as it can affect the spacing of subscripts and
superscripts. For example, suppose you have defined the following
entry:
\begin{verbatim}
\newglossaryentry{Falpha}{name={F_\alpha},
description=sample}
\end{verbatim}
and later you use it in math mode:
\begin{verbatim}
$\gls{Falpha}^2$
\end{verbatim}
This will result in $F_\alpha{}^2$ instead of $F_\alpha^2$. In this
situation it's best to bring the superscript into the hyperlink using
the final \meta{insert} optional argument:
\begin{verbatim}
$\gls{Falpha}[^2]$
\end{verbatim}
\end{important}

Note that \ics{glslink} doesn't use or affect the \firstuseflag,
nor does it use \ics{glsdisplay} or \ics{glsdisplayfirst} (see 
\sectionref{sec:glsdisplay}). Instead, you can use:
\begin{definition}[\DescribeMacro{\glsdisp}]
\cs{glsdisp}\oarg{options}\marg{label}\marg{link text}
\end{definition}
This behaves in the same way as \ics{gls}, except that it uses
\meta{link text} instead of the value of the \gloskey{first}
or \gloskey{text} key. (Note that this command always sets
\meta{insert} to nothing.) This command affects the \firstuseflag,
and uses \ics{glsdisplay} or \ics{glsdisplayfirst}.

The command:
\begin{definition}[\DescribeMacro{\glstext}]
\cs{glstext}\oarg{options}\marg{label}\oarg{insert}
\end{definition}
is similar to \ics{gls} except that it always uses the value
of the \gloskey{text} key and does not affect the 
\firstuseflag. 
Unlike \cs{gls}, the inserted text \meta{insert} 
is always appended to the link text since \cs{glstext} doesn't
use \ics{glsdisplay} or \ics{glsdisplayfirst}. (The same is 
true for all the following commands described in this section.)

There are also analogous commands:
\begin{definition}[\DescribeMacro{\Glstext}]
\cs{Glstext}\oarg{options}\marg{text}\oarg{insert}
\end{definition}
\begin{definition}[\DescribeMacro{\GLStext}]
\cs{GLStext}\oarg{options}\marg{text}\oarg{insert}
\end{definition}
As before, these commands also have a starred version that disable
the hyperlink.

The command:
\begin{definition}[\DescribeMacro{\glsfirst}]
\cs{glsfirst}\oarg{options}\marg{label}\oarg{insert}
\end{definition}
is similar to \cs{glstext} except that it always uses the value
of the \gloskey{first} key. Again, \meta{insert} is always appended
to the end of the link text and does not affect the 
\firstuseflag.

There are also analogous commands:
\begin{definition}[\DescribeMacro{\Glsfirst}]
\cs{Glsfirst}\oarg{options}\marg{text}\oarg{insert}
\end{definition}
\begin{definition}[\DescribeMacro{\GLSfirst}]
\cs{GLSfirst}\oarg{options}\marg{text}\oarg{insert}
\end{definition}
As before, these commands also have a starred version that disable
the hyperlink.

The command:
\begin{definition}[\DescribeMacro{\glsplural}]
\cs{glsplural}\oarg{options}\marg{label}\oarg{insert}
\end{definition}
is similar to \ics{glstext} except that it always uses the value
of the \gloskey{plural} key. Again, \meta{insert} is always appended
to the end of the link text and does not mark the entry as having
been used.

There are also analogous commands:
\begin{definition}[\DescribeMacro{\Glsplural}]
\cs{Glsplural}\oarg{options}\marg{text}\oarg{insert}
\end{definition}
\begin{definition}[\DescribeMacro{\GLSplural}]
\cs{GLSplural}\oarg{options}\marg{text}\oarg{insert}
\end{definition}
As before, these commands also have a starred version that disable
the hyperlink.

The command:
\begin{definition}[\DescribeMacro{\glsfirstplural}]
\cs{glsfirstplural}\oarg{options}\marg{label}\oarg{insert}
\end{definition}
is similar to \ics{glstext} except that it always uses the value
of the \gloskey{firstplural} key. Again, \meta{insert} is always 
appended to the end of the link text and does not mark the entry as
having been used.

There are also analogous commands:
\begin{definition}[\DescribeMacro{\Glsfirstplural}]
\cs{Glsfirstplural}\oarg{options}\marg{text}\oarg{insert}
\end{definition}
\begin{definition}[\DescribeMacro{\GLSfirstplural}]
\cs{GLSfirstplural}\oarg{options}\marg{text}\oarg{insert}
\end{definition}
As before, these commands also have a starred version that disable
the hyperlink.

The command:
\begin{definition}[\DescribeMacro{\glsname}]
\cs{glsname}\oarg{options}\marg{label}\oarg{insert}
\end{definition}
is similar to \ics{glstext} except that it always uses the value of
the \gloskey{name} key. Again, \meta{insert} is always appended to
the end of the link text and does not mark the entry as having been
used. Note: if you want to use this command and the \gloskey{name}
key contains commands, you will have to disable the
\glsation{sanitize} of the \gloskey{name} key and
protect fragile commands.

There are also analogous commands:
\begin{definition}[\DescribeMacro{\Glsname}]
\cs{Glsname}\oarg{options}\marg{text}\oarg{insert}
\end{definition}
\begin{definition}[\DescribeMacro{\GLSname}]
\cs{GLSname}\oarg{options}\marg{text}\oarg{insert}
\end{definition}
As before, these commands also have a starred version that disable
the hyperlink.

The command:
\begin{definition}[\DescribeMacro{\glssymbol}]
\cs{glssymbol}\oarg{options}\marg{label}\oarg{insert}
\end{definition}
is similar to \ics{glstext} except that it always uses the value of
the \gloskey{symbol} key. Again, \meta{insert} is always appended to
the end of the link text and does not mark the entry as having been
used. Note: if you want to use this command and the \gloskey{symbol}
key contains commands, you will have to disable the
\glsation{sanitize} of the \gloskey{symbol} key and
protect fragile commands.

There are also analogous commands:
\begin{definition}[\DescribeMacro{\Glssymbol}]
\cs{Glssymbol}\oarg{options}\marg{text}\oarg{insert}
\end{definition}
\begin{definition}[\DescribeMacro{\GLSsymbol}]
\cs{GLSsymbol}\oarg{options}\marg{text}\oarg{insert}
\end{definition}
As before, these commands also have a starred version that disable
the hyperlink.

The command:
\begin{definition}[\DescribeMacro{\glsdesc}]
\cs{glsdesc}\oarg{options}\marg{label}\oarg{insert}
\end{definition}
is similar to \ics{glstext} except that it always uses the value of
the \gloskey{description} key. Again, \meta{insert} is always
appended to the end of the link text and does not mark the entry as
having been used. Note: if you want to use this command and the
\gloskey{description} key contains commands, you will have to
disable the \glsation{sanitize} of the
\gloskey{description} key and protect fragile commands.

There are also analogous commands:
\begin{definition}[\DescribeMacro{\Glsdesc}]
\cs{Glsdesc}\oarg{options}\marg{text}\oarg{insert}
\end{definition}
\begin{definition}[\DescribeMacro{\GLSdesc}]
\cs{GLSdesc}\oarg{options}\marg{text}\oarg{insert}
\end{definition}
As before, these commands also have a starred version that disable
the hyperlink.

The command:
\begin{definition}[\DescribeMacro{\glsuseri}]
\cs{glsuseri}\oarg{options}\marg{label}\oarg{insert}
\end{definition}
is similar to \ics{glstext} except that it always uses the value
of the \gloskey{user1} key. Again, \meta{insert} is always
appended to the end of the link text and does not mark the entry as 
having been used.

There are also analogous commands:
\begin{definition}[\DescribeMacro{\Glsuseri}]
\cs{Glsuseri}\oarg{options}\marg{text}\oarg{insert}
\end{definition}
\begin{definition}[\DescribeMacro{\GLSuseri}]
\cs{GLSuseri}\oarg{options}\marg{text}\oarg{insert}
\end{definition}
As before, these commands also have a starred version that disable
the hyperlink. Similarly for the other user keys:
\begin{definition}[\DescribeMacro{\glsuserii}]
\cs{glsuserii}\oarg{options}\marg{text}\oarg{insert}
\end{definition}
\begin{definition}[\DescribeMacro{\Glsuserii}]
\cs{Glsuserii}\oarg{options}\marg{text}\oarg{insert}
\end{definition}
\begin{definition}[\DescribeMacro{\GLSuserii}]
\cs{GLSuserii}\oarg{options}\marg{text}\oarg{insert}
\end{definition}
\begin{definition}[\DescribeMacro{\glsuseriii}]
\cs{glsuseriii}\oarg{options}\marg{text}\oarg{insert}
\end{definition}
\begin{definition}[\DescribeMacro{\Glsuseriii}]
\cs{Glsuseriii}\oarg{options}\marg{text}\oarg{insert}
\end{definition}
\begin{definition}[\DescribeMacro{\GLSuseriii}]
\cs{GLSuseriii}\oarg{options}\marg{text}\oarg{insert}
\end{definition}
\begin{definition}[\DescribeMacro{\glsuseriv}]
\cs{glsuseriv}\oarg{options}\marg{text}\oarg{insert}
\end{definition}
\begin{definition}[\DescribeMacro{\Glsuseriv}]
\cs{Glsuseriv}\oarg{options}\marg{text}\oarg{insert}
\end{definition}
\begin{definition}[\DescribeMacro{\GLSuseriv}]
\cs{GLSuseriv}\oarg{options}\marg{text}\oarg{insert}
\end{definition}
\begin{definition}[\DescribeMacro{\glsuserv}]
\cs{glsuserv}\oarg{options}\marg{text}\oarg{insert}
\end{definition}
\begin{definition}[\DescribeMacro{\Glsuserv}]
\cs{Glsuserv}\oarg{options}\marg{text}\oarg{insert}
\end{definition}
\begin{definition}[\DescribeMacro{\GLSuserv}]
\cs{GLSuserv}\oarg{options}\marg{text}\oarg{insert}
\end{definition}
\begin{definition}[\DescribeMacro{\glsuservi}]
\cs{glsuservi}\oarg{options}\marg{text}\oarg{insert}
\end{definition}
\begin{definition}[\DescribeMacro{\Glsuservi}]
\cs{Glsuservi}\oarg{options}\marg{text}\oarg{insert}
\end{definition}
\begin{definition}[\DescribeMacro{\GLSuservi}]
\cs{GLSuservi}\oarg{options}\marg{text}\oarg{insert}
\end{definition}

\section{Changing the format of the link text}
\label{sec:glsdisplay}

The format of the \gls{linktext} for \ics{gls}, \ics{glspl}
and their uppercase variants is governed by two commands:
\begin{definition}[\DescribeMacro{\glsdisplayfirst}]
\cs{glsdisplayfirst}\marg{first/first plural}\marg{description}\marg{symbol}\marg{insert}
\end{definition}
which is used the
\glslink{firstuse}{first time} a glossary entry is used in the text and
\begin{definition}[\DescribeMacro{\glsdisplay}]
\cs{glsdisplay}\marg{text/plural}\marg{description}\marg{symbol}\marg{insert}
\end{definition}
which is used subsequently.
Both commands take four arguments: the first is either the singular or
plural form given by the \gloskey{text}, \gloskey{plural},
\gloskey{first} or \gloskey{firstplural} keys (set when the term was
defined) depending on context; the second argument is the term's
description (as supplied by the \gloskey{description} or
\gloskey{descriptionplural} keys); the third
argument is the symbol associated with the term (as supplied by the
\gloskey{symbol} or \gloskey{symbolplural} keys) and the fourth argument is the additional text
supplied in the final optional argument to \ics{gls} or
\ics{glspl} (or their uppercase variants). The default definitions
of \cs{glsdisplay} and \cs{glsdisplayfirst} simply
print the first argument immediately followed by the fourth argument.
The remaining arguments are ignored.

\begin{important}
Care needs to be taken when redefining \cs{glsdisplay} and
\cs{glsdisplayfirst} as commands like \ics{Gls} will expand the
displayed text before applying \ics{makefirstuc}. If you want to use
formatting commands, it's best to define a robust version that deals
with all the formatting. For example, suppose you want
the text to appear in bold italic, it's best to do something like:
\begin{verbatim}
\DeclareRobustCommand{\textbfit}[1]{\emph{\bfseries 
#1}}
\renewcommand{\glsdisplay}[4]{\textbfit{#1#4}}
\end{verbatim}
See the \sty{mfirstuc} documentation for further details on the
limitations of \ics{makefirstuc}.
\end{important}

If required, you can access the label for the given entry via
\begin{definition}[\DescribeMacro{\glslabel}]
\cs{glslabel}
\end{definition}
so it is possible to use this label in the definition of
\cs{glsdisplay} or \cs{glsdisplayfirst} to supply additional
information using any of the commands described in
\sectionref{sec:glsnolink}, if required.

Note that \cs{glsdisplay} and \cs{glsdisplayfirst} are not used
by \ics{glslink}. If you want to supply your own link text, you
need to use \ics{glsdisp} instead.

For example, suppose you want a glossary of measurements and
units, you can use the \gloskey{symbol} key to store the unit:
\begin{verbatim}
\newglossaryentry{distance}{name=distance,
description={The length between two points},
symbol={km}}
\end{verbatim}
and now suppose you want \verb|\gls{distance}| to produce
\qt{distance (km)} on \firstuse, then you can redefine
\cs{glsdisplayfirst} as follows:
\begin{verbatim}
\renewcommand{\glsdisplayfirst}[4]{#1#4 (#3)}
\end{verbatim}
Note that the additional text is placed after \verb|#1|, so
\verb|\gls{distance}['s]| will produce \qt{distance's (km)}
rather than \qt{distance (km)'s} which looks a bit odd (even though
it may be in the context of \qt{the distance (km) is measured between
the two points} --- but in this instance it would be better not to
use a contraction).

Note also that all of the \gls{linktext} will be formatted according
to \ics{glstextformat} (described earlier). So if you do, say:
\begin{verbatim}
\renewcommand{\glstextformat}[1]{\textbf{#1}}
\renewcommand{\glsdisplayfirst}[4]{#1#4 (#3)}
\end{verbatim}
then \verb|\gls{distance}| will produce \qt{\textbf{distance (km)}}.

If you have multiple glossaries, changing \cs{glsdisplayfirst} and
\cs{glsdisplay} will change the way entries for all of the
glossaries appear when using the commands \ics{gls}, \ics{glspl}, their
uppercase variants and \ics{glsdisp}.  If you only
want the change to affect entries for a given glossary, then you need
to use
\begin{definition}[\DescribeMacro{\defglsdisplay}]
\cs{defglsdisplay}\oarg{type}\marg{definition}
\end{definition}
and
\begin{definition}[\DescribeMacro{\defglsdisplayfirst}]
\cs{defglsdisplayfirst}\oarg{type}\marg{definition}
\end{definition}
instead of redefining \cs{glsdisplay} and \cs{glsdisplayfirst}.

Both \cs{defglsdisplay} and \cs{defglsdisplayfirst} take two arguments:
the first (which is optional) is the glossary's label\footnote{\texttt{main} for the main
(default) glossary, \cs{acronymtype} for the list of acronyms, or the
name supplied in the first mandatory argument to \cs{newglossary} for
additional glossaries.} and the second is how the term should be
displayed when it is invoked using commands \cs{gls}, 
\cs{glspl}, their uppercase variants and \cs{glsdisp}. This is similar to the way \cs{glsdisplayfirst} was
redefined above.

For example, suppose you have created a new glossary called
\texttt{notation} and you want to change the way the entry is
displayed on \firstuse\ so that it includes the symbol, you can do:
\begin{verbatim}
\defglsdisplayfirst[notation]{#1#4 (denoted #3)}
\end{verbatim}
Now suppose you have defined an entry as follows:
\begin{verbatim}
\newglossaryentry{set}{type=notation,
  name=set,
  description={A collection of objects},
  symbol={$S$}
}
\end{verbatim}
The \glslink{firstuse}{first time} you reference this entry it will be displayed as:
\qt{set (denoted $S$)} (assuming \ics{gls} was used).

Remember that if you use the \gloskey{symbol} key, you need to use a
glossary style that displays the symbol, as many of the styles
ignore it. In addition, if you want either the description or symbol
to appear in the \gls{linktext}, you will have to disable the
\glsation{sanitize} of these keys and protect fragile
commands.

\section{Enabling and disabling hyperlinks to glossary entries}
\label{sec:disablehyperlinks}

If you load the \sty{hyperref} or \sty{html} packages prior to
loading the \styfmt{glossaries} package, commands such as \ics{glslink}
and \ics{gls}, described above, will automatically have hyperlinks
to the relevant glossary entry, unless the \gloskey[glslink]{hyper}
option has been set to \texttt{false}. You can disable or enable links using:
\begin{definition}[\DescribeMacro{\glsdisablehyper}]
\cs{glsdisablehyper}
\end{definition}
and
\begin{definition}[\DescribeMacro{\glsenablehyper}]
\cs{glsenablehyper}
\end{definition}
respectively. The effect can be localised by placing the commands
within a group. Note that you should only use \cs{glsenablehyper}
if the commands \ics{hyperlink} and \ics{hypertarget} have been
defined (for example, by the \sty{hyperref} package).

You can disable just the \gls{firstuse} links using the package
option \pkgopt[false]{hyperfirst}. Note that this option only
affects commands that recognise the \firstuseflag, for example
\ics{gls}, \ics{glspl} and \ics{glsdisp} but not \ics{glslink}.


\chapter{Adding an Entry to the Glossary Without Generating Text}
\label{sec:glsadd}

It is possible to add a line in the glossary file without
generating any text at that point in the document using:
\begin{definition}[\DescribeMacro{\glsadd}]
\cs{glsadd}\oarg{options}\marg{label}
\end{definition}
This is similar to \ics{glslink}, only it doesn't produce
any text (so therefore, there is no \gloskey[glslink]{hyper} key
available in \meta{options} but all the other options that can
be used with \ics{glslink} can be passed to \cs{glsadd}).
For example, to add a page range to the glossary number list for the
entry whose label is given by \texttt{set}:
\begin{verbatim}
\glsadd[format=(]{set}
Lots of text about sets spanning many pages.
\glsadd[format=)]{set}
\end{verbatim}

To add all entries that have been defined, use:
\begin{definition}[\DescribeMacro{\glsaddall}]
\cs{glsaddall}\oarg{options}
\end{definition}
The optional argument is the same as for \cs{glsadd}, except
there is also a key \gloskey[glsaddall]{types} which can be
used to specify which glossaries to use. This should be a
comma separated list. For example, if you only want to add
all the entries belonging to the list of acronyms (specified by
the glossary type \ics{acronymtype}) and a list of
notation (specified by the glossary type \texttt{notation}) then you can
do:
\begin{verbatim}
\glsaddall[types={\acronymtype,notation}]
\end{verbatim}

\begin{important}
Note that \cs{glsadd} and \cs{glsaddall} add the current location to
the number list. In the case of \cs{glsaddall}, all entries in the
glossary will have the same location in the number list. If you want
to use \cs{glsaddall}, it's best to suppress the number list with
the \pkgopt{nonumberlist} package option. (See
sections~\ref{sec:pkgopts-printglos} and~\ref{sec:numberlists}.)
\end{important}

The sample file \samplefile{-dual} makes use of \cs{glsadd} to
allow for an entry that should appear both in the main glossary and
in the list of acronyms. This example sets up the list of acronyms
using the \pkgopt{acronym} package option:
\begin{verbatim}
\usepackage[acronym]{glossaries}
\end{verbatim}
A new command is then defined to make it easier to define dual
entries:
\begin{verbatim}
\newcommand*{\newdualentry}[5][]{%
  \newglossaryentry{main-#2}{name={#4},%
  text={#3\protect\glsadd{#2}},%
  description={#5},%
  #1
  }%
  \newacronym{#2}{#3\protect\glsadd{main-#2}}{#4}
}
\end{verbatim}
This has the following syntax:
\begin{definition}
\ics{newdualentry}\oarg{options}\marg{label}\marg{abbrv}\marg{long}\marg{description}
\end{definition}
You can then define a new dual entry:
\begin{verbatim}
\newdualentry{svm}% label
  {SVM}% abbreviation
  {support vector machine}% long form
  {Statistical pattern recognition technique}% description
\end{verbatim}
Now you can reference the acronym with \verb|\gls{svm}| or you can
reference the entry in the main glossary with \verb|\gls{main-svm}|.

\chapter{Cross-Referencing Entries}
\label{sec:crossref}

There are several ways of cross-referencing entries in the 
glossary: 

\begin{enumerate}
\item You can use commands such as \ics{gls} in the
entries description. For example:
\begin{verbatim}
\newglossaryentry{apple}{name=apple,
description={firm, round fruit. See also \gls{pear}}}
\end{verbatim}
Note that with this method, if you don't use the cross-referenced term 
in the main part of the document, you will need two runs of
\gls{makeglossaries}:
\begin{verbatim}
latex filename
makeglossaries filename
latex filename
makeglossaries filename
latex filename
\end{verbatim}

\begin{important}
If you switch off the description
\glsation{sanitize}, you must protect fragile
commands:\footnote{As from v3.01, \ics{gls} is no longer fragile.}
\begin{verbatim}
\newglossaryentry{apple}{name=apple,
description={firm, round fruit. See also 
\protect\gls{pear}}}
\end{verbatim}
\end{important}

\item As described in \sectionref{sec:newglosentry}, you can use the
\gloskey{see} key when you define the entry. For example:
\begin{verbatim}
\newglossaryentry{MaclaurinSeries}{name={Maclaurin 
series},
description={Series expansion},
see={TaylorsTheorem}}
\end{verbatim}
Note that in this case, the entry with the \gloskey{see} key will
automatically be added to the glossary, but the cross-referenced
entry won't. You therefore need to ensure that you use the
cross-referenced term with the commands described in
\sectionref{sec:glslink} or \sectionref{sec:glsadd}.

The \qt{see} tag is produce using \ics{seename}, but can be
overridden in specific instances using square brackets
at the start of the \gloskey{see} value. For example:
\begin{verbatim}
\newglossaryentry{MaclaurinSeries}{name={Maclaurin 
series},
description={Series expansion},
see=[see also]{TaylorsTheorem}}
\end{verbatim}

\item After you have defined the entry, use
\begin{definition}[\DescribeMacro{\glssee}]
\cs{glssee}\oarg{tag}\marg{label}\marg{xr label list}
\end{definition}
where \meta{xr label list} is a comma-separated list of entry 
labels to be cross-referenced, \meta{label} is the label of the
entry doing the cross-referencing and \meta{tag} is the \qt{see} tag.
(The default value of \meta{tag} is \ics{seename}.)
For example:
\begin{verbatim}
\glssee[see also]{series}{FourierSeries,TaylorsTheorem}
\end{verbatim}
Note that this automatically adds the entry given by \meta{label} to
the glossary but doesn't add the cross-referenced entries (specified
by \meta{xr label list}) to the glossary.
\end{enumerate}

In both cases~2 and 3 above, the cross-referenced information
appears in the \gls{numberlist}, whereas in case~1, the
cross-referenced information appears in the description. (See the
\samplefile{-crossref} example file that comes with this package.)
This means that in cases~2 and~3, the cross-referencing information
won't appear if you have suppressed the \gls*{numberlist}. In this
case, you will need to activate the \gls*{numberlist} for the given
entries using \gloskey{nonumberlist=false}. Alternatively, if you
just use the \gloskey{see} key instead of \ics{glssee}, you can
automatically activate the \gls*{numberlist} using the
\pkgopt{seeautonumberlist} package option.

\section{Customising Cross-reference Text}
\label{sec:customxr}

When you use either the \gloskey{see} key or the command
\cs{glssee}, the cross-referencing information will be typeset in the 
glossary according to:
\begin{definition}[\DescribeMacro{\glsseeformat}]
\cs{glsseeformat}\oarg{tag}\marg{label-list}\marg{location}
\end{definition}
The default definition of \cs{glsseeformat} is:
\begin{display}
\cs{emph}\marg{tag} \cs{glsseelist}\marg{label-list}
\end{display}
Note that the location is always ignored.\footnote{\gls{makeindex}
will always assign a location number, even if it's not needed, so it
needs to be discarded.} For example, if you want the tag to appear
in bold, you can do:\footnote{If you redefine \cs{glsseeformat},
keep the default value of the optional argument as \ics{seename} as
both \gloskey{see} and \cs{glssee} explicitly write
\texttt[\cs{seename}\texttt] in the output file if no optional
argument is given.}
\begin{verbatim}
\renewcommand*{\glsseeformat}[3][\seename]{\textbf{#1} 
 \glsseelist{#2}}
\end{verbatim}


The list of labels is dealt with by \cs{glsseelist}, which iterates
through the list and typesets each entry in the label. The entries
are separated by
\begin{definition}[\DescribeMacro{\glsseesep}]
\cs{glsseesep}
\end{definition}
or (for the last pair)
\begin{definition}[\DescribeMacro{\glsseelastsep}]
\cs{glsseelastsep}
\end{definition}
These default to ``",\space"'' and
``\cs{space}\ics{andname}\cs{space}'' respectively. The list entry text
is displayed using:
\begin{definition}[\DescribeMacro{\glsseeitemformat}]
\cs{glsseeitemformat}\marg{label}
\end{definition}
This defaults to \ics{glsentrytext}\marg{label}.\footnote{In
versions before 3.0, \cs{glsentryname} was used, but this can cause
problems when the \gloskey{name} key is \glsd{sanitize}.} For example, to
make the cross-referenced list use small caps:
\begin{verbatim}
\renewcommand{\glsseeitemformat}[1]{%
  \textsc{\glsentrytext{#1}}}
\end{verbatim}

\begin{important}
You can use \ics{glsseeformat} and \ics{glsseelist} in the main body
of the text, but they won't automatically add the cross-referenced
entries to the glossary. If you want them added with that location,
you can do:
\begin{verbatim}
Some information (see also 
\glsseelist{FourierSeries,TaylorsTheorem}%
\glsadd{FourierSeries}\glsadd{TaylorsTheorem}).
\end{verbatim}
\end{important}

\chapter{Using Glossary Terms Without Links}
\label{sec:glsnolink}

The commands described in this section display entry details without
adding any information to the glossary. They don't use
\ics{glstextformat}, they don't have any optional arguments, they
don't affect the \firstuseflag\ and, apart from \ics{glshyperlink},
they don't produce hyperlinks.
\begin{definition}[\DescribeMacro{\glsentryname}]
\cs{glsentryname}\marg{label}
\end{definition}
\begin{definition}[\DescribeMacro{\Glsentryname}]
\cs{Glsentryname}\marg{label}
\end{definition}
These commands display the name of the glossary entry given by
\meta{label}, as specified by the \gloskey{name} key.
\cs{Glsentryname} makes the first letter uppercase.
\begin{definition}[\DescribeMacro{\glsentrytext}]
\cs{glsentrytext}\marg{label}
\end{definition}
\begin{definition}[\DescribeMacro{\Glsentrytext}]
\cs{Glsentrytext}\marg{label}
\end{definition}
These commands display the subsequent use text for the glossary
entry given by \meta{label}, as specified by the \gloskey{text} key.
\cs{Glsentrytext} makes the first letter uppercase.
\begin{definition}[\DescribeMacro{\glsentryplural}]
\cs{glsentryplural}\marg{label}
\end{definition}
\begin{definition}[\DescribeMacro{\Glsentryplural}]
\cs{Glsentryplural}\marg{label}
\end{definition}
These commands display the subsequent use plural text for the
glossary entry given by \meta{label}, as specified by the
\gloskey{plural} key. \cs{Glsentryplural} makes the first letter
uppercase.
\begin{definition}[\DescribeMacro{\glsentryfirst}]
\cs{glsentryfirst}\marg{label}
\end{definition}
\begin{definition}[\DescribeMacro{\Glsentryfirst}]
\cs{Glsentryfirst}\marg{label}
\end{definition}
These commands display the \firstusetext\ for the glossary entry
given by \meta{label}, as specified by the \gloskey{first} key.
\cs{Glsentryfirst} makes the first letter uppercase.
\begin{definition}[\DescribeMacro{\glsentryfirstplural}]
\cs{glsentryfirstplural}\marg{label}
\end{definition}
\begin{definition}[\DescribeMacro{\Glsentryfirstplural}]
\cs{Glsentryfirstplural}\marg{label}
\end{definition}
These commands display the plural form of the \firstusetext\ for the
glossary entry given by \meta{label}, as specified by the
\gloskey{firstplural} key. \cs{Glsentryfirstplural} makes the first
letter uppercase.
\begin{definition}[\DescribeMacro{\glsentrydesc}]
\cs{glsentrydesc}\marg{label}
\end{definition}
\begin{definition}[\DescribeMacro{\Glsentrydesc}]
\cs{Glsentrydesc}\marg{label}
\end{definition}
These commands display the description for the glossary entry given
by \meta{label}. \cs{Glsentrydesc} makes the first letter uppercase.
\begin{definition}[\DescribeMacro{\glsentrydescplural}]
\cs{glsentrydescplural}\marg{label}
\end{definition}
\begin{definition}[\DescribeMacro{\Glsentrydescplural}]
\cs{Glsentrydescplural}\marg{label}
\end{definition}
These commands display the plural description for the glossary entry
given by \meta{label}. \cs{Glsentrydescplural} makes the first
letter uppercase.
\begin{definition}[\DescribeMacro{\glsentrysymbol}]
\cs{glsentrysymbol}\marg{label}
\end{definition}
\begin{definition}[\DescribeMacro{\Glsentrysymbol}]
\cs{Glsentrysymbol}\marg{label}
\end{definition}
These commands display the symbol for the glossary entry given by
\meta{label}. \cs{Glsentrysymbol} makes the first letter uppercase.
\begin{definition}[\DescribeMacro{\glsentrysymbolplural}]
\cs{glsentrysymbolplural}\marg{label}
\end{definition}
\begin{definition}[\DescribeMacro{\Glsentrysymbolplural}]
\cs{Glsentrysymbolplural}\marg{label}
\end{definition}
These commands display the plural symbol for the glossary entry
given by \meta{label}. \cs{Glsentrysymbolplural} makes the first
letter uppercase.
\begin{definition}[\DescribeMacro{\glsentryuseri}]
\cs{glsentryuseri}\marg{label}
\end{definition}
\begin{definition}[\DescribeMacro{\Glsentryuseri}]
\cs{Glsentryuseri}\marg{label}
\end{definition}
\begin{definition}[\DescribeMacro{\glsentryuserii}]
\cs{glsentryuserii}\marg{label}
\end{definition}
\begin{definition}[\DescribeMacro{\Glsentryuserii}]
\cs{Glsentryuserii}\marg{label}
\end{definition}
\begin{definition}[\DescribeMacro{\glsentryuseriii}]
\cs{glsentryuseriii}\marg{label}
\end{definition}
\begin{definition}[\DescribeMacro{\Glsentryuseriii}]
\cs{Glsentryuseriii}\marg{label}
\end{definition}
\begin{definition}[\DescribeMacro{\glsentryuseriv}]
\cs{glsentryuseriv}\marg{label}
\end{definition}
\begin{definition}[\DescribeMacro{\Glsentryuseriv}]
\cs{Glsentryuseriv}\marg{label}
\end{definition}
\begin{definition}[\DescribeMacro{\glsentryuserv}]
\cs{glsentryuserv}\marg{label}
\end{definition}
\begin{definition}[\DescribeMacro{\Glsentryuserv}]
\cs{Glsentryuserv}\marg{label}
\end{definition}
\begin{definition}[\DescribeMacro{\glsentryuservi}]
\cs{glsentryuservi}\marg{label}
\end{definition}
\begin{definition}[\DescribeMacro{\Glsentryuservi}]
\cs{Glsentryuservi}\marg{label}
\end{definition}
These commands display the value of the user keys for the glossary
entry given by \meta{label}.
\begin{definition}[\DescribeMacro{\glshyperlink}]
\cs{glshyperlink}\oarg{link text}\marg{label}
\end{definition}
This command provides a hyperlink to the glossary entry given by
\meta{label} \textbf{but does not add any information to the
glossary file}. The link text is given by
\ics{glsentrytext}\marg{label} by default\footnote{versions before
3.0 used \ics{glsentryname} as the default, but this can cause
problems when \gloskey{name} has been \glsd{sanitize}.}, but can be
overridden using the optional argument.
\begin{important}
If you use \cs{glshyperlink}, you need to ensure that the relevant
entry has been added to the glossary using any of the commands
described in \sectionref{sec:glslink} or \sectionref{sec:glsadd}
otherwise you will end up with an undefined link.
\end{important}

The next two commands are only available with the
\pkgopt{savenumberlist} package option:
\begin{definition}[\DescribeMacro{\glsentrynumberlist}]
\cs{glsentrynumberlist}\marg{label}
\end{definition}
\begin{definition}[\DescribeMacro{\glsdisplaynumberlist}]
\cs{glsdisplaynumberlist}\marg{label}
\end{definition}
Both display the \gls{numberlist} for the entry given by
\meta{label} and require a run of \gls{makeglossaries} (or
\gls{xindy} or \gls{makeindex}) followed by one or two runs of
\LaTeX. The first command, \cs{glsentrynumberlist}, simply displays
the number list as is. The second command,
\linebreak\cs{glsdisplaynumberlist}, formats the list using:
\begin{definition}[\DescribeMacro{\glsnumlistsep}]
\cs{glsnumlistsep}
\end{definition}
as the separator between all but the last two elements and
\begin{definition}[\DescribeMacro{\glsnumlistlastsep}]
\cs{glsnumlistlastsep}
\end{definition}
between the final two elements. The defaults are
\verb*|, | and \verb*| \& |, respectively.

\begin{important}
\cs{glsdisplaynumberlist} is fairly experimental. It only works when
the default counter format is used (that is, when the
\gloskey[glslink]{format} key is set to \texttt{glsnumberformat}).
This command also doesn't work with \sty{hyperref}. If you try using
it with that package, \cs{glsentrynumberlist} will be used instead.
\end{important}

For further information see \ifpdf section~\ref*{sec:code:glsnolink}
\fi \qt{Displaying entry details without adding information to the
glossary} in the documented code (\texttt{glossaries.pdf}).

\chapter{Displaying a glossary}
\label{sec:printglossary}

The command
\begin{definition}[\DescribeMacro{\printglossaries}]
\cs{printglossaries}
\end{definition}
will display all the glossaries in the order in which they were
defined. Note that no glossaries will appear until you have either
used the Perl script \gls{makeglossaries} or have directly used
\gls{makeindex} or \gls{xindy} (as described in
\sectionref{sec:makeglossaries}). If the glossary 
still does not appear after you re-\LaTeX\ your document, check the
\gls*{makeindex}/\gls*{xindy} log files to see if there is a problem.
Remember that you also need to use the command \ics{makeglossaries}
in the preamble to enable the glossaries.

An individual glossary can be displayed using:
\begin{definition}[\DescribeMacro{\printglossary}]
\cs{printglossary}\oarg{options}
\end{definition}
where \meta{options} is a \meta{key}=\meta{value} list of options. 
The following keys are available:
\begin{description}
\item[{\gloskey[printglossary]{type}}] The value of this key
specifies which glossary to print. If omitted, the default
glossary is assumed. For example, to print the list of acronyms:
\begin{verbatim}
\printglossary[type=\acronymtype]
\end{verbatim}

\item[{\gloskey[printglossary]{title}}] This is the glossary's
title (overriding the title specified when the glossary was 
defined).

\item[{\gloskey[printglossary]{toctitle}}] This is the title
to use for the table of contents (if the \pkgopt{toc} package
option has been used). It may also be used for the page header,
depending on the page style. If omitted, the value of 
\gloskey[printglossary]{title} is used.

\item[{\gloskey[printglossary]{style}}] This specifies which
glossary style to use for this glossary, overriding the effect
of the \pkgopt{style} package option or \ics{glossarystyle}.

\item[{\gloskey[printglossary]{numberedsection}}] This specifies
whether to use a numbered section for this glossary, overriding
the effect of the \pkgopt{numberedsection} package option. This
key has the same syntax as the \pkgopt{numberedsection} package
option, described in \sectionref{sec:pkgopts-sec}.

\item[{\gloskey[printglossary]{nonumberlist}}] This is a boolean
key. If true (\verb|nonumberlist=true|) the numberlist is suppressed
for this glossary. If false \linebreak(\verb|nonumberlist=false|) the
numberlist is displayed for this glossary. If no value is supplied,
true is assumed.

\end{description}

By default, the glossary is started either by \ics{chapter*} or by
\ics{section*}, depending on whether or not \ics{chapter} is defined.
This can be overridden by the \pkgopt{section} package option or the
\linebreak\ics{setglossarysection} command. Numbered sectional units can be
obtained using the \pkgopt{numberedsection} package option. Each
glossary sets the page header via the command \ics{glossarymark}. If
this mechanism is unsuitable for your chosen class file or page
style package, you will need to redefine \ics{glossarymark}. Further
information about these options and commands is given in
\sectionref{sec:pkgopts-sec}.

Information can be added to the start of the glossary (after the
title and before the main body of the glossary) by redefining
\begin{definition}[\DescribeMacro{\glossarypreamble}]
\cs{glossarypreamble}
\end{definition}
For example:
\begin{verbatim}
\renewcommand{\glossarypreamble}{Numbers in italic 
indicate primary definitions.}
\end{verbatim}
This needs to be done before the glossary is displayed using
\linebreak\cs{printglossaries} or \cs{printglossary}. Note that if you
want a different preamble for each glossary, you will need to
use a separate \cs{printglossary} for each glossary and change
the definition of \cs{glossarypreamble} between each glossary.
For example:
\begin{verbatim}
\renewcommand{\glossarypreamble}{Numbers in italic 
indicate primary definitions.}
\printglossary
\renewcommand{\glossarypreamble}{}
\printglossary[type=acronym]
\end{verbatim}
Alternatively, you can do something like:
\begin{verbatim}
\renewcommand{\glossarypreamble}{Numbers in italic 
indicate primary definitions.\gdef\glossarypreamble{}}
\printglossaries
\end{verbatim}
which will print the preamble text for the first glossary and 
change the preamble to do nothing for subsequent glossaries.
(Note that \cs{gdef} is required as the glossary is placed within
a group.)

There is an analogous command called
\begin{definition}[\DescribeMacro{\glossarypostamble}]
\cs{glossarypostamble}
\end{definition}
which is placed at the end of each glossary.

For example: suppose you are using the \glostyle{superheaderborder}
style\footnote{you can't use the \glostyle{longheaderborder} style
for this example as you can't use the \env{longtable} environment in
two column mode.}, and you want the glossary to be in two columns,
but after the glossary you want to switch back to one column mode,
you could do:
\begin{verbatim}
\renewcommand*{\glossarysection}[2][]{%
  \twocolumn[{\chapter*{#2}}]%
  \setlength\glsdescwidth{0.6\linewidth}%
  \glossarymark{\glossarytoctitle}%
}

\renewcommand*{\glossarypostamble}{\onecolumn}
\end{verbatim}

Within each glossary, each entry name is formatted according to
\begin{definition}[\DescribeMacro{\glsnamefont}]
\cs{glsnamefont}\marg{name}
\end{definition}
which takes one argument: the entry name. This command is always
used regardless of the glossary style. By default, \cs{glsnamefont}
simply displays its argument in whatever the surrounding font
happens to be. This means that in the list-like glossary styles
(defined in the \sty{glossary-list} style file) the name will appear
in bold, since the name is placed in the optional argument of
\cs{item}, whereas in the tabular styles (defined in the
\sty{glossary-long} and \sty{glossary-super} style files) the name
will appear in the normal font. The hierarchical glossary styles
(defined in the \sty{glossary-tree} style file) also set the name in
bold.

For example, suppose you want all the entry names to appear in 
medium weight small caps, then you can do:
\begin{verbatim}
\renewcommand{\glsnamefont}[1]{\textsc{\mdseries #1}}
\end{verbatim}


\chapter{Xindy}
\label{sec:xindy}

If you want to use \gls{xindy} to sort the glossary, you
must use the package option \pkgopt{xindy}:
\begin{verbatim}
\usepackage[xindy]{glossaries}
\end{verbatim}
This ensures that the glossary information is written in 
\gls*{xindy} syntax.

\sectionref{sec:makeglossaries} covers how to use the external
\gls{indexingapp}. This section covers the commands provided
by the \styfmt{glossaries} package that allow you to adjust the
\gls{xindy} style file (\filetype{.xdy}) and parameters.

To assist writing information to the \gls{xindy} style
file, the \styfmt{glossaries} package provides the following
commands:
\begin{definition}[\DescribeMacro{\glsopenbrace}]
\cs{glsopenbrace}
\end{definition}
\begin{definition}[\DescribeMacro{\glsclosebrace}]
\cs{glsclosebrace}
\end{definition}
which produce an open and closing brace. (This is needed because
\verb|\{| and \verb|\}| don't expand to a simple brace character
when written to a file.)

In addition, if you are using a package that makes the double
quote character active (e.g. \sty{ngerman}) you can use:
\begin{definition}[\DescribeMacro{\glsquote}]
\cs{glsquote}\marg{text}
\end{definition}
which will produce \verb|"|\meta{text}\verb|"|. Alternatively,
you can use \verb|\string"| to write the double-quote character. 
This document assumes that the double quote character has not been
made active, so the examples just use \verb|"| for clarity.

If you want greater control over the \gls{xindy} style file than is
available through the \LaTeX\ commands provided by the
\styfmt{glossaries} package, you will need to edit the \gls*{xindy}
style file. In which case, you must use \ics{noist} to prevent the
style file from being overwritten by the \styfmt{glossaries}
package. For additional information about \gls*{xindy}, read the
\gls*{xindy} documentation.

\section{Language and Encodings}
\label{sec:langenc}

When you use \gls{xindy}, you need to specify the language
and encoding used (unless you have written your own custom
\gls*{xindy} style file that defines the relevant alphabet
and sort rules). If you use \gls{makeglossaries},
this information is obtained from the document's auxiliary 
(\filetype{.aux}) file.  The \styfmt{glossaries} package attempts to 
find the root language, but in the event that it gets it wrong or if 
\gls*{xindy} doesn't support that language, then you can
specify the language using:
\begin{definition}[\DescribeMacro{\GlsSetXdyLanguage}]
\cs{GlsSetXdyLanguage}\oarg{glossary type}\marg{language}
\end{definition}
where \meta{language} is the name of the language. The
optional argument can be used if you have multiple glossaries
in different languages. If \meta{glossary type} is omitted, it
will be applied to all glossaries, otherwise the language
setting will only be applied to the glossary given by
\meta{glossary type}.

If the \sty{inputenc} package is used, the encoding will be
obtained from the value of \ics{inputencodingname}. 
Alternatively, you can specify the encoding using:
\begin{definition}[\DescribeMacro{\GlsSetXdyCodePage}]
\cs{GlsSetXdyCodePage}\marg{code}
\end{definition}
where \meta{code} is the name of the encoding. For example:
\begin{verbatim}
\GlsSetXdyCodePage{utf8}
\end{verbatim}

Note that you can also specify the language and encoding using
the package option 
\verb|xindy={language=|\meta{lang}\verb|,codepage=|\meta{code}\verb|}|.
For example:
\begin{verbatim}
\usepackage[xindy={language=english,codepage=utf8}]{glossaries}
\end{verbatim}

If you write your own custom \gls{xindy} style file that 
includes the language settings, you need to set the language
to nothing:
\begin{verbatim}
\GlsSetXdyLanguage{}
\end{verbatim}
(and remember to use \ics{noist} to prevent the style file from
being overwritten).

\begin{important}
The commands \cs{GlsSetXdyLanguage} and \cs{GlsSetXdyCodePage}
have no effect if you don't use \gls{makeglossaries}. If
you call \gls{xindy} without \gls*{makeglossaries} you
need to remember to set the language and encoding using the
\texttt{-L} and \texttt{-C} switches.
\end{important}


\section{Locations and Number lists}
\label{sec:xindyloc}

If you use \pkgopt{xindy}, the \styfmt{glossaries} package needs to
know which counters you will be using in the \gls{numberlist} in order to correctly format the \gls{xindy} style
file. Counters specified using the \pkgopt{counter} package option
or the \meta{counter} option of \ics{newglossary} are
automatically taken care of, but if you plan to use a different
counter in the \gloskey[glslink]{counter} key for commands like 
\ics{glslink}, then you need to identify these counters \emph{before} 
\ics{makeglossaries} using:
\begin{definition}[\DescribeMacro{\GlsAddXdyCounters}]
\cs{GlsAddXdyCounters}\marg{counter list}
\end{definition}
where \meta{counter list} is a comma-separated list of counter names.

The most likely attributes used in the \gloskey[glslink]{format} 
key (\locfmt{textrm}, \locfmt{hyperrm} etc) are automatically added 
to the \gls{xindy} style file, but if you want to use another 
attribute, you need to add it using:
\begin{definition}[\DescribeMacro{\GlsAddXdyAttribute}]
\cs{GlsAddXdyAttribute}\marg{name}
\end{definition}
where \meta{name} is the name of the attribute, as used in
the \gloskey[glslink]{format} key. For example, suppose I want a
bold, italic, hyperlinked location. I first need to define a
command that will do this:
\begin{verbatim}
\newcommand*{\hyperbfit}[1]{\textit{\hyperbf{#1}}}
\end{verbatim}
but with \gls{xindy}, I also need to add this as an allowed
attribute:
\begin{verbatim}
\GlsAddXdyAttribute{hyperbfit}
\end{verbatim}

\begin{important}
Note that \cs{GlsAddXdyAttribute} has no effect if \ics{noist} is
used or if \ics{makeglossaries} is omitted.
\cs{GlsAddXdyAttribute} must be used before \ics{makeglossaries}.
Additionally, \cs{GlsAddXdyCounters} must come before
\cs{GlsAddXdyAttribute}.
\end{important}

If the location numbers don't get expanded to a simple Arabic or 
Roman number or a letter from a, \ldots, z or A, \ldots, Z, then
you need to add a location style in the appropriate format using
\begin{definition}[\DescribeMacro{\GlsAddXdyLocation}]
\cs{GlsAddXdyLocation}\oarg{prefix-location}\marg{name}\marg{definition}
\end{definition}
where \meta{name} is the name of the format and \meta{definition} is
the \gls{xindy} definition. The optional argument \meta{prefix-location}
is needed if \linebreak\cs{theH}\meta{counter} either isn't defined or is
different from \cs{the}\meta{counter}.

\begin{important}
Note that \cs{GlsAddXdyLocation} has no effect if \ics{noist} is
used or if \ics{makeglossaries} is omitted.
\cs{GlsAddXdyLocation} must be used before \ics{makeglossaries}.
\end{important}

For example, suppose I decide to use a somewhat eccentric numbering
system for sections that redefines \cs{thesection} as follows:
\begin{verbatim}
\renewcommand*{\thesection}{[\thechapter]\arabic{section}}
\end{verbatim}
If I haven't done "counter=section" in the package
option, I need to specify that the counter will be used as a
location number:
\begin{verbatim}
\GlsAddXdyCounters{section}
\end{verbatim}
Next I need to add the location style (\cs{thechapter} is assumed to
be the standard \verb"\arabic{chapter}"):
\begin{verbatim}
\GlsAddXdyLocation{section}{:sep "[" "arabic-numbers" :sep "]"
  "arabic-numbers"
}
\end{verbatim}
Note that if I have further decided to use the \sty{hyperref}
package and want to redefine \cs{theHsection} as:
\begin{verbatim}
\renewcommand*{\theHsection}{\thepart.\thesection}
\renewcommand*{\thepart}{\Roman{part}}
\end{verbatim}
then I need to modify the \cs{GlsAddXdyLocation} code above to:
\begin{verbatim}
\GlsAddXdyLocation["roman-numbers-uppercase"]{section}{:sep "[" 
  "arabic-numbers" :sep "]" "arabic-numbers"
}
\end{verbatim}
Since \ics{Roman} will result in an empty string if the counter is
zero, it's a good idea to add an extra location to catch this:
\begin{verbatim}
\GlsAddXdyLocation{zero.section}{:sep "[" 
  "arabic-numbers" :sep "]" "arabic-numbers"
}
\end{verbatim}
The above example is illustrated in the accompanying sample file
\samplefile{xdy2}.

Another example: suppose you want the page numbers written as words
rather than digits and you use the \sty{fmtcount} package to
do this. You can redefine \ics{thepage} as follows:
\begin{verbatim}
\renewcommand*{\thepage}{\Numberstring{page}}
\end{verbatim}
This gets expanded to \verb|\protect \Numberstringnum |\marg{n}
where \meta{n} is the Arabic page number. This means that you need to
define a new location that has that form:
\begin{verbatim}
\GlsAddXdyLocation{Numberstring}{:sep "\string\protect\space
  \string\Numberstringnum\space\glsopenbrace"
  "arabic-numbers" :sep "\glsclosebrace"}
\end{verbatim}
Note that it's necessary to use \cs{space} to indicate that 
spaces also appear in the format, since, unlike \TeX,
\gls{xindy} doesn't ignore spaces after control sequences.

\cs{GlsAddXdyLocation}\marg{name}\marg{definition} will define commands
\begin{definition}
\cs{glsX}\meta{counter}"X"\meta{name}\marg{Hprefix}\marg{location}
\end{definition}
for each counter that has been identified either by the
\pkgopt{counter} package option, the \meta{counter} option for
\ics{newglossary} or in the argument of \ics{GlsAddXdyCounters}.

The first argument \meta{Hprefix} is only relevant when used with
the \sty{hyperref} package and indicates that \cs{the}\meta{Hcounter}
is given by \cs{Hprefix}"."\cs{the}\meta{counter}. The sample
file \samplefile{xdy}, which comes with the \styfmt{glossaries}
package, uses the default \ctr{page} counter for locations, and it
uses the default \ics{glsnumberformat} and a custom \cs{hyperbfit}
format. A new \gls{xindy} location called \texttt{Numberstring}, as
illustrated above, is defined to make the page numbers appear as
\qt{One}, \qt{Two}, etc. In order for the location numbers to
hyperlink to the relevant pages, I need to redefine the necessary
\cs{glsX}\meta{counter}"X"\meta{format} commands:
\begin{verbatim}
\renewcommand{\glsXpageXglsnumberformat}[2]{%
 \linkpagenumber#2%
}

\renewcommand{\glsXpageXhyperbfit}[2]{%
 \textbf{\em\linkpagenumber#2}%
}

\newcommand{\linkpagenumber}[3]{\hyperlink{page.#3}{#1#2{#3}}}
\end{verbatim}

In the \gls{numberlist}, the locations are sorted according to
type. The default ordering is: \texttt{roman-page-numbers} (e.g.\
i), \texttt{arabic-page-numbers} (e.g.\ 1),
\texttt{arabic-section-numbers} (e.g.\ 1.1 if the compositor is a
full stop or 1-1 if the compositor is a hyphen\footnote{see
\ics{setCompositor} described in \sectionref{sec:setup}}),
\texttt{alpha-page-numbers} (e.g.\ a), \texttt{Roman-page-numbers}
(e.g.\ I), \texttt{Alpha-page-numbers} (e.g.\ A),
\texttt{Appendix-page-numbers} (e.g.\ A.1 if the Alpha compositor is
a full stop or A-1 if the Alpha compositor is a hyphen\footnote{see
\ics{setAlphaCompositor} described in
\sectionref{sec:setup}}), user defined location names (as
specified by \ics{GlsAddXdyLocation} in the order in which they were
defined), \texttt{see} (cross-referenced entries). This ordering can
be changed using:

\DescribeMacro{\GlsSetXdyLocationClassOrder}
\begin{definition}
\cs{GlsSetXdyLocationClassOrder}\marg{location names}
\end{definition}
where each location name is delimited by double quote marks and
separated by white space. For example:
\begin{verbatim}
\GlsSetXdyLocationClassOrder{
  "arabic-page-numbers"
  "arabic-section-numbers"
  "roman-page-numbers"
  "Roman-page-numbers"
  "alpha-page-numbers"
  "Alpha-page-numbers"
  "Appendix-page-numbers"
  "see"
}
\end{verbatim}

\begin{important}\raggedright
Note that \cs{GlsSetXdyLocationClassOrder} has no effect if 
\ics{noist} is used or if \ics{makeglossaries} is omitted.
\cs{GlsSetXdyLocationClassOrder} must be used before 
\ics{makeglossaries}.
\end{important}

If a \gls{numberlist} consists of a sequence of consecutive 
numbers, the range will be concatenated. The 
number of consecutive locations that causes a range formation 
defaults to 2, but can be changed using:
\begin{definition}[\DescribeMacro{\GlsSetXdyMinRangeLength}]
\cs{GlsSetXdyMinRangeLength}\marg{n}
\end{definition}
For example:
\begin{verbatim}
\GlsSetXdyMinRangeLength{3}
\end{verbatim}
The argument may also be the keyword \texttt{none}, to indicate that
there should be no range formations. See the \gls{xindy}
manual for further details on range formations.

\begin{important}
Note that \cs{GlsSetXdyMinRangeLength} has no effect if \ics{noist}
is used or if \ics{makeglossaries} is omitted.
\cs{GlsSetXdyMinRangeLength} must be used before 
\ics{makeglossaries}.
\end{important}

See \sectionref{sec:numberlists} for further details.


\section{Glossary Groups}
\label{sec:groups}

The glossary is divided into groups according to the first
letter of the sort key. The \styfmt{glossaries} package also adds
a number group by default, unless you suppress it in the
\pkgopt{xindy} package option. For example:
\begin{verbatim}
\usepackage[xindy={glsnumbers=false}]{glossaries}
\end{verbatim}
Any entry that doesn't go in one of the letter groups or the
number group is placed in the default group.

If you have a number group, the default behaviour is to locate
it before the \qt{A} letter group. If you are not using a
Roman alphabet, you can change this using:

\DescribeMacro{\GlsSetXdyFirstLetterAfterDigits}
\begin{definition}
\cs{GlsSetXdyFirstLetterAfterDigits}\marg{letter}
\end{definition}

\begin{important}\raggedright
Note that \cs{GlsSetXdyFirstLetterAfterDigits} has no effect if 
\ics{noist} is used or if \ics{makeglossaries} is omitted.
\cs{GlsSetXdyFirstLetterAfterDigits} must be used before 
\ics{makeglossaries}.\par
\end{important}


\chapter{Defining New Glossaries}
\label{sec:newglossary}

A new glossary can be defined using:
\begin{definition}[\DescribeMacro{\newglossary}]
\cs{newglossary}\oarg{log-ext}\marg{name}\marg{in-ext}\marg{out-ext}\marg{title}\linebreak\oarg{counter}
\end{definition}
where \meta{name} is the label to assign to this glossary. The
arguments \meta{in-ext} and \meta{out-ext} specify the extensions to
give to the input and output files for that glossary, \meta{title}
is the default title for this new glossary and the final optional
argument \meta{counter} specifies which counter to use for the
associated \glspl{numberlist} (see also
\sectionref{sec:numberlists}). The first optional argument specifies
the extension for the \gls{makeindex} or \gls{xindy} transcript file
(this information is only used by \gls{makeglossaries} which picks
up the information from the auxiliary file).

Note that the main (default) glossary is automatically created as:
\begin{verbatim}
\newglossary{main}{gls}{glo}{\glossaryname}
\end{verbatim}
so it can be identified by the label \texttt{main} (unless the
\pkgopt{nomain} package option is used). Using the
\pkgopt{acronym} package option is equivalent to:
\begin{verbatim}
\newglossary[alg]{acronym}{acr}{acn}{\acronymname}
\end{verbatim}
so it can be identified by the label \texttt{acronym}. If you are
not sure whether the \pkgopt{acronym} option has been used, you
can identify the list of acronyms by the command 
\DescribeMacro{\acronymtype}\cs{acronymtype} which is set to
\texttt{acronym}, if the \pkgopt{acronym} option has been used,
otherwise it is set to \texttt{main}. Note that if you are using
the main glossary as your list of acronyms, you need to declare
it as a list of acronyms using the package option 
\pkgopt{acronymlists}.

\begin{important}
All glossaries must be defined before \ics{makeglossaries} to 
ensure that the relevant output files are opened.

See \sectionref{sec:fixednames} if you want to redefine \cs{glossaryname}, 
especially if you are using \sty{babel} or \sty{translator}.
\end{important}


\chapter{Acronyms}
\label{sec:acronyms}

You may have noticed in \sectionref{sec:newglosentry} that when you
specify a new entry, you can specify alternate text to use when the
term is \glsd{firstuse} in the document. This provides a
useful means to define acronyms. For convenience, the
\styfmt{glossaries} package defines the command:
\begin{definition}[\DescribeMacro{\newacronym}]
\cs{newacronym}\oarg{key-val
list}\marg{label}\marg{abbrv}\marg{long}
\end{definition}

This uses \ics{newglossaryentry} to create an entry with the given
label in the glossary given by \ics{acronymtype}. Amongst other
things, it sets up the \gloskey{first} and \gloskey{text} keys and,
depending on the acronym style, may also use \ics{defdisplayfirst}
and \ics{defdisplay} (see \sectionref{sec:glsdisplay}). 

The optional argument \marg{key-val list} allows you to specify keys
such as \gloskey{description} (when used with the
\pkgopt{description} package option, described below) or you can
override plural forms of \meta{abbrv} or \meta{long} using the
\gloskey{shortplural} or \gloskey{longplural} keys.
For example:
\begin{verbatim}
\newacronym[longplural={diagonal matrices}]{dm}{DM}{diagonal 
matrix}
\end{verbatim}
If the \firstuse\ uses the plural form, \verb|\glspl{dm}| will
display: diagonal matrices (DMs).


The following package options are available to change the acronym
style:

\begin{description}
\item[\pkgopt{description}]
With this package option, the \gloskey{description} key needs to be
set in the optional argument \meta{key-val list} of \cs{newacronym}.
(If this package option isn't used, the long form \meta{long} is put
in the \gloskey{description} key.)

\item[\pkgopt{footnote}]
With this package option, on \gls{firstuse} the long form \meta{long} is
placed in a footnote. By default the long form in the footnote will
link to the relevant entry in the glossary or list of acronyms. You
can prevent this link by doing:
\begin{verbatim}
\let\acrfootnote\acrnolinkfootnote
\end{verbatim}

\item[\pkgopt{smallcaps}]
With this package option, the short form \meta{abbrv} is typeset
using \ics{textsc}. (Recall that \cs{textsc} converts lower case
characters into small capitals and leaves upper case characters as
they are. Therefore, you need to have lower case characters in
\meta{abbrv} for this option to have an effect.)

\item[\pkgopt{smaller}]
This is similar to \pkgopt{smallcaps}, except that \ics{textsmaller}
is used instead of \ics{textsc}. Note that the \styfmt{glossaries}
package doesn't define \ics{textsmaller} nor does it load any
package that does so. If you use this option, you must make sure
\ics{textsmaller} is defined (for example by loading \sty{relsize}).

\item[\pkgopt{dua}]
This option will set both the \gloskey{first} and \gloskey{text}
keys to the long form \meta{long}.
\end{description}

If you want to define your own custom acronym style, see
\sectionref{sec:customacronym}.

\begin{important}
If you try using \ics{newglossaryentry} for entries in a designated
list of acronyms in combination with any of the above named package 
options you are likely to get unexpected results such as empty
brackets or empty footnotes.

If you don't intend to use \ics{newacronym} you should skip this section]
and not use the above package options.
\end{important}

As mentioned in \sectionref{sec:pkgopts-sec}, the command
\ics{acronymtype} is the name of the glossary in which the acronyms
should appear. If the \pkgopt{acronym} package option has been used,
this will be \texttt{acronym}, otherwise it will be \texttt{main}.
The acronyms can then be used in exactly the same way as any other
glossary entry. If you want more than one list of acronyms, you must
identify the others using the package options \pkgopt{acronymlists}.
This ensures that options such as \pkgopt{footnote} and
\pkgopt{smallcaps} work for the additional lists of acronyms.

\begin{important}\raggedright
Since \cs{newacronym} sets \verb|type=\acronymtype|,
if you want to load a file containing acronym definitions using
\ics{loadglsentries}\oarg{type}\marg{filename}, the optional argument
\meta{type} will not have an effect unless you explicitly set the
type as \verb|type=\glsdefaulttype| in the optional argument to
\ics{newacronym}. See \sectionref{sec:loadglsentries}.
\end{important}

Since \ics{newacronym} uses \ics{newglossaryentry}, you can use
commands like \ics{gls} and \ics{glsreset} as with any other
glossary entry.

For example, the following defines the acronym IDN:
\begin{verbatim}
\newacronym{idn}{IDN}{identification number}
\end{verbatim}
\verb|\gls{idn}| will produce \qt{identification number (IDN)} on
\firstuse\ and \qt{IDN} on subsequent uses. If you want to use the
\pkgopt{smallcaps} package option, you need to use lower case
characters for the shortened form:
\begin{verbatim}
\newacronym{idn}{idn}{identification number}
\end{verbatim}
Now \verb|\gls{idn}| will produce \qt{identification number
(\textsc{idn})} on \firstuse\ and \qt{\textsc{IDN}} on subsequent
uses.

If you use any of the package options \pkgopt{smallcaps},
\pkgopt{smaller}, \pkgopt{description} or \pkgopt{footnote}, the
short form \meta{abbrv} will be displayed in the document using:
\begin{definition}[\DescribeMacro{\acronymfont}]
\cs{acronymfont}\marg{text}
\end{definition}
and
\begin{definition}[\DescribeMacro{\firstacronymfont}]
\cs{firstacronymfont}\marg{text}
\end{definition}
where \cs{firstacronymfont} is applied on \firstuse\ and
\cs{acronymfont} is applied on subsequent use. Note that if you
don't use any of the above package options, changing the definition
of \cs{acronymfont} or \cs{firstacronymfont} will have no effect. In
this case, the recommended route is to use either the
\pkgopt{smaller} or the \pkgopt{smallcaps} package option and
redefine \cs{acronymfont} and \cs{firstacronymfont} as required.
(The \pkgopt{smallcaps} option sets the default plural suffix in an
upright font to cancel the effect of \ics{textsc}, whereas
\pkgopt{smaller} sets the suffix in the surrounding font.) For
example, if you want acronyms in a normal font on \gls{firstuse} and
emphasized on subsequent use, do:
\begin{verbatim}
\usepackage[smaller]{glossaries}
\renewcommand*{\firstacronymfont}[1]{#1}
\renewcommand*{\acronymfont}[1]{\emph{#1}}
\end{verbatim}
(Note that it is for this reason that the \sty{relsize} package is
not automatically loaded when selecting the \pkgopt{smaller} package
option.)

There are commands analogous to \ics{glstext} (described in
\sectionref{sec:glslink}) that allow you to access just the short
form, just the long form or the full form, without affecting the
\gls{firstuseflag}. (Note that the full form isn't necessarily the same
as the text produced on \firstuse.)

\begin{definition}[\DescribeMacro{\acrshort}]
\cs{acrshort}\oarg{options}\marg{label}\oarg{insert}
\end{definition}
This displays the short form for the entry given by \meta{label}.
The optional arguments are the same as those for \ics{glstext}.
There is also a starred version to suppress the hyperlink. There are
also analogous upper case variants:
\begin{definition}[\DescribeMacro{\Acrshort}]
\cs{Acrshort}\oarg{options}\marg{label}\oarg{insert}
\end{definition}
\begin{definition}[\DescribeMacro{\ACRshort}]
\cs{ACRshort}\oarg{options}\marg{label}\oarg{insert}
\end{definition}

\begin{definition}[\DescribeMacro{\acrlong}]
\cs{acrlong}\oarg{options}\marg{label}\oarg{insert}
\end{definition}
This displays the long form for the entry given by \meta{label}.
The optional arguments are the same as before.  There is also a starred 
version to suppress the hyperlink. There are also analogous upper case 
variants:
\begin{definition}[\DescribeMacro{\Acrlong}]
\cs{Acrlong}\oarg{options}\marg{label}\oarg{insert}
\end{definition}
\begin{definition}[\DescribeMacro{\ACRlong}]
\cs{ACRlong}\oarg{options}\marg{label}\oarg{insert}
\end{definition}

\begin{definition}[\DescribeMacro{\acrfull}]
\cs{acrfull}\oarg{options}\marg{label}\oarg{insert}
\end{definition}
This is equivalent to: 
\DescribeMacro{\acrfullformat}\cs{acrfullformat}"{"\ics{acrlong}"}{"\ics{acrshort}"}".
This defaults to \meta{long} (\ics{acronymfont}\marg{short})
regardless of the package options. There are also analogous upper
case variants:
\begin{definition}[\DescribeMacro{\Acrfull}]
\cs{Acrfull}\oarg{options}\marg{label}\oarg{insert}
\end{definition}
\begin{definition}[\DescribeMacro{\ACRfull}]
\cs{ACRfull}\oarg{options}\marg{label}\oarg{insert}
\end{definition}

If you find the above commands too cumbersome to write, you can use
the \pkgopt{shortcuts} package option to activate the shorter
command names listed in \tableref{tab:shortcuts}.

\begin{table}[htbp]
\caption[Synonyms provided by the package option shortcuts]{Synonyms provided by the package option \pkgopt{shortcuts}}
\label{tab:shortcuts}
\vskip\baselineskip
\centering
\begin{tabular}{ll}
\bfseries Shortcut Command & \bfseries Equivalent Command\\
\ics{acs} & \ics{acrshort}\\
\ics{Acs} & \ics{Acrshort}\\
\ics{acsp} & \ics{acrshortpl}\\
\ics{Acsp} & \ics{Acrshortpl}\\
\ics{acl} & \ics{acrlong}\\
\ics{Acl} & \ics{Acrlong}\\
\ics{aclp} & \ics{acrlongpl}\\
\ics{Aclp} & \ics{Acrlongpl}\\
\ics{acf} & \ics{acrfull}\\
\ics{Acf} & \ics{Acrfull}\\
\ics{acfp} & \ics{acrfullpl}\\
\ics{Acfp} & \ics{Acrfullpl}\\
\ics{ac} & \ics{gls}\\
\ics{Ac} & \ics{Gls}\\
\ics{acp} & \ics{glspl}\\
\ics{Acp} & \ics{Glspl}
\end{tabular}
\end{table}

It is also possible to access the long and short forms without
adding information to the glossary using commands analogous to
\ics{glsentrytext} (described in \sectionref{sec:glsnolink}).

The long form can be accessed using:
\begin{definition}[\DescribeMacro{\glsentrylong}]
\cs{glsentrylong}\marg{label}
\end{definition}
or, with the first letter converted to upper case:
\begin{definition}[\DescribeMacro{\Glsentrylong}]
\cs{Glsentrylong}\marg{label}
\end{definition}
Plural forms:
\begin{definition}[\DescribeMacro{\glsentrylongpl}]
\cs{glsentrylongpl}\marg{label}
\end{definition}
\begin{definition}[\DescribeMacro{\Glsentrylongpl}]
\cs{Glsentrylongpl}\marg{label}
\end{definition}

Similarly, to access the short form:
\begin{definition}[\DescribeMacro{\glsentryshort}]
\cs{glsentryshort}\marg{label}
\end{definition}
or, with the first letter converted to upper case:
\begin{definition}[\DescribeMacro{\Glsentryshort}]
\cs{Glsentryshort}\marg{label}
\end{definition}
Plural forms:
\begin{definition}[\DescribeMacro{\glsentryshortpl}]
\cs{glsentryshortpl}\marg{label}
\end{definition}
\begin{definition}[\DescribeMacro{\Glsentryshortpl}]
\cs{Glsentryshortpl}\marg{label}
\end{definition}

And the full form, \meta{long} (\meta{short}), can be obtained
using:
\begin{definition}[\DescribeMacro{\glsentryfull}]
\cs{glsentryfull}\marg{label}
\end{definition}
\begin{definition}[\DescribeMacro{\Glsentryfull}]
\cs{Glsentryfull}\marg{label}
\end{definition}
\begin{definition}[\DescribeMacro{\glsentryfullpl}]
\cs{glsentryfullpl}\marg{label}
\end{definition}
\begin{definition}[\DescribeMacro{\Glsentryfullpl}]
\cs{Glsentryfullpl}\marg{label}
\end{definition}

\section{Predefined Acronym Styles}
\label{sec:predefinedacrstyles}

Some of the acronym-related package options may be combined. Listed
below are the effects of different combinations. If you use an
invalid combination, you'll get an error message.

\begin{description}
\item[\pkgopt{description},\pkgopt{footnote}]\ifpdf\mbox{}\par\fi
When these two package options are used together, the \firstuse\ displays
the entry as:
\begin{display}
\cs{firstacronymfont}\marg{abbrv}\meta{insert}\cs{footnote}\marg{long} 
\end{display}
while subsequent use displays the entry as:
\begin{display}
\cs{acronymfont}\marg{abbrv}\meta{insert}
\end{display}
where \meta{insert} indicates the text supplied in the final
optional argument to \ics{gls}, \ics{glspl} or their uppercase 
variants.

\item[\pkgopt{dua}]\ifpdf\mbox{}\par\fi
The \pkgopt{dua} package option always makes \ics{gls} display the 
expanded form and so may not be used with \pkgopt{footnote}, 
\pkgopt{smaller} or \pkgopt{smallcaps}. Both \firstuse\ and subsequent 
use displays the entry in the form:
\begin{display}
\meta{long}\meta{insert}
\end{display}
You can, however, access the short form using \ics{acrshort} and its
variants.

\item[\pkgopt{description}]\ifpdf\mbox{}\par\fi
This package option displays the entry on \firstuse\ as:
\begin{display}
\meta{long}\meta{insert} (\ics{firstacronymfont}\marg{abbrv})
\end{display}
while subsequent use displays the entry as:
\begin{display}
\cs{acronymfont}\marg{abbrv}\meta{insert}
\end{display}

Note that with this option, you need to specify the description
using the \gloskey{description} key in the optional argument
of \ics{newacronym}.

\item[\pkgopt{footnote}]\ifpdf\mbox{}\par\fi
This package option displays the entry on \firstuse\ as:
\begin{display}
\cs{firstacronymfont}\marg{abbrv}\meta{insert}\cs{footnote}\marg{long}
\end{display}
while subsequent use displays the entry as:
\begin{display}
\cs{acronymfont}\marg{abbrv}\meta{insert}
\end{display}
Acronyms will be sorted according to the short form.

Note that on \firstuse, it is the long form in the footnote that
links to the relevant glossary entry (where hyperlinks are enabled),
whereas on subsequent use, the acronym links to the relevant
glossary entry. You can suppress the long form link by setting:
\begin{verbatim}
\let\acrfootnote\acrnolinkfootnote
\end{verbatim}

\item[\pkgopt{smallcaps}]\ifpdf\mbox{}\par\fi
If neither the \pkgopt{footnote} nor \pkgopt{description} options
have been set, this option displays the entry on \firstuse\ as:
\begin{display}
\meta{long}\meta{insert} (\ics{firstacronymfont}\marg{abbrv})
\end{display}
while subsequent use displays the entry as:
\begin{display}
\cs{acronymfont}\marg{abbrv}\meta{insert}
\end{display}
where \ics{acronymfont} is set to \verb|\textsc{#1}|.

\begin{important}
Note that since the acronym is displayed using \ics{textsc}, 
the short form, \meta{abbrv}, should be specified in lower case.
\ifpdf(Recall that "\textsc{abc}" produces \textsc{abc} whereas 
"\textsc{ABC}" produces \textsc{ABC}.)\fi
\end{important}

\item[\pkgopt{smaller}]\ifpdf\mbox{}\par\fi
If neither the \pkgopt{footnote} nor \pkgopt{description} options
have been set, this option displays the entry on \firstuse\ as:
\begin{display}
\meta{long}\meta{insert} (\ics{firstacronymfont}\marg{abbrv})
\end{display}
while subsequent use displays the entry as:
\begin{display}
\ics{acronymfont}\marg{abbrv}\meta{insert}
\end{display}
where \ics{acronymfont} is set to 
\verb|\textsmaller{#1}|.\footnote{not that this was change from 
using \ics{smaller} to \ics{textsmaller} as declarations
cause a problem for \ics{makefirstuc}.}
The entries will be sorted according to the short form.

\begin{important}
Remember to load a package that defines \ics{textsmaller} (such as
\sty{relsize}) if you want to use this option, unless you want
to redefine \ics{acronymfont} to use some other formatting command.
\end{important}

\item[None of the above]\ifpdf \mbox{}\par\fi
If none of the package options \pkgopt{smallcaps}, \pkgopt{smaller},
\pkgopt{footnote}, \pkgopt{dua} or \pkgopt{description} are used, 
then on \firstuse\ the entry is displayed as:
\begin{display}
\meta{long} (\meta{abbrv})\meta{insert}
\end{display}
while subsequent use displays the entry as:
\begin{display}
\meta{abbrv}\meta{insert}
\end{display}
Entries will be sorted according to the short form.
\end{description}

\section{Displaying the List of Acronyms}
\label{sec:disploa}

The list of acronyms is just like any other type of glossary and can
be displayed on its own using
\ics{printglossary}"[type=\acronymtype]" or with all the other
glossaries using \ics{printglossaries}. However, care must be taken
to choose a glossary style that's appropriate to your acronym style.
The different acronym-related package options store different
information in the \gloskey{name}, \gloskey{description} and
\gloskey{symbol} keys.

\Tableref{tab:acronymsettings} lists the package options that govern
the acronym styles and how the information is stored in the keys
used when displaying the glossary. Note that the \pkgopt{description} 
package option uses the following command in the \gloskey{name}:
\begin{definition}[\DescribeMacro{\acrnameformat}]
\cs{acrnameformat}\marg{abbrv}\marg{long}
\end{definition}
This defaults to just \cs{acronymfont}\marg{abbrv}.

For example, if I use the package options \pkgopt{description} and
\pkgopt{footnote}, then the name field will contain the abbreviation
and the symbol field will contain the long form. If I then use the 
\glostyle{long4col} style, each entry in the list of acronyms will
appear in the form:
\begin{display}
\ics{acronymfont}\marg{abbrv} \meta{description} \meta{long}
\meta{location list}
\end{display}
Alternatively, you can define your own custom style (see 
\sectionref{sec:newglossarystyle} for further details).

\begin{table}[htbp]
\caption{Package options governing \cs{newacronym} and how the 
information is stored}
\label{tab:acronymsettings}
\begin{center}
\begin{tabular}{lllll}
\bfseries Package Option &
\gloskey{name} &
\gloskey{description} &
\gloskey{symbol}\\
\pkgopt{description},\pkgopt{footnote} &
\ics{acronymfont}\marg{abbrv} &
user supplied &
\meta{long}\\
\pkgopt{description},\pkgopt{dua} &
\meta{long} &
user supplied &
\meta{abbrv}\\
\pkgopt{description} &
\ics{acrnameformat}\marg{abbrv}\marg{long} &
user supplied &
\meta{abbrv}\\
\pkgopt{footnote} &
\ics{acronymfont}\marg{abbrv} &
\meta{long} &
 \\
\pkgopt{smallcaps} &
\ics{acronymfont}\marg{abbrv} &
\meta{long} &
\meta{abbrv}\\
\pkgopt{smaller} &
\ics{acronymfont}\marg{abbrv} &
\meta{long} &
\meta{abbrv}\\
\pkgopt{dua} &
\meta{abbrv} &
\meta{long} &
\meta{abbrv}\\
None of the above&
\meta{abbrv} &
\meta{long} &
\end{tabular}
\end{center}
\end{table}

\section{Defining A Custom Acronym Style}
\label{sec:customacronym}

You may find that the predefined acronyms styles that come with the
\styfmt{glossaries} package don't suit your requirements. In this
case you can define your own style. This is done by redefining the
following commands:
\begin{definition}[\DescribeMacro{\CustomAcronymFields}]
\cs{CustomAcronymFields}
\end{definition}
This command sets up the keys for \ics{newglossaryentry} when
you define an acronym using \ics{newacronym}. Within the definition
of \cs{CustomAcronymFields}, you may use \cs{the}\ics{glslongtok} to
access the long form, \cs{the}\ics{glsshorttok} to access the short
form and \cs{the}\ics{glslabeltok} to access the label. This command
is typically used to set the \gloskey{name}, \gloskey{first},
\gloskey{firstplural}, \gloskey{text} and \gloskey{plural} keys. It
may also be used to set the \gloskey{symbol} or
\gloskey{description} keys depending on your requirements.

\begin{definition}[\DescribeMacro{\SetCustomDisplayStyle}]
\cs{SetCustomDisplayStyle}\marg{type}
\end{definition}
This is used to set up the display style for the glossary given by
\meta{type}. This should typically just use \ics{defglsdisplayfirst}
and \ics{defglsdisplay}.

Once you have redefined \cs{CustomAcronymFields} and
\linebreak\cs{SetCustomDisplayStyle}, you must then switch to this style using
\begin{definition}[\DescribeMacro{\SetCustomStyle}]
\cs{SetCustomStyle}
\end{definition}
Note that you may still use the \pkgopt{shortcuts} package option
with your custom style.

\begin{important}
If you omit \cs{SetCustomStyle}, or use it before you redefine
\cs{CustomAcronymFields} and \cs{SetCustomDisplayStyle}, your new
style won't be correctly implemented. You must set up the custom style 
before defining new acronyms. The acronyms must be defined using
\ics{newacronym} not \ics{newglossaryentry}.
\end{important}

As an example, suppose I want my acronym on \gls{firstuse} to have the
short form in the text and the long form with the description in a
footnote. Suppose also that I want the short form to be put in small
caps in the main body of the document, but I want it in normal
capitals in the list of acronyms. In my list of acronyms, I want the
long form as the name with the short form in brackets followed by
the description. That is, in the text I want \ics{gls} on \gls{firstuse}
to display:
\begin{display}
\ics{textsc}\marg{abbrv}\cs{footnote}"{"\meta{long}: \meta{description}"}"
\end{display}
on subsequent use:
\begin{display}
\ics{textsc}\marg{abbrv}
\end{display}
and in the list of acronyms, each entry will be displayed in the
form:
\begin{display}
\meta{long} (\meta{short}) \meta{description}
\end{display}

First, I need to redefine \ics{CustomAcronymFields} so that
\ics{newacronym} will correctly set the \gloskey{name}, 
\gloskey{text} and \gloskey{plural} keys. I want the long form to be
stored in the \gloskey{name} and the short form to be stored in
\gloskey{text}. In addition, I'm going to set the \gloskey{symbol}
to the short form in upper case so that it will appear in the list
of acronyms.

\begin{verbatim}
\renewcommand*{\CustomAcronymFields}{%
  name={\the\glslongtok},%
  symbol={\MakeUppercase{\the\glsshorttok}},%
  text={\textsc{\the\glsshorttok}},%
  plural={\textsc{\the\glsshorttok}\noexpand\acrpluralsuffix}%
}
\end{verbatim}

When using \ics{newacronym}, the short and long forms are stored in
the \gloskey{short} and \gloskey{long} keys, and the plural forms
are stored in \gloskey{shortplural} and \gloskey{longplural}.
So when I use \ics{defglsdisplayfirst}
and \ics{defglsdisplay}, I can use \ics{glsentrylong} to access
the long form. Recall from \sectionref{sec:glsdisplay}, that the
optional argument to \ics{defglsdisplayfirst} and \ics{defglsdisplay}
indicates the glossary type. This is passed to
\ics{SetCustomDisplayStyle}. The mandatory argument sets up the
definition of \ics{glsdisplayfirst} and \ics{glsdisplay} for the
given glossary, where the first argument corresponds to the
\gloskey{first}, \gloskey{firstplural}, \gloskey{text} or 
\gloskey{plural}, as appropriate, the second argument corresponds to
the \gloskey{description}, the third corresponds to the
\gloskey{symbol} and the fourth argument is the inserted text.

\begin{verbatim}
\renewcommand*{\SetCustomDisplayStyle}[1]{%
  \defglsdisplayfirst[#1]{##1##4\protect\footnote{%
    \glsentrylong{\glslabel}: ##2%
  }}
  \defglsdisplay[#1]{##1##4}%
}
\end{verbatim}

Since we have a definition inside a definition, \verb|#1| refers to
the argument of \ics{SetCustomDisplayStyle}, and \verb|##1|, \ldots,
\verb|##4|, refer to the arguments of \ics{glsdisplayfirst} and
\ics{glsdisplay}.

Now that I've redefined \ics{CustomAcronymFields} and
\linebreak\ics{SetCustomDisplayStyle}, I can set this style using
\begin{verbatim}
\SetCustomStyle
\end{verbatim}
and now I can define my acronyms:
\begin{verbatim}
\newacronym[description={set of tags for use in 
developing hypertext documents}]{html}{html}{Hyper 
Text Markup Language}

\newacronym[description={language used to describe the 
layout of a document written in a markup language}]{css}
{css}{Cascading Style Sheet}
\end{verbatim}

Note that since I've used the description in the main body of the
text, I need to switch off the sanitization otherwise any commands
within the description won't get interpreted. Also I want to use the
\sty{hyperref} package, but this will cause a problem on \gls{firstuse}
as I'll get nested hyperlinks, so I need to switch off the
hyperlinks on \gls{firstuse}. In addition, I want to use a glossary style
that displays the symbol. Therefore, in my preamble I have:

\begin{verbatim}
\usepackage[colorlinks]{hyperref}
\usepackage
 [acronym,         % create list of acronyms
  nomain,          % don't need main glossary for this example
  style=tree,      % need a style that displays the symbol
  hyperfirst=false,% don't hyperlink first use
  sanitize=none    % switch off sanitization as description 
                   % will be used in the main text
 ]{glossaries}
\end{verbatim}

Note that I haven't used the \pkgopt{description} or
\pkgopt{footnote} package options.

\section{Upgrading From the glossary Package}
\label{sec:oldacronym}

Users of the obsolete \sty{glossary} package may recall that
the syntax used to define new acronyms has changed with the
replacement \styfmt{glossaries} package. In addition, the old
\sty{glossary} package created the command 
\cs{}\meta{acr-name} when defining the acronym \meta{acr-name}.

In order to facilitate migrating from the old package to the new
one, the \styfmt{glossaries} package\footnote{as from version 1.18} 
provides the command:
\begin{definition}[\DescribeMacro{\oldacronym}]
\cs{oldacronym}\oarg{label}\marg{abbrv}\marg{long}\marg{key-val list}
\end{definition}
This uses the same syntax as the \sty{glossary} package's
method of defining acronyms. It is equivalent to:\\[10pt]
\ics{newacronym}\oarg{key-val list}\marg{label}\marg{abbrv}\marg{long}\\[10pt]
In addition, \ics{oldacronym} also defines the commands
\cs{}\meta{label}, which is equivalent to \ics{gls}\marg{label},
and \cs{}\meta{label}\texttt{*}, which is equivalent to
\ics{Gls}\marg{label}. If \meta{label} is omitted, \meta{abbrv}
is used. Since commands names must consist only of alphabetical
characters, \meta{label} must also only consist of alphabetical
characters. Note that \cs{}\meta{label} doesn't allow you to use
the first optional argument of \ics{gls} or \ics{Gls} --- you will
need to explicitly use \ics{gls} or \ics{Gls} to change the
settings.

\begin{important}
Recall that, in general, \LaTeX\ ignores spaces following command 
names consisting of alphabetical characters. This is also true for 
\cs{}\meta{label} unless you additionally load the
\sty{xspace} package.
\end{important}

The \styfmt{glossaries} package doesn't load the \sty{xspace} package
since there are both advantages and disadvantages to using
\ics{xspace} in \cs{}\meta{label}. If you don't use the 
\sty{xspace} package you need to explicitly force a space using
\verb*|\ | (backslash space) however you can follow \cs{}\meta{label}
with additional text in square brackets (the final optional
argument to \ics{gls}). If you use the \sty{xspace} package
you don't need to escape the spaces but you can't use
the optional argument to insert text (you will have to explicitly
use \ics{gls}).

To illustrate this, suppose I define the acronym \qt{abc} as
follows:
\begin{verbatim}
\oldacronym{abc}{example acronym}{}
\end{verbatim}
This will create the command \cs{abc} and its starred version
\cs{abc*}. \Tableref{tab:xspace} illustrates the effect of
\cs{abc} (on subsequent use) according to whether or not the
\sty{xspace} package has been loaded. As can be seen from the
final row in the table, the \sty{xspace} package prevents the
optional argument from being recognised.

\begin{table}[htbp]
\caption[The effect of using xspace]{The effect of using 
\styfmt{xspace} with \cs{oldacronym}}
\label{tab:xspace}
\vskip\baselineskip
\centering
\begin{tabular}{lll}
\bfseries Code & \bfseries With \styfmt{xspace} &
\bfseries Without \styfmt{xspace}\\
\verb|\abc.| & abc. & abc.\\
\verb|\abc xyz| & abc xyz & abcxyz\\
\verb|\abc\ xyz| & abc xyz & abc xyz\\
\verb|\abc* xyz| & Abc xyz & Abc xyz\\
\verb|\abc['s] xyz| & abc ['s] xyz & abc's xyz
\end{tabular}
\par
\end{table}



\chapter{Unsetting and Resetting Entry Flags}
\label{sec:glsunset}

When using \ics{gls}, \ics{glspl} and their uppercase variants it is
possible that you may want to use the value given by the
\gloskey{first} key, even though you have already \glslink{firstuse}{used} the glossary
entry. Conversely, you may want to use the value given by the
\gloskey{text} key, even though you haven't used the glossary entry.
The former can be achieved by one of the following commands:
\begin{definition}[\DescribeMacro{\glsreset}]
\cs{glsreset}\marg{label}
\end{definition}
\begin{definition}[\DescribeMacro{\glslocalreset}]
\cs{glslocalreset}\marg{label}
\end{definition}
while the latter can be achieved by one of the following commands:
\begin{definition}[\DescribeMacro{\glsunset}]
\cs{glsunset}\marg{label}
\end{definition}
\begin{definition}[\DescribeMacro{\glslocalunset}]
\cs{glslocalunset}\marg{label}
\end{definition}
You can also reset or unset all entries for a given glossary or list of
glossaries using:
\begin{definition}[\DescribeMacro{\glsresetall}]
\cs{glsresetall}\oarg{glossary list}
\end{definition}
\begin{definition}[\DescribeMacro{\glslocalresetall}]
\cs{glslocalresetall}\oarg{glossary list}
\end{definition}
\begin{definition}[\DescribeMacro{\glsunsetall}]
\cs{glsunsetall}\oarg{glossary list}
\end{definition}
\begin{definition}[\DescribeMacro{\glslocalunsetall}]
\cs{glslocalunsetall}\oarg{glossary list}
\end{definition}
where \meta{glossary list} is a comma-separated list of 
glossary labels. If omitted, all defined glossaries are assumed.
For example, to reset all entries in the main glossary and the
list of acronyms:
\begin{verbatim}
\glsresetall[main,acronym]
\end{verbatim}

You can determine whether an entry's \gls{firstuseflag} is set using:
\begin{definition}[\DescribeMacro{\ifglsused}]
\cs{ifglsused}\marg{label}\marg{true part}\marg{false part}
\end{definition}
where \meta{label} is the label of the required entry. If the
entry has been used, \meta{true part} will be done, otherwise
\meta{false part} will be done.


\chapter{Glossary Styles}
\label{sec:styles}
The \styfmt{glossaries} package comes with some pre-defined
glossary styles. Note that the styles are suited to different
types of glossaries: some styles ignore the associated
symbol; some styles are not designed for hierarchical entries,
so they display sub-entries in the same way as they display top
level entries; some styles are designed for homographs, so they 
ignore the name for sub-entries. You should therefore pick a style 
that suits your type of glossary. See \tableref{tab:styles} for
a summary of the available styles. The predefined styles can
accommodate numbered level~0 (main) and level~1 entries. See the
package options \pkgopt{entrycounter}, \pkgopt{counterwithin} and
\pkgopt{subentrycounter} described in
\sectionref{sec:pkgopts-printglos}.

\begin{table}[htbp]
\caption[Glossary Styles]{Glossary Styles. An asterisk in the style 
name indicates anything that matches that doesn't match any
previously listed style (e.g.\ \texttt{long3col*}
matches \glostyle{long3col}, \glostyle{long3colheader}, 
\glostyle{long3colborder} and \glostyle{long3colheaderborder}).
A maximum level of 0 indicates a flat glossary (sub-entries
are displayed in the same way as main entries). Where the maximum
level is given as --- there is no limit, but note that 
\gls{makeindex} imposes a limit of 2 sub-levels. If the
homograph column is checked, then the name is not displayed for
sub-entries. If the symbol column is checked, then the symbol will
be displayed.}
\label{tab:styles}
\vskip\baselineskip
\centering
\begin{tabular}{lccc}
\bfseries Style & \bfseries Maximum Level &
\bfseries Homograph & \bfseries Symbol\\
\ttfamily listdotted & 0 & & \\
\ttfamily sublistdotted & 1 & & \\
\ttfamily list* & 1 & \tick & \\
\ttfamily altlist* & 1 & \tick & \\
\ttfamily long*3col* & 1 & \tick & \\
\ttfamily long4col* & 1 & \tick & \tick \\
\ttfamily altlong*4col* & 1 & \tick & \tick \\
\ttfamily long* & 1 & \tick & \\
\ttfamily super*3col* & 1 & \tick & \\
\ttfamily super4col* & 1 & \tick & \tick \\
\ttfamily altsuper*4col* & 1 & \tick & \tick \\
\ttfamily super* & 1 & \tick & \\
\ttfamily *index* & 2 & & \tick\\
\ttfamily treenoname* & --- & \tick & \tick\\
\ttfamily *tree* & --- & & \tick\\
\ttfamily *alttree* & --- & & \tick\\
\ttfamily inline & 1 & \tick &  
\end{tabular}
\par
\end{table}

The glossary style can be set using the \gloskey[printglossary]{style} key in the optional
argument to \ics{printglossary} or using the command:
\begin{definition}[\DescribeMacro{\glossarystyle}]
\cs{glossarystyle}\marg{style-name}
\end{definition}
Some of the glossary styles may also be set using the \pkgopt{style} 
package option, it depends if the package in which they are defined
is automatically loaded by the \styfmt{glossaries} package.

The tabular-like styles that allow multi-line descriptions and page
lists use the length \DescribeMacro{\glsdescwidth}\cs{glsdescwidth}
to set the width of the description column and the length
\DescribeMacro{\glspagelistwidth}\cs{glspagelistwidth} to set the
width of the page list column.\footnote{these lengths will not be
available if you use both the \pkgopt{nolong} and \pkgopt{nosuper}
package options or if you use the \pkgopt{nostyles} package option
unless you explicitly load the relevant package.}
These will need to be changed using \cs{setlength} if the glossary
is too wide. Note that the \glostyle{long4col} and
\glostyle{super4col} styles (and their header and border variations)
don't use these lengths as they are designed for single line
entries. Instead you should use the analogous \glostyle{altlong4col}
and \glostyle{altsuper4col} styles.  If you want to
explicitly create a line-break within a multi-line description in
a tabular-like style you should use \ics{newline} instead of
\verb|\\|.

Note that if you use the \gloskey[printglossary]{style} key in the
optional argument to \ics{printglossary}, it will override any 
previous style settings for the given glossary, so if, for example,
you do
\begin{verbatim}
\renewcommand*{\glsgroupskip}{}
\printglossary[style=long]
\end{verbatim}
then the new definition of \ics{glsgroupskip} will not have an affect
for this glossary, as \cs{glsgroupskip} is redefined by 
\verb|style=long|. Likewise, \ics{glossarystyle} will also
override any previous style definitions, so, again
\begin{verbatim}
\renewcommand*{\glsgroupskip}{}
\glossarystyle{long}
\end{verbatim}
will reset \cs{glsgroupskip} back to its default definition for the
named glossary style (\glostyle{long} in this case). If you want to 
modify the styles, either use \ics{newglossarystyle} (described
in the next section) or make the modifications after 
\ics{glossarystyle}, e.g.:
\begin{verbatim}
\glossarystyle{long}
\renewcommand*{\glsgroupskip}{}
\end{verbatim}
As from version 3.03, you can now use the package option
\pkgopt{nogroupskip} to suppress the gap between groups for the
default styles.

All the styles except for the three- and four-column styles and the
\glostyle{listdotted} style use the command
\begin{definition}[\DescribeMacro{\glspostdescription}]
\cs{glspostdescription}
\end{definition}
after the description. This simply displays a full stop by default.
To eliminate this full stop (or replace it with something else, say,
a comma) you will need to redefine \cs{glspostdescription} before
the glossary is displayed. Alternatively, you can suppress it for a
given entry by placing \ics{nopostdesc} in the entry's description.

As from version 3.03 you can now use the package option
\pkgopt{nopostdot} to suppress this full stop.

\section{List Styles}
\label{sec:liststyles}

The styles described in this section are all defined in the package
\sty{glossary-list}. Since they all use the \env{description}
environment, they are governed by the same parameters as that
environment. These styles all ignore the entry's symbol.  Note that
these styles will automatically be available unless you use the
\pkgopt{nolist} or \pkgopt{nostyles} package options.

\begin{description}
\item[list] The \glostyle{list} style uses the \env{description}
environment. The entry name is placed in the optional argument of
the \ics{item} command (so it will appear in bold by default). The
description follows, and then the associated \gls{numberlist} for 
that entry. The symbol is ignored.  If the entry has child entries,
the description and number list follows (but not the name) for each
child entry.  Groups are separated using \ics{indexspace}.

\item[listgroup] The \glostyle{listgroup} style is like 
\glostyle{list} but the glossary groups have headings.

\item[listhypergroup] The \glostyle{listhypergroup} style is like
\glostyle{listgroup} but has a navigation line at the start of the
glossary with links to each group that is present in the glossary.
This requires an additional run through \LaTeX\ to ensure the group
information is up to date. In the navigation line, each group is
separated by
\begin{definition}[\DescribeMacro{\glshypernavsep}]
\cs{glshypernavsep}
\end{definition}
which defaults to a vertical bar with a space on either side. For
example, to simply have a space separating each group, do:
\begin{verbatim}
\renewcommand*{\glshypernavsep}{\space}
\end{verbatim}

Note that the hyper-navigation line is now (as from version 1.14) 
set inside the optional argument to \ics{item} instead of after it
to prevent a spurious space at the start. This can be changed
by redefining \ics{glossaryheader}, but note that this needs to
be done \emph{after} the glossary style has been set.

\item[altlist] The \glostyle{altlist} style is like \glostyle{list}
but the description starts on the line following the name. (As
with the \glostyle{list} style, the symbol is ignored.) Each child
entry starts a new line, but as with the \glostyle{list} style,
the name associated with each child entry is ignored.

\item[altlistgroup] The \glostyle{altlistgroup} style is like 
\glostyle{altlist} but the glossary groups have headings.

\item[altlisthypergroup] The \glostyle{altlisthypergroup} style is like 
\glostyle{altlistgroup} but has a set of links to the glossary 
groups. The navigation line is the same as that for 
\glostyle{listhypergroup}, described above.

\item[listdotted] This style uses the \env{description}
environment.\footnote{This style was supplied by Axel~Menzel.} Each
entry starts with \verb|\item[]|, followed by the name followed by a
dotted line, followed by the description. Note that this style
ignores both the \gls{numberlist} and the symbol. The length
\begin{definition}[\DescribeMacro{\glslistdottedwidth}]
\cs{glslistdottedwidth}
\end{definition}
governs where the description should start. This is a flat style, so
child entries are formatted in the same way as the parent entries.

\item[sublistdotted] This is a variation on the \glostyle{listdotted}
style designed for hierarchical glossaries. The main entries
have just the name displayed. The sub entries are displayed in
the same manner as \glostyle{listdotted}.
\end{description}


\section{Longtable Styles}
\label{sec:longstyles}

The styles described in this section are all defined in the package
\sty{glossary-long}. Since they all use the \env{longtable}
environment, they are governed by the same parameters as that
environment.  Note that these styles will automatically be available
unless you use the \pkgopt{nolong} or \pkgopt{nostyles} package
options.  These styles fully justify the description and page list
columns.  If you want ragged right formatting instead, use the
analogous styles described in \sectionref{sec:longraggedstyles}.


\begin{description}
\item[long] The \glostyle{long} style uses the \env{longtable}
environment (defined by the \sty{longtable} package). It has two
columns: the first column contains the entry's name and the second
column contains the description followed by the \gls{numberlist}.
The entry's symbol is ignored.
Sub groups are separated with a blank row. The width of the
first column is governed by the widest entry in that column. The
width of the second column is governed by the length
\ics{glsdescwidth}. Child entries have a similar format to the
parent entries except that their name is suppressed.

\item[longborder] The \glostyle{longborder} style is like
\glostyle{long} but has horizontal and vertical lines around it.

\item[longheader] The \glostyle{longheader} style is like
\glostyle{long} but has a header row.

\item[longheaderborder] The \glostyle{longheaderborder} style is
like \glostyle{longheader} but has horizontal and vertical lines
around it.

\item[long3col] The \glostyle{long3col} style is like
\glostyle{long} but has three columns. The first column contains
the entry's name, the second column contains the description
and the third column contains the \gls{numberlist}.
The entry's symbol is ignored. The width of the
first column is governed by the widest entry in that column, the
width of the second column is governed by the length
\ics{glsdescwidth}, and the width of the third column is governed by the
length \ics{glspagelistwidth}.

\item[long3colborder] The \glostyle{long3colborder} style is like
the \glostyle{long3col} style but has horizontal and vertical
lines around it.

\item[long3colheader] The \glostyle{long3colheader} style is like
\glostyle{long3col} but has a header row.

\item[long3colheaderborder] The \glostyle{long3colheaderborder} style is
like \glostyle{long3colheader} but has horizontal and vertical lines
around it.

\item[long4col] The \glostyle{long4col} style is like 
\glostyle{long3col} but has an additional column in which the
entry's associated symbol appears. This style is used for brief
single line descriptions. The column widths are governed by the
widest entry in the given column. Use \glostyle{altlong4col} for 
multi-line descriptions.

\item[long4colborder] The \glostyle{long4colborder} style is like
the \glostyle{long4col} style but has horizontal and vertical
lines around it.

\item[long4colheader] The \glostyle{long4colheader} style is like
\glostyle{long4col} but has a header row.

\item[long4colheaderborder] The \glostyle{long4colheaderborder} style is
like \glostyle{long4colheader} but has horizontal and vertical lines
around it.

\item[altlong4col] The \glostyle{altlong4col} style is like
\glostyle{long4col} but allows multi-line descriptions and page
lists.  The width of the description column is governed by the
length \ics{glsdescwidth} and the width of the page list column is
governed by the length \linebreak\ics{glspagelistwidth}. The widths of the
name and symbol columns are governed by the widest entry in the
given column.

\item[altlong4colborder] The \glostyle{altlong4colborder} style is like
the \glostyle{long4colborder} but allows multi-line descriptions and
page lists.

\item[altlong4colheader] The \glostyle{altlong4colheader} style is like
\glostyle{long4colheader} but allows multi-line descriptions and
page lists.

\item[altlong4colheaderborder] The \glostyle{altlong4colheaderborder} 
style is like \linebreak\glostyle{long4colheaderborder} but allows multi-line
descriptions and page lists.
\end{description}


\section{Longtable Styles (Ragged Right)}
\label{sec:longraggedstyles}

The styles described in this section are all defined in the package
\sty{glossary-longragged}. These styles are analogous to those
defined in \sty{glossary-long} but the multiline columns are left
justified instead of fully justified. Since these styles all use the
\env{longtable} environment, they are governed by the same
parameters as that environment. The \sty{glossary-longragged}
package additionally requires the \sty{array} package.  Note that
these styles will only be available if you explicitly load
\sty{glossary-longragged}:
\begin{verbatim}
\usepackage{glossaries}
\usepackage{glossary-longragged}
\end{verbatim}
Note that you can't set these styles using the \pkgopt{style}
package option since the styles aren't defined until after
the \styfmt{glossaries} package has been loaded.

\begin{description}
\item[longragged] The \glostyle{longragged} style has two
columns: the first column contains the entry's name and the second
column contains the (left-justified) description followed by the 
\gls{numberlist}.  The entry's symbol is ignored.
Sub groups are separated with a blank row. The width of the
first column is governed by the widest entry in that column. The
width of the second column is governed by the length
\ics{glsdescwidth}. Child entries have a similar format to the
parent entries except that their name is suppressed.

\item[longraggedborder] The \glostyle{longraggedborder} style is like
\glostyle{longragged} but has horizontal and vertical lines around it.

\item[longraggedheader] The \glostyle{longraggedheader} style is like
\glostyle{longragged} but has a header row.

\item[longraggedheaderborder] The \glostyle{longraggedheaderborder} 
style is like \glostyle{longraggedheader} but has horizontal and 
vertical lines around it.

\item[longragged3col] The \glostyle{longragged3col} style is like
\glostyle{longragged} but has three columns. The first column
contains the entry's name, the second column contains the (left
justified) description and the third column contains the (left
justified) \gls{numberlist}.  The entry's symbol is ignored. The
width of the first column is governed by the widest entry in that
column, the width of the second column is governed by the length
\ics{glsdescwidth}, and the width of the third column is governed by
the length \ics{glspagelistwidth}.

\item[longragged3colborder] The \glostyle{longragged3colborder}
style is like the \glostyle{longragged3col} style but has horizontal
and vertical lines around it.

\item[longragged3colheader] The \glostyle{longragged3colheader} style is like
\glostyle{longragged3col} but has a header row.

\item[longragged3colheaderborder] The \glostyle{longragged3colheaderborder} style is
like \glostyle{longragged3colheader} but has horizontal and vertical lines
around it.

\item[altlongragged4col] The \glostyle{altlongragged4col} style is
like \glostyle{longragged3col} but has an additional column in which
the entry's associated symbol appears. The width of the description
column is governed by the length \ics{glsdescwidth} and the width of
the page list column is governed by the length
\ics{glspagelistwidth}. The widths of the name and symbol columns
are governed by the widest entry in the given column.

\item[altlongragged4colborder] The
\glostyle{altlongragged4colborder} style is like the
\glostyle{altlongragged4col} but has horizontal and vertical lines
around it.

\item[altlongragged4colheader] The
\glostyle{altlongragged4colheader} style is like
\glostyle{altlongragged4col} but has a header row.

\item[altlongragged4colheaderborder] The
\glostyle{altlongragged4colheaderborder} style is like
\glostyle{altlongragged4colheader} but has horizontal and vertical
lines around it.

\end{description}


\section{Supertabular Styles}
\label{sec:superstyles}

The styles described in this section are all defined in the package
\sty{glossary-super}. Since they all use the \env{supertabular}
environment, they are governed by the same parameters as that
environment.  Note that these styles will automatically be available
unless you use the \pkgopt{nosuper} or \pkgopt{nostyles} package
options.  In general, the \env{longtable} environment is better,
but there are some circumstances where it is better to use
\env{supertabular}.\footnote{e.g.\ with the \sty{flowfram}
package.} These styles fully justify the description and page list
columns.  If you want ragged right formatting instead, use the
analogous styles described in \sectionref{sec:superraggedstyles}.

\begin{description}
\item[super] The \glostyle{super} style uses the \env{supertabular}
environment (defined by the \sty{supertabular} package). It has two
columns: the first column contains the entry's name and the second
column contains the description followed by the \gls{numberlist}.
The entry's symbol is ignored.
Sub groups are separated with a blank row. The width of the
first column is governed by the widest entry in that column. The
width of the second column is governed by the length
\ics{glsdescwidth}. Child entries have a similar format to the
parent entries except that their name is suppressed.

\item[superborder] The \glostyle{superborder} style is like
\glostyle{super} but has horizontal and vertical lines around it.

\item[superheader] The \glostyle{superheader} style is like
\glostyle{super} but has a header row.

\item[superheaderborder] The \glostyle{superheaderborder} style is
like \glostyle{superheader} but has horizontal and vertical lines
around it.

\item[super3col] The \glostyle{super3col} style is like
\glostyle{super} but has three columns. The first column contains
the entry's name, the second column contains the description
and the third column contains the \gls{numberlist}. The
entry's symbol is ignored. The width of the
first column is governed by the widest entry in that column. The
width of the second column is governed by the length
\ics{glsdescwidth}. The width of the third column is governed by the
length \ics{glspagelistwidth}.

\item[super3colborder] The \glostyle{super3colborder} style is like
the \glostyle{super3col} style but has horizontal and vertical
lines around it.

\item[super3colheader] The \glostyle{super3colheader} style is like
\glostyle{super3col} but has a header row.

\item[super3colheaderborder] The \glostyle{super3colheaderborder} style
is like the \linebreak\glostyle{super3colheader} style but has horizontal and vertical
lines around it.

\item[super4col] The \glostyle{super4col} style is like 
\glostyle{super3col} but has an additional column in which the
entry's associated symbol appears. This style is designed for
entries with brief single line descriptions. The column widths are governed by the
widest entry in the given column. Use \glostyle{altsuper4col}
for multi-line descriptions.

\item[super4colborder] The \glostyle{super4colborder} style is like
the \glostyle{super4col} style but has horizontal and vertical
lines around it.

\item[super4colheader] The \glostyle{super4colheader} style is like
\glostyle{super4col} but has a header row.

\item[super4colheaderborder] The \glostyle{super4colheaderborder} style
is like the \linebreak\glostyle{super4colheader} style but has horizontal and vertical
lines around it.

\item[altsuper4col] The \glostyle{altsuper4col} style is like 
\glostyle{super4col} but allows multi-line descriptions and page
lists.
The width of the description column is governed by the length
\ics{glsdescwidth} and the width of the page list column is
governed by the length \ics{glspagelistwidth}. The width of the name 
and symbol columns is governed by the widest entry in the 
given column.

\item[altsuper4colborder] The \glostyle{altsuper4colborder} style is like
the \glostyle{super4colborder} style but allows multi-line descriptions
and page lists.

\item[altsuper4colheader] The \glostyle{altsuper4colheader} style is like
\glostyle{super4colheader} but allows multi-line descriptions and
page lists.

\item[altsuper4colheaderborder] The \glostyle{altsuper4colheaderborder}
style is like \glostyle{super4colheaderborder} but allows multi-line
descriptions and page lists.
\end{description}


\section{Supertabular Styles (Ragged Right)}
\label{sec:superraggedstyles}

The styles described in this section are all defined in the package
\sty{glossary-superragged}. These styles are analogous to those
defined in \sty{glossary-super} but the multiline columns are left
justified instead of fully justified. Since these styles all use the
\env{supertabular} environment, they are governed by the same
parameters as that environment. The \sty{glossary-superragged}
package additionally requires the \sty{array} package.  Note that
these styles will only be available if you explicitly load
\sty{glossary-superragged}:
\begin{verbatim}
\usepackage{glossaries}
\usepackage{glossary-superragged}
\end{verbatim}
Note that you can't set these styles using the \pkgopt{style}
package option since the styles aren't defined until after
the \styfmt{glossaries} package has been loaded.

\begin{description}
\item[superragged] The \glostyle{superragged} style uses the
\env{supertabular} environment (defined by the
\sty{supertabular} package). It has two columns: the first column
contains the entry's name and the second column contains the (left
justified) description followed by the \gls{numberlist}.  The
entry's symbol is ignored.  Sub groups are separated with a blank
row. The width of the first column is governed by the widest entry
in that column. The width of the second column is governed by the
length \ics{glsdescwidth}. Child entries have a similar format to
the parent entries except that their name is suppressed.

\item[superraggedborder] The \glostyle{superraggedborder} style is
like \glostyle{superragged} but has horizontal and vertical lines
around it.

\item[superraggedheader] The \glostyle{superraggedheader} style is
like \glostyle{superragged} but has a header row.

\item[superraggedheaderborder] The
\glostyle{superraggedheaderborder} style is like
\glostyle{superraggedheader} but has horizontal and vertical lines
around it.

\item[superragged3col] The \glostyle{superragged3col} style is like
\glostyle{superragged} but has three columns. The first column
contains the entry's name, the second column contains the (left
justified) description and the third column contains the (left
justified) \gls{numberlist}. The entry's symbol is ignored. The
width of the first column is governed by the widest entry in that
column. The width of the second column is governed by the length
\ics{glsdescwidth}. The width of the third column is governed by the
length \ics{glspagelistwidth}.

\item[superragged3colborder] The \glostyle{superragged3colborder}
style is like the \glostyle{superragged3col} style but has
horizontal and vertical lines around it.

\item[superragged3colheader] The \glostyle{superragged3colheader}
style is like \glostyle{superragged3col} but has a header row.

\item[superragged3colheaderborder] The
\glostyle{superragged3colheaderborder} style is like
\glostyle{superragged3colheader} but has horizontal and vertical
lines around it.

\item[altsuperragged4col] The \glostyle{altsuperragged4col} style is
like \glostyle{superragged3col} but has an additional column in
which the entry's associated symbol appears. The column widths for
the name and symbol column are governed by the widest entry in the
given column.

\item[altsuperragged4colborder] The
\glostyle{altsuperragged4colborder} style is like the
\glostyle{altsuperragged4col} style but has horizontal and vertical
lines around it.

\item[altsuperragged4colheader] The
\glostyle{altsuperragged4colheader} style is like
\glostyle{altsuperragged4col} but has a header row.

\item[altsuperragged4colheaderborder] The
\glostyle{altsuperragged4colheaderborder} style is like
\glostyle{altsuperragged4colheader} but has horizontal and vertical
lines around it.

\end{description}

\section{Tree-Like Styles}
\label{sec:treestyles}

The styles described in this section are all defined in the package
\sty{glossary-tree}. These styles are designed for hierarchical
glossaries but can also be used with glossaries that don't have
sub-entries. These styles will display the entry's symbol if it
exists. Note that these styles will automatically be available
unless you use the \pkgopt{notree} or \pkgopt{nostyles} package
options.

\begin{description}
\item[index] The \glostyle{index} style is similar to the way
indices are usually formatted in that it has a hierarchical
structure up to three levels (the main level plus two sub-levels).
The name is typeset in bold, and if the symbol is present it is set
in parentheses after the name and before the description.
Sub-entries are indented and also include the name, the symbol
in brackets (if present) and the description.  Groups are separated
using \ics{indexspace}.

\item[indexgroup] The \glostyle{indexgroup} style is similar to
the \glostyle{index} style except that each group has a heading.

\item[indexhypergroup] The \glostyle{indexhypergroup} style is like 
\glostyle{indexgroup} but has a set of links to the glossary 
groups. The navigation line is the same as that for 
\glostyle{listhypergroup}, described above.

\item[tree] The \glostyle{tree} style is similar to the
\glostyle{index} style except that it can have arbitrary levels.
(Note that \gls{makeindex} is limited to three levels, so
you will need to use \gls{xindy} if you want more than 
three levels.) Each sub-level is indented by 
\DescribeMacro{\glstreeindent}\cs{glstreeindent}. Note that the
name, symbol (if present) and description are placed in the
same paragraph block. If you want the name to be apart from the
description, use the \glostyle{alttree} style instead. (See below.)

\item[treegroup] The \glostyle{treegroup} style is similar to
the \glostyle{tree} style except that each group has a heading.

\item[treehypergroup] The \glostyle{treehypergroup} style is like 
\glostyle{treegroup} but has a set of links to the glossary 
groups. The navigation line is the same as that for 
\glostyle{listhypergroup}, described above.

\item[treenoname] The \glostyle{treenoname} style is like
the \glostyle{tree} style except that the name for each sub-entry
is ignored.

\item[treenonamegroup] The \glostyle{treenonamegroup} style is 
similar to the \glostyle{treenoname} style except that each group 
has a heading.

\item[treenonamehypergroup] The \glostyle{treenonamehypergroup} 
style is like \glostyle{treenonamegroup} but has a set of links to 
the glossary groups. The navigation line is the same as that for 
\glostyle{listhypergroup}, described above.

\item[alttree] The \glostyle{alttree} style is similar to the
\glostyle{tree} style except that the indentation for each level
is determined by the width of the text specified by
\begin{definition}[\DescribeMacro{\glssetwidest}]
\cs{glssetwidest}\oarg{level}\marg{text}
\end{definition}
The optional argument \meta{level} indicates the level, where
0 indicates the top-most level, 1 indicates the first level 
sub-entries, etc. If \cs{glssetwidest} hasn't been used for a 
given sub-level, the level~0 widest text is used instead. If
\meta{level} is omitted, 0 is assumed.

For each level, the name is placed to the left of the paragraph
block containing the symbol (optional) and the description. If the
symbol is present, it is placed in parentheses before the
description.

\item[alttreegroup] The \glostyle{alttreegroup} is like the
\glostyle{alttree} style except that each group has a heading.

\item[alttreehypergroup] The \glostyle{alttreehypergroup} style is 
like \glostyle{alttreegroup} but has a set of links to the glossary 
groups. The navigation line is the same as that for 
\glostyle{listhypergroup}, described above.

\end{description}

\section{Multicols Style}
\label{sec:mcolstyles}

The \sty{glossary-mcols} package provides tree-like styles that are
in the \env{multicols} environment (defined by the \sty{multicol}
package). The style names are as their analogous tree styles (as
defined in \sectionref{sec:treestyles}) but are prefixed with ``mcol''.
For example, the \glostyle{mcolindex} style is essentially the
\glostyle{index} style but put in a \env{multicols} environment.
For the complete list, see \tableref{tab:mcols}.

\begin{important}
Note that \sty{glossary-mcols} is not loaded by \styfmt{glossaries}. If
you want to use any of the multicol styles in that package you need
to load it explicitly with \cs{usepackage} and set the required glossary
style using \ics{glossarystyle}.
\end{important}

The default number of columns is 2, but can be changed by redefining
\begin{definition}[\DescribeMacro{\glsmcols}]
\cs{glsmcols}
\end{definition}
to the required number. For example, for a three column glossary:
\begin{verbatim}
\usepackage{glossary-mcols}
\renewcommand*{\glsmcols}{3}
\glossarystyle{mcolindex}
\end{verbatim}

\begin{table}[htbp]
\caption{Multicolumn Styles}
\label{tab:mcols}
\centering
\begin{tabular}{ll}
\bfseries
\sty{glossary-mcols} Style &
\bfseries
Analogous Tree Style\\
\glostyle{mcolindex} & \glostyle{index}\\
\glostyle{mcolindexgroup} & \glostyle{indexgroup}\\
\glostyle{mcolindexhypergroup} & \glostyle{indexhypergroup}\\
\glostyle{mcoltree} & \glostyle{tree}\\
\glostyle{mcoltreegroup} & \glostyle{treegroup}\\
\glostyle{mcoltreehypergroup} & \glostyle{treehypergroup}\\
\glostyle{mcoltreenoname} & \glostyle{treenoname}\\
\glostyle{mcoltreenonamegroup} & \glostyle{treenonamegroup}\\
\glostyle{mcoltreenonamehypergroup} & \glostyle{treenonamehypergroup}\\
\glostyle{mcolalttree} & \glostyle{alttree}\\
\glostyle{mcolalttreegroup} & \glostyle{alttreegroup}\\
\glostyle{mcolalttreehypergroup} & \glostyle{alttreehypergroup}
\end{tabular}
\end{table}

\section{In-Line Style}
\label{sec:inline}

This section covers the \sty{glossary-inline} package that supplies
the \glostyle{inline} style. This is a style that is designed for
in-line use (as opposed to block styles, such as lists or tables).
This style doesn't display the \gls{numberlist}.

You will most likely need to redefine \ics{glossarysection} with
this style. For example, suppose you are required to have your
glossaries and list of acronyms in a footnote, you can do:
\begin{verbatim}
 \usepackage{glossary-inline}

 \renewcommand*{\glossarysection}[2][]{\textbf{#1}: }
 \glossarystyle{inline}
\end{verbatim}

\begin{important}
Note that you need to include \sty{glossary-inline} with
\cs{usepackage} as it's not automatically included by the
\styfmt{glossaries} package and then set the style using 
\ics{glossarystyle}.
\end{important}

Where you need to include your glossaries as a footnote you can do:
\begin{verbatim}
\footnote{\printglossaries}
\end{verbatim}

The \glostyle{inline} style is governed by the following:
\begin{definition}[\DescribeMacro{\glsinlineseparator}]
\ics{glsinlineseparator}
\end{definition}
This defaults to ``\texttt{\glsinlineseparator}'' and is used between
main (i.e.\ level~0) entries.
\begin{definition}[\DescribeMacro{\glsinlinesubseparator}]
\ics{glsinlinesubseparator}
\end{definition}
This defaults to ``\texttt{\glsinlinesubseparator}'' and is used between
sub-entries.
\begin{definition}[\DescribeMacro{\glsinlineparentchildseparator}]
\ics{glsinlineparentchildseparator}
\end{definition}
This defaults to ``\texttt{\glsinlineparentchildseparator}'' and is used between
a parent main entry and its first sub-entry.
\begin{definition}[\DescribeMacro{\glspostinline}]
\ics{glspostinline}
\end{definition}
This defaults to ``\texttt{\glsinlineseparator}'' and is used at the end
of the glossary.

\chapter{Defining your own glossary style}
\label{sec:newglossarystyle}

If the predefined styles don't fit your requirements, you can
define your own style using:
\begin{definition}[\DescribeMacro{\newglossarystyle}]
\cs{newglossarystyle}\marg{name}\marg{definitions}
\end{definition}
where \meta{name} is the name of the new glossary style (to be
used in \cs{glossarystyle}). The second argument \meta{definitions}
needs to redefine all of the following:

\begin{definition}[\DescribeEnv{theglossary}]
\env{theglossary}
\end{definition}
This environment defines how the main body of the glossary should
be typeset. Note that this does not include the section heading,
the glossary preamble (defined by \ics{glossarypreamble}) or the
glossary postamble (defined by \ics{glossarypostamble}). For example,
the \glostyle{list} style uses the \env{description} environment,
so the \env{theglossary} environment is simply redefined to begin
and end the \env{description} environment.

\begin{definition}[\DescribeMacro{\glossaryheader}]
\cs{glossaryheader}
\end{definition}
This macro indicates what to do at the start of the main body
of the glossary. Note that this is not the same as 
\ics{glossarypreamble}, which should not be affected by changes in
the glossary style. The \glostyle{list} glossary style redefines
\ics{glossaryheader} to do nothing, whereas the \glostyle{longheader}
glossary style redefines \cs{glossaryheader} to do a header row.

\begin{definition}[\DescribeMacro{\glsgroupheading}]
\cs{glsgroupheading}\marg{label}
\end{definition}
This macro indicates
what to do at the start of each logical block within the main body
of the glossary. If you use \gls{makeindex} the glossary is
sub-divided into a maximum of twenty-eight logical blocks that are
determined by the first character of the \gloskey{sort} key (or
\gloskey{name} key if the \gloskey{sort} key is omitted). The
sub-divisions are in the following order: symbols, numbers, A,
\ldots, Z\@. If you use \gls{xindy}, the sub-divisions depend on
the language settings.

Note that the argument to \cs{glsgroupheading}
is a label \emph{not} the group title. The group title can be obtained
via
\begin{definition}[\DescribeMacro{\glsgetgrouptitle}]
\cs{glsgetgrouptitle}\marg{label}
\end{definition}
This obtains the title as follows: if 
\cs{}\meta{label}\texttt{groupname} exists, this is taken to be the 
title, otherwise the title is just \meta{label}.

A navigation hypertarget can be created using
\begin{definition}[\DescribeMacro{\glsnavhypertarget}]
\cs{glsnavhypertarget}\marg{label}\marg{text}
\end{definition}
For further details about \cs{glsnavhypertarget}, see
\ifpdf section~\ref*{sec:code:hypernav} in \fi the documented code 
(\texttt{glossaries.pdf}).

Most of the predefined glossary styles redefine \cs{glsgroupheading}
to simply ignore its argument.  The \glostyle{listhypergroup} style
redefines \cs{glsgroupheading} as follows:
\begin{verbatim}
\renewcommand*{\glsgroupheading}[1]{%
\item[\glsnavhypertarget{##1}{\glsgetgrouptitle{##1}}]}
\end{verbatim}
See also \cs{glsgroupskip} below. (Note that command definitions within 
\ics{newglossarystyle} must use \verb|##1| instead of \verb|#1| etc.)

\begin{definition}[\DescribeMacro{\glsgroupskip}]
\cs{glsgroupskip}
\end{definition}
This macro determines what to do after one logical group but before
the header for the next logical group. The \glostyle{list} glossary
style simply redefines \cs{glsgroupskip} to be \ics{indexspace},
whereas the tabular-like styles redefine \cs{glsgroupskip} to
produce a blank row.

As from version 3.03, the package option \pkgopt{nogroupskip} can be
used to suppress this default gap for the predefined styles.

\begin{definition}[\DescribeMacro{\glossaryentryfield}]
\cs{glossaryentryfield}\marg{label}\marg{formatted name}\marg{description}
\marg{symbol}\marg{number list}
\end{definition}
This macro indicates what to do for a given glossary entry.
Note that \meta{formatted name} will always be in the form
\ics{glsnamefont}\marg{name}. This allows the user to set a given
font for the entry name, regardless of the glossary style used.
Note that \meta{label} is the label used when the glossary entry
was defined via either \ics{newglossaryentry} or \ics{newacronym}. 

\begin{definition}[\DescribeMacro{\glsentryitem}]
\cs{glsentryitem}\marg{label}
\end{definition}
This macro will increment and display the associated counter for the
main (level~0) entries if the \pkgopt{entrycounter} or
\pkgopt{counterwithin} package options have been used. This macro
is typically called by \cs{glossaryentryfield} before \cs{glstarget}. 
The format of the counter is controlled by the macro
\begin{definition}[\DescribeMacro{\glsentrycounterlabel}]
\cs{glsentrycounterlabel}
\end{definition}

Each time you use a glossary entry it creates a hyperlink (if
hyperlinks are enabled) to the relevant line in the glossary.
Your new glossary style must therefore redefine
\cs{glossaryentryfield} to set the appropriate target. This
is done using
\begin{definition}[\DescribeMacro{\glstarget}]
\cs{glstarget}\marg{label}\marg{text}
\end{definition}
where \meta{label} is the entry's label. Note that you
don't need to worry about whether the \sty{hyperref} package has
been loaded, as \cs{glstarget} won't create a target if
\cs{hypertarget} hasn't been defined.

For example, the 
\glostyle{list} style defines \cs{glossaryentryfield} as follows:
\begin{verbatim}
\renewcommand*{\glossaryentryfield}[5]{%
\item[\glsentryitem{##1}\glstarget{##1}{##2}] 
  ##3\glspostdescription \space ##5}
\end{verbatim}

Note also that \meta{number list} will always be of the form\\
\begin{definition}
\ics{glossaryentrynumbers}\{\cs{relax}\\
\ics{setentrycounter}\oarg{Hprefix}\marg{counter name}\meta{format cmd}\marg{number(s)}\}
\end{definition}
where \meta{number(s)}
may contain \ics{delimN} (to delimit individual numbers) and/or
\ics{delimR} (to indicate a range of numbers). There may be 
multiple occurrences of 
\ics{setentrycounter}\oarg{Hprefix}\marg{counter name}\meta{format cmd}\marg{number(s)}, but note
that the entire number list is enclosed within the argument
of \linebreak\ics{glossaryentrynumbers}. The user can redefine this to change
the way the entire number list is formatted, regardless of 
the glossary style. However the most common use of
\ics{glossaryentrynumbers} is to provide a means of suppressing the
number list altogether. (In fact, the \pkgopt{nonumberlist} option 
redefines \ics{glossaryentrynumbers} to ignore its argument.)
Therefore, when you define a new glossary style, you don't need
to worry about whether the user has specified the 
\pkgopt{nonumberlist} package option.

\begin{definition}[\DescribeMacro{\glossarysubentryfield}]
\cs{glossarysubentryfield}\marg{level}\marg{label}\marg{formatted name}
\marg{description}\marg{symbol}\marg{number list}
\end{definition}
This is new to version 1.17, and is used to display sub-entries.
The first argument, \meta{level}, indicates the sub-entry level.
This must be an integer from 1 (first sub-level) onwards. The
remaining arguments are analogous to those for
\ics{glossaryentryfield} described above.

\begin{definition}[\DescribeMacro{\glssubentryitem}]
\cs{glssubentryitem}\marg{label}
\end{definition}
This macro will increment and display the associated counter for the
level~1 entries if the \pkgopt{subentrycounter} package options have
been used. This macro is typically called by \cs{glossarysubentryfield}
before \ics{glstarget}. The format of the counter is controlled by
the macro
\begin{definition}[\DescribeMacro{\glssubentrycounterlabel}]
\cs{glssubentrycounterlabel}
\end{definition}

Note that \ics{printglossary} (which \ics{printglossaries} calls)
sets
\begin{definition}[\DescribeMacro{\currentglossary}]
\cs{currentglossary}
\end{definition}
to the current glossary label, so it's possible to create a glossary
style that varies according to the glossary type.

For further details of these commands, see \ifpdf
section~\ref*{sec:code:printglos} \fi \qt{Displaying the glossary}
in the documented code (\texttt{glossaries.pdf}).

\section{Example: creating a completely new style}
\label{sec:exnewstyle}

If you want a completely new style, you will need to redefine all
of the commands and the environment listed above. 

For example, suppose you want each entry to start with a bullet point.
This means that the glossary should be placed in the \env{itemize}
environment, so \env{theglossary} should start and end that
environment. Let's also suppose that you don't want anything between
the glossary groups (so \ics{glsgroupheading} and \ics{glsgroupskip}
should do nothing) and suppose you don't want anything to appear
immediately after \verb|\begin{theglossary}| (so \ics{glossaryheader}
should do nothing). In addition, let's suppose the symbol should
appear in brackets after the name, followed by the description and
last of all the \gls{numberlist} should appear within square brackets
at the end. Then you can create this new glossary style, called, say,
\texttt{mylist}, as follows:
\begin{verbatim}
 \newglossarystyle{mylist}{%
 % put the glossary in the itemize environment:
 \renewenvironment{theglossary}%
   {\begin{itemize}}{\end{itemize}}%
 % have nothing after \begin{theglossary}:
 \renewcommand*{\glossaryheader}{}%
 % have nothing between glossary groups:
 \renewcommand*{\glsgroupheading}[1]{}%
 \renewcommand*{\glsgroupskip}{}%
 % set how each entry should appear:
 \renewcommand*{\glossaryentryfield}[5]{%
 \item % bullet point
 \glstarget{##1}{##2}% the entry name
 \space (##4)% the symbol in brackets
 \space ##3% the description
 \space [##5]% the number list in square brackets
 }%
 % set how sub-entries appear:
 \renewcommand*{\glossarysubentryfield}[6]{%
   \glossaryentryfield{##2}{##3}{##4}{##5}{##6}}%
 }
\end{verbatim}
Note that this style creates a flat glossary, where sub-entries
are displayed in exactly the same way as the top level entries.
It also hasn't used \ics{glsentryitem} or \ics{glssubentryitem} so
it won't be affected by the \pkgopt{entrycounter},
\pkgopt{counterwithin} or \pkgopt{subentrycounter} package options.

\section{Example: creating a new glossary style based on an
existing style}
\label{sec:exadaptstyle}

If you want to define a new style that is a slightly modified
version of an existing style, you can use \ics{glossarystyle}
within the second argument of \ics{newglossarystyle} followed by
whatever alterations you require. For example, suppose you want 
a style like the \glostyle{list} style but you don't want the extra
vertical space created by \ics{indexspace} between groups, then you
can create a new glossary style called, say, \texttt{mylist} as
follows:
\begin{verbatim}
\newglossarystyle{mylist}{%
\glossarystyle{list}% base this style on the list style
\renewcommand{\glsgroupskip}{}% make nothing happen 
                              % between groups
}
\end{verbatim} 

\section{Example: creating a glossary style that uses the
\texorpdfstring{\gloskey{user1}}{user1}, \ldots, 
\texorpdfstring{\gloskey{user6}}{user6} keys}
\label{sec:exuserstyle}

Since \ics{glossaryentryfield} and \ics{glossarysubentryfield}
provide the label for the entry, it's also possible to access
the values of the generic user keys, such as \gloskey{user1}. For
example, suppose each entry not only has an associated symbol,
but also units (stored in \gloskey{user1}) and dimension
(stored in \gloskey{user2}). Then you can define a glossary style
that displays each entry in a \env{longtable} as follows:
\begin{verbatim}
\newglossarystyle{long6col}{%
 % put the glossary in a longtable environment:
 \renewenvironment{theglossary}%
  {\begin{longtable}{lp{\glsdescwidth}cccp{\glspagelistwidth}}}%
  {\end{longtable}}%
 % Set the table's header:
 \renewcommand*{\glossaryheader}{%
  \bfseries Term & \bfseries Description & \bfseries Symbol &
  \bfseries Units & \bfseries Dimensions & \bfseries Page List
  \\\endhead}%
 % No heading between groups:
  \renewcommand*{\glsgroupheading}[1]{}%
 % Main (level 0) entries displayed in a row optionally numbered:
  \renewcommand*{\glossaryentryfield}[5]{%
    \glsentryitem{##1}% Entry number if required
    \glstarget{##1}{##2}% Name
    & ##3% Description
    & ##4% Symbol
    & \glsentryuseri{##1}% Units
    & \glsentryuserii{##1}% Dimensions
    & ##5% Page list
    \\% end of row
  }%
 % Similarly for sub-entries (no sub-entry numbers):
 \renewcommand*{\glossarysubentryfield}[6]{%
    % ignoring first argument (sub-level)
    \glstarget{##2}{##3}% Name
    & ##4% Description
    & ##5% Symbol
    & \glsentryuseri{##2}% Units
    & \glsentryuserii{##2}% Dimensions
    & ##6% Page list
    \\% end of row
  }%
 % Nothing between groups:
 \renewcommand*{\glsgroupskip}{}%
}
\end{verbatim}

\chapter{Utilities}
\label{sec:utilities}

This section describes some utility commands. Additional commands
can be found in the documented code (glossaries.pdf).

\begin{definition}[\DescribeMacro{\forallglossaries}]
\cs{forallglossaries}\oarg{glossary list}\marg{cs}\marg{body}
\end{definition}
This iterates through \meta{glossary list}, a comma-separated list
of glossary labels (as supplied when the glossary was defined).
At each iteration \meta{cs} (which must be a control sequence) is
set to the glossary label for the current iteration and \meta{body}
is performed. If \meta{glossary list} is omitted, the default is to
iterate over all glossaries.

\begin{definition}[\DescribeMacro{\forglsentries}]
\cs{forglsentries}\oarg{glossary label}\marg{cs}\marg{body}
\end{definition}
This iterates through all entries in the glossary given by
\meta{glossary label}. At each iteration \meta{cs} (which must be a
control sequence) is set to the entry label for the current
iteration and \meta{body} is performed. If \meta{glossary label} is
omitted, \ics{glsdefaulttype} (usually the main glossary) is used.

\begin{definition}[\DescribeMacro{\forallglsentries}]
\cs{forallglsentries}\oarg{glossary list}\marg{cs}\marg{body}
\end{definition}
This is like \cs{forglsentries} but for each glossary in
\meta{glossary list} (a comma-separated list of glossary labels). If
\meta{glossary list} is omitted, the default is the list of all
defined glossaries. At each iteration \meta{cs} is set to the entry
label and \meta{body} is performed. (The current glossary label can
be obtained using \cs{glsentrytype}\marg{cs} within \meta{body}.)

\begin{definition}[\DescribeMacro{\ifglossaryexists}]
\cs{ifglossaryexists}{\meta{label}}{\meta{true part}}{\meta{false
part}}
\end{definition}
This checks if the glossary given by \meta{label} exists. If it
does \meta{true part} is performed, otherwise \meta{false part}.

\begin{definition}[\DescribeMacro{\ifglsentryexists}]
\cs{ifglsentryexists}{\meta{label}}{\meta{true part}}{\meta{false
part}}
\end{definition}
This checks if the glossary entry given by \meta{label} exists. If it
does \meta{true part} is performed, otherwise \meta{false part}.

\begin{definition}[\DescribeMacro{\ifglsused}]
\cs{ifglsentryexists}{\meta{label}}{\meta{true part}}{\meta{false
part}}
\end{definition}
See \sectionref{sec:glsunset}.

\begin{definition}[\DescribeMacro{\ifglshaschildren}]
\cs{ifglshaschildren}{\meta{label}}{\meta{true part}}{\meta{false
part}}
\end{definition}
This checks if the glossary entry given by \meta{label} has any
sub-entries. If it does, \meta{true part} is performed, otherwise
\meta{false part}.

\begin{definition}[\DescribeMacro{\ifglshasparent}]
\cs{ifglshasparent}{\meta{label}}{\meta{true part}}{\meta{false
part}}
\end{definition}
This checks if the glossary entry given by \meta{label} has a parent
entry. If it does, \meta{true part} is performed, otherwise
\meta{false part}.

\chapter{Accessibility Support}\label{sec:accsupp}
Limited accessibility support is provided by the accompanying
\sty{glossaries-accsupp} package, but note that this package is
experimental and it requires the \sty{accsupp} package 
which is also listed as experimental. This package defines
additional keys that may be used when defining glossary entries.
The keys are as follows:
\begin{description}
\item[{\gloskey{access}}] The replacement text corresponding to
the \gloskey{name} key.

\item[{\gloskey{textaccess}}] The replacement text corresponding
to the \gloskey{text} key.

\item[{\gloskey{firstaccess}}] The replacement text corresponding
to the \gloskey{first} key.

\item[{\gloskey{pluralaccess}}] The replacement text corresponding
to the \gloskey{plural} key.

\item[{\gloskey{firstpluralaccess}}] The replacement text corresponding
to the \gloskey{firstplural} key.

\item[{\gloskey{symbolaccess}}] The replacement text corresponding
to the \gloskey{symbol} key.

\item[{\gloskey{symbolpluralaccess}}] The replacement text corresponding
to the \gloskey{symbolplural} key.

\item[{\gloskey{descriptionaccess}}] The replacement text corresponding
to the \gloskey{description} key.

\item[{\gloskey{descriptionpluralaccess}}] The replacement text corresponding
to the \gloskey{descriptionplural} key.

\item[{\gloskey{longaccess}}] The replacement text corresponding to
the \gloskey{long} key (used by \ics{newacronym}).

\item[{\gloskey{shortaccess}}] The replacement text corresponding to
the \gloskey{short} key (used by \ics{newacronym}).

\item[{\gloskey{longpluralaccess}}] The replacement text corresponding to
the \gloskey{longplural} key (used by \ics{newacronym}).

\item[{\gloskey{shortpluralaccess}}] The replacement text corresponding to
the \gloskey{shortplural} key (used by \ics{newacronym}).

\end{description}
For example:
\begin{verbatim}
\newglossaryentry{tex}{name={\TeX},description={Document 
preparation language},access={TeX}}
\end{verbatim}
Now \verb|\gls{tex}| will be equivalent to
\begin{verbatim}
\BeginAccSupp{ActualText=TeX}\TeX\EndAccSupp{}
\end{verbatim}
The sample file \samplefile{accsupp} illustrates the
\sty{glossaries-accsupp} package.

See \ifpdf section~\ref*{sec:code:accsupp} in \fi the documented code
(\texttt{glossaries.pdf}) for further details. It is recommended
that you also read the \sty{accsupp} documentation. 

\chapter{Troubleshooting}
\label{sec:trouble}

The \styfmt{glossaries} package comes with a minimal file called 
\texttt{minimalgls.tex} which can be used for testing. This
should be located in the \texttt{samples} subdirectory (folder)
of the \styfmt{glossaries} documentation directory. The location
varies according to your operating system and \TeX\ installation.
For example, on my Linux partition it can be found in
\texttt{/usr/local/texlive/2008/texmf-dist\linebreak/doc/latex/glossaries/}. 
Further information on debugging \LaTeX\ code is available at
\url{http://theoval.cmp.uea.ac.uk/~nlct/latex/minexample/}.

Below is a list of the most frequently asked questions. For other
queries, consult the \styfmt{glossaries} FAQ at
\url{http://www.dickimaw-books.com/faqs/glossariesfaq.html}. If that
doesn't help, try posting your query to somewhere like the
comp.text.tex newsgroup, the \LaTeX\ Community Forum
(\url{http://www.latex-community.org/}) or Stack Exchange
(\url{http://tex.stackexchange.com/}). I read all those three places
and respond to queries there far quicker than to email messages (my
inbox is always very cluttered).
Bug reports can be submitted at
\url{http://www.dickimaw-books.com/bug-report.html}. 

\begin{enumerate}
\item \textbf{Q.} I get the error message:
\begin{verbatim}
Missing \begin{document}
\end{verbatim}

\textbf{A.} Check you are using an up to date version of the \sty{xkeyval}
package.

\item \textbf{Q.} When I use \gls{xindy}, I get the following error
message:
\begin{verbatim}
ERROR: CHAR: index 0 should be less than the length of 
the string
\end{verbatim}

\textbf{A.} \gls{xindy} discards all commands and braces from the
sort string. If your sort string (either specified by the \gloskey{sort}
key or the \gloskey{name} key) only consists of commands, this will
be treated by \gls*{xindy} as an empty sort string, which produces an
error message in newer versions of \gls*{xindy}. For example, the
following will cause a problem:
\begin{verbatim}
\newglossaryentry{alpha}{name={\ensuremath{\alpha}},
  description=alpha}
\end{verbatim}
Either use a different sort key for the entry, for example:
\begin{verbatim}
\newglossaryentry{alpha}{sort=alpha,
  name={\ensuremath{\alpha}},
  description=alpha}
\end{verbatim}
or, if all entries are like this, you may prefer to use the
\pkgopt[use]{sort} or \pkgopt[def]{sort} package options. See
\sectionref{sec:pkgopts-sort} for further details of the \pkgopt{sort}
option.

\item \textbf{Q.} I've used the \pkgopt{smallcaps} option, but the acronyms
are displayed in normal sized upper case letters.

\textbf{A.} The \pkgopt{smallcaps} package option uses \ics{textsc} to typeset
the acronyms. This command converts lower case letters to small 
capitals, while upper case letters remain their usual size. Therefore
you need to specify the acronym in lower case letters.

\item \textbf{Q.} My acronyms won't break across a line when they're
expanded.

\textbf{A.} PDF\LaTeX\ can break hyperlinks across a line, but
\LaTeX\ can't. If you can't use PDF\LaTeX\ then disable the \gls{firstuse}
links using the package option \pkgopt[false]{hyperfirst}.

\item \textbf{Q.} How do I change the font that the acronyms are displayed in?

\textbf{A.} The easiest way to do this is to specify the \pkgopt{smaller} package
option and redefine \ics{acronymfont} to use the required typesetting
command.  For example, suppose you would like the acronyms displayed in 
a sans-serif font, then you can do:
\begin{verbatim}
\usepackage[smaller]{glossaries}
\renewcommand*{\acronymfont}[1]{\textsf{#1}}
\end{verbatim}

\item \textbf{Q.} How do I change the font that the acronyms are displayed in
on \firstuse?

\textbf{A.} The easiest way to do this is to specify the \pkgopt{smaller} package
option and redefine \ics{firstacronymfont} to use the required 
command. Note that if you don't want the acronym on subsequent use
to use \ics{textsmaller}, you will also need to redefine \ics{acronymfont},
as above. For example to make the acronym emphasized on 
\firstuse, but use the surrounding font for subsequent use, you can do:
\begin{verbatim}
\usepackage[smaller]{glossaries}
\renewcommand*{\firstacronymfont}[1]{\emph{#1}}
\renewcommand*{\acronymfont}[1]{#1}
\end{verbatim}

\item \textbf{Q.} I don't have Perl installed, do I have to use 
\gls{makeglossaries}?

\textbf{A.} No. Although it is strongly recommended, you don't have to
use \gls{makeglossaries}. If you prefer a GUI application and have
Java installed, you can use \gls{makeglossariesgui} instead.
Otherwise you can just call \gls{makeindex} explicitly (see
\sectionref{sec:makeindexapp}). Note that you can't use \gls{xindy}
if you don't have Perl installed.

\item \textbf{Q.} I'm used to using the \styfmt{glossary} package: are there any
instructions on migrating from the \sty{glossary} package to the
\styfmt{glossaries} package?

\textbf{A.} Read \qtdocref{Upgrading from the glossary package to
the glossaries package}{glossary2glossaries} which should be
available from the same location as this document.

\item \textbf{Q.} I'm using \sty{babel} but the fixed names haven't
been translated.

\textbf{A.} The \styfmt{glossaries} package currently only supports
the following languages: Brazilian Portuguese, Danish, Dutch,
English, French, German, Irish, Italian, Hungarian, Polish, Serbian and
Spanish. If you want to add another language, send me the
translations, and I'll add them to the next version.

If you are using one of the above languages, but the text hasn't been
translated, try adding the \sty{translator} package with the
required languages explicitly (before you load the \styfmt{glossaries}
package). For example:
\begin{verbatim}
\usepackage[ngerman]{babel}
\usepackage[ngerman]{translator}
\usepackage{glossaries}
\end{verbatim}
Alternatively, you can add the language as a global option
to the class file. Check the \styfmt{translator} package documentation 
for further details.

\item \textbf{Q.} My acronyms contain strange characters when I use commands
like \ics{acrlong}.

\textbf{A.} Switch off the sanitization:
\begin{verbatim}
\usepackage[sanitize=none]{glossaries}
\end{verbatim}
and protect fragile commands.

\item \textbf{Q.} Weird characters appear when I use
\ics{glsentryname} or \ics{glsname}.

\textbf{A.} Either use \ics{glsentrytext} or \ics{glstext},
respectively, or switch off the sanitization for the \gloskey{name}
key:
\begin{verbatim}
\usepackage[sanitize={name=false}]{glossaries}
\end{verbatim}
and protect fragile commands.

\item \textbf{Q.} Weird characters appear when I try to display an
entry's description.

\textbf{A.} Switch off the sanitization for the \gloskey{description}
key:
\begin{verbatim}
\usepackage[sanitize={description=false}]{glossaries}
\end{verbatim}
and protect fragile commands.

\item \textbf{Q.} My glossaries haven't appeared.

\textbf{A.} Remember to do the following:
\begin{itemize}
\item Add \ics{makeglossaries} to the document preamble.

\item Use either \ics{printglossary} for each glossary that has
been defined or \ics{printglossaries}.

\item Use the commands listed in \sectionref{sec:glslink},
\sectionref{sec:glsadd} or \sectionref{sec:crossref} for each entry
that you want to appear in the glossary.

\item Run \LaTeX\ on your document, then run \gls{makeglossaries},
then run \LaTeX\ on your document again. If you want the glossaries
to appear in the table of contents, you will need an extra \LaTeX\
run. If any of your entries cross-reference an entry that's not
referenced in the main body of the document, you will need to run
\gls*{makeglossaries} (see \sectionref{sec:makeglossaries}) after the
second \LaTeX\ run, followed by another \LaTeX\ run.
\end{itemize}

Check the log files (\filetype{.log}, \filetype{.glg} etc) for any 
warnings.

\item \textbf{Q.} It is possible to change the rules used to sort
the glossary entries?

\textbf{A.} If it's for an individual entry, then you can use the
entry's \gloskey{sort} key to sort it according to a different term.
If it's for the entire alphabet, then you will need to use
\gls{xindy} (instead of \gls{makeindex}) and use an appropriate
\gls*{xindy} language module. Writing \gls*{xindy} modules or styles
is beyond the scope of this manual. Further information about
\gls*{xindy} can be found at the
\urlfootref{http://xindy.sourceforge.net/}{Xindy Web Site}. There is
also a link to the \gls*{xindy} mailing list from that site.

If you want to sort according to order of definition or order of
use, use the \pkgopt{sort} package option described in 
\sectionref{sec:pkgopts-sort}.

\item \textbf{Q.} I get an error when using TeX4HT\index{TeX4HT|hyperpage} with
\styfmt{glossaries}.

\textbf{A.} TeX4HT seems to have a problem with the glossary styles
that use \ics{indexspace}. I don't know enough about TeX4HT to find
out why. Either use a different glossary style or redefine the
style command that uses \cs{indexspace}. For example, if you are
using the \glostyle{list} style, try:
\begin{verbatim}
\renewcommand*{\glsgroupskip}{}
\end{verbatim}
or
\begin{verbatim}
\renewcommand*{\glsgroupskip}{\item[]}
\end{verbatim}

\end{enumerate}


\clearpage\phantomsection
\addcontentsline{toc}{chapter}{Index}\PrintIndex

\end{document}
